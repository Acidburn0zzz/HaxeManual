\documentclass{../haxe}

% todo-related
\usepackage[left=4.7cm, right=2cm, top=2cm, bottom=4.2cm]{geometry}
\usepackage[draft]{todonotes}
\reversemarginpar

% title (TODO: move this to class file once it looks good)

\renewcommand{\maketitle}{
   \begin{titlepage}
     \setcounter{page}{-1}
			\begin{center}
				~\\[3cm]
				\includegraphics[scale=1.25]{../assets/logo.pdf}~\\[1cm]
				{\huge \bfseries Haxe 3 Manual}\\[7cm]
				Haxe Foundation\\
				\today
			\end{center}
   \end{titlepage}
}


\input{../tikz}

% Conventions:

% run-time, compile-time
% Haxe, Haxelib (unless we are talking about the command itself)
% Haxe Standard Library, Haxe Compiler
% object-oriented

% code example width for ebooks: 47

\begin{document}
\title{Haxe 3 Manual}
\author{Haxe Foundation}
\date{\today}
\maketitle


\clearpage
\todototoc
\listoftodos
\clearpage

\clearpage
\tableofcontents
\clearpage

\chapter{Introduction}
\label{introduction}
\state{NoContent}

\section{What is Haxe?}
\label{introduction-what-is-haxe}

\todo{Could we have a big Haxe logo in the First Manual Page (Introduction) under the menu (a bit like a book cover ?) It looks a bit empty now and is a landing page for "Manual"}

Haxe consists of a high-level, open source programming language and a compiler. It allows compilation of programs, written using an ECMAScript\footnote{http://www.ecma-international.org/publications/standards/Ecma-327.htm}-oriented syntax, to multiple target languages. Employing proper abstraction, it is possible to maintain a single code-base which compiles to multiple targets.

Haxe is strongly typed but the typing system can be subverted where required. Utilizing type information, the Haxe type system can detect errors at compile-time which would only be noticeable at run-time in the target language. Furthermore, type information can be used by the target generators to generate optimized and robust code.

Currently, there are nine supported target languages which allow for different use-cases:

\begin{center}
\begin{tabular}{| l | l | l |}
	\hline
	Name & Output type & Main usages \\ \hline
	JavaScript & Sourcecode & Browser, Desktop, Mobile, Server \\
	Neko & Bytecode & Desktop, Server, CLI \\
	HashLink & Bytecode & Desktop, Mobile, Game consoles \\
	PHP & Sourcecode & Server \\
	Python & Sourcecode & Desktop, Server \\
	Lua & Sourcecode & Desktop, Scripting \\
	C++ & Sourcecode & Desktop, Mobile, Server, Game consoles \\
	ActionScript 3 & Sourcecode & Desktop, Mobile \\
	Flash & Bytecode & Desktop, Mobile \\ 
	Java & Sourcecode & Desktop, Mobile, Server \\
	C\# & Sourcecode & Desktop, Mobile, Server \\ \hline
\end{tabular}
\end{center}

The remainder of section \ref{introduction} gives a brief overview of what a Haxe program looks like and how Haxe evolved since its inception in 2005.

\Fullref{types} introduces the seven different kinds of types in Haxe and how they interact with each other. The discussion of types is continued in \Fullref{type-system}, where features like \emph{unification}, \emph{type parameters} and \emph{type inference} are explained.

\Fullref{class-field} is all about the structure of Haxe classes and, among other topics, deals with \emph{properties}, \emph{inline fields} and \emph{generic functions}.

In \Fullref{expression} we see how to actually get programs to do something by using \emph{expressions}.

\Fullref{lf} describes some of the Haxe features in detail such as \emph{pattern matching}, \emph{string interpolation} and \emph{dead code elimination}. This concludes the Haxe language reference.

We continue with the Haxe compiler reference, which first handles the basics in \Fullref{compiler-usage} before getting into the advanced features in \Fullref{cr-features}. Finally, we will venture into the exciting land of \emph{haxe macros} in \Fullref{macro} to see how some common tasks can be greatly simplified.

In the following chapter, \Fullref{std}, we explore important types and concepts from the Haxe Standard Library. We then learn about Haxe's package manager Haxelib in \Fullref{haxelib}.

Haxe abstracts away many target differences, but sometimes it is important to interact with a target directly, which is the subject of \Fullref{target-details}.

\section{About this Document}
\label{introduction-about-this-document}

This document is the official manual for Haxe 3. As such, it is not a beginner's tutorial and does not teach programming. However, the topics are roughly designed to be read in order and there are references to topics ``previously seen'' and topics ``yet to come''. In some cases, an earlier section makes use of the information of a later section if it simplifies the explanation. These references are linked accordingly and it should generally not be a problem to read ahead on other topics.

We use a lot of Haxe source code to keep a practical connection of theoretical matters. These code examples are often complete programs that come with a main function and can be compiled as-is. However, sometimes only the most important parts are shown.
Source code looks like this:

\begin{lstlisting}
Haxe code here
\end{lstlisting}
Occasionally, we demonstrate how Haxe code is generated, for which we usually show the \target{JavaScript} target.

Furthermore, we define a set of terms in this document. Predominantly, this is done when introducing a new type or when a term is specific to Haxe. We do not define every new aspect we introduce, e.g. what a class is, to avoid cluttering the text. A definition looks like this:

\define{Definition name}{define-definition}{Definition description}

In a few places, this document has \emph{trivia}-boxes. These include off-the-record information such as why certain decisions were made during Haxe's development or how a particular feature has been changed in past Haxe versions. This information is generally not important and can be skipped as it is only meant to convey trivia:

\trivia{About Trivia}{This is trivia.}

\subsection{Authors and contributions}
\label{introduction-authors-and-contributions}

Most of this document's content was written by Simon Krajewski while working for the Haxe Foundation. We would like to thank these people for their contributions:

\begin{itemize}
	\item Dan Korostelev: Additional content and editing
	\item Caleb Harper: Additional content and editing
	\item Josefiene Pertosa: Editing
	\item Miha Lunar: Editing
	\item Nicolas Cannasse: Haxe creator
\end{itemize}

\subsection{License}
\label{introduction-license}

The Haxe Manual by \href{http://haxe.org/foundation}{Haxe Foundation} is licensed under a \href{http://creativecommons.org/licenses/by/4.0/}{Creative Commons Attribution 4.0 International License}.

Based on a work at \href{https://github.com/HaxeFoundation/HaxeManual}{github.com/HaxeFoundation/HaxeManual}.

\section{Hello World}
\label{introduction-hello-world}

The following program prints ``Hello World'' after being compiled and run:

\haxe{assets/HelloWorld.hx}
This can be tested by saving the above code to a file named \ic{Main.hx} and invoking the Haxe Compiler like so: \ic{haxe -main Main --interp}. It then generates the following output: \ic{Main.hx:3: Hello world}. There are several things to learn from this:
\todo{This generates the following output: too many 'this'. You may like a passive sentence: the following output will be generated...though this is to be avoided, generally}

\begin{itemize}
	\item Haxe programs are saved in files with an extension of \ic{.hx}.
	\item The Haxe Compiler is a command-line tool which can be invoked with parameters such as \ic{-main Main} and \ic{--interp}.
	\item Haxe programs have classes (\type{Main}, upper-case), which have functions (\expr{main}, lower-case). 
	\item The name of the file containing main Haxe class is the same as name of the class itself (in this case \type{Main.hx}). 
\end{itemize}

\paragraph{Related content}
\begin{itemize}
	\item \href{http://code.haxe.org/category/beginner/}{Beginner tutorials and examples} in the Haxe Code Cookbook.
\end{itemize}

\section{History}
\label{introduction-haxe-history}
\state{Reviewed}

The Haxe project was started on 22 October 2005 by French developer \emph{Nicolas Cannasse} as a successor to the popular open-source ActionScript 2 compiler \emph{MTASC} (Motion-Twin Action Script Compiler) and the in-house \emph{MTypes} language, which experimented with the application of type inference to an object oriented language. Nicolas' long-time passion for programming language design and the rise of new opportunities to mix different technologies as part of his game developer work at \emph{Motion-Twin} led to the creation of a whole new language.

Being spelled \emph{haXe} back then, its beta version was released in February 2006 with the first supported targets being AVM\footnote{Adobe Virtual Machine}-bytecode and Nicolas' own \emph{Neko} virtual machine\footnote{http://nekovm.org}.

Nicolas Cannasse, who remains leader of the Haxe project to this date, kept on designing Haxe with a clear vision, subsequently leading to the Haxe 1.0 release in May 2006. This first major release came with support for \target{JavaScript} code generation and already had some of the features that define Haxe today such as type inference and structural sub-typing.

Haxe 1 saw several minor releases over the course of two years, adding the \target{Flash AVM2} target along with the \emph{haxelib}-tool in August 2006 and the \target{ActionScript 3} target in March 2007. During these months, there was a strong focus on improving stability, which resulted in several minor bug-fix releases.

Haxe 2.0 was released in July 2008, including the \target{PHP} target, courtesy of \emph{Franco Ponticelli}. A similar effort by \emph{Hugh Sanderson} lead to the addition of the \target{C++} target in July 2009 with the Haxe 2.04 release.

Just as with Haxe 1, what followed were several months of stability releases. In January 2011, Haxe 2.07 was released with the support of \emph{macros}. Around that time, \emph{Bruno Garcia} joined the team as maintainer of the \target{JavaScript} target, which saw vast improvements in the subsequent 2.08 and 2.09 releases.

After the release of 2.09, \emph{Simon Krajewski} joined the team and work towards Haxe 3 began. Furthermore, \emph{Cau\^{e} Waneck}'s \target{Java} and \target{C\#} targets found their way into the Haxe builds. It was then decided to make one final Haxe 2 release, which happened in July 2012 with the release of Haxe 2.10.

In late 2012, the Haxe 3 switch was flipped and the Haxe Compiler team, now backed by the newly established \emph{Haxe Foundation}\footnote{http://haxe-foundation.org}, focused on this next major version. Haxe 3 was subsequently released in May 2013.


\part{Language Reference}
\chapter{Types}
\label{types}

The Haxe Compiler employs a rich type system which helps detecting type-related errors in a program at compile-time. A type error is an invalid operation on a given type such as dividing by a String, trying to access a field of an Integer or calling a function with not enough (or too many) arguments.

In some languages this additional safety comes at a price because programmers are forced to explicitly assign types to syntactic constructs:

\lang{as3}\begin{lstlisting}
var myButton:MySpecialButton = new MySpecialButton(); // As3
\end{lstlisting}
\lang{cpp}\begin{lstlisting}
MySpecialButton* myButton = new MySpecialButton(); // C++ 
\end{lstlisting}
The explicit type annotations are not required in Haxe, because the compiler can \emph{infer} the type:

\begin{lstlisting}
var myButton = new MySpecialButton(); // Haxe
\end{lstlisting}
We will explore type inference in detail later in \Fullref{type-system-type-inference}. For now, it is sufficient to say that the variable \expr{myButton} in the above code is known to be an \emph{instance of class} \type{MySpecialButton}. 

The Haxe type system knows seven type groups:

\begin{description}
 \item[\emph{Class instance}:] an object of a given class or interface
 \item[\emph{Enum instance}:] a value of a Haxe enumeration
 \item[\emph{Structure}:] an anonymous structure, i.e. a collection of named fields
 \item[\emph{Function}:] a compound type of several arguments and one return
 \item[\emph{Dynamic}:] a wildcard type which is compatible with any type
 \item[\emph{Abstract}:] a compile-time type which is represented by a different type at runtime
 \item[\emph{Monomorph}:] an unknown type which may later become a different type
\end{description}

We will describe each of these type groups and how they relate to each other in the next chapters.

\define{Compound Type}{define-compound-type}{A compound type is a type which has sub-types. This includes any type with \tref{type parameters}{type-system-type-parameters} and the \tref{function}{types-function} type.}


\section{Basic Types}
\label{types-basic-types}

\emph{Basic types} are \type{Bool}, \type{Float} and \type{Int}. They can easily be identified in the syntax by values such as

\begin{itemize}
	\item \expr{true} and \expr{false} for \type{Bool},
	\item \expr{1}, \expr{0}, \expr{-1} and \expr{0xFF0000} for \type{Int} and
	\item \expr{1.0}, \expr{0.0}, \expr{-1.0}, \expr{1e10} for \type{Float}.
\end{itemize}

Basic types are not \tref{classes}{types-class-instance} in Haxe. They are implemented as \tref{abstract types}{types-abstract} and are tied to the compiler's internal operator-handling as described in the following sections.

\subsection{Numeric types}
\label{types-numeric-types}

\define[Type]{Float}{define-float}{Represents a double-precision IEEE 64-bit floating point number.}

\define[Type]{Int}{define-int}{Represents an integral number.}
While every \type{Int} can be used where a \type{Float} is expected (that is, \type{Int} \emph{is assignable to} or \emph{unifies with} \type{Float}), the reverse is not true: Assigning a \type{Float} to an \type{Int} might lose precision and is not allowed implicitly.

\subsection{Overflow}
\label{types-overflow}

For performance reasons, the Haxe Compiler does not enforce any overflow behavior. The burden of checking for overflows falls to the target platform. Here are some platform specific notes on overflow behavior:

\begin{description}
 \item[C++, Java, C\#, Neko, Flash:] 32-bit signed integers with usual overflow practices 
 \item[PHP, JS, Flash 8:] No native \emph{Int} type, loss of precision will occur if they reach their float limit (2\textsuperscript{52})
\end{description}

Alternatively, the \emph{haxe.Int32} and \emph{haxe.Int64} classes can be used to ensure correct overflow behavior regardless of the platform at the cost of additional computations depending on the platform.

\subsection{Numeric Operators}
\label{types-numeric-operators}

\todo{make sure the types are right for inc, dec, negate, and bitwise negate}
\todo{While introducing the different operations, we should include that information as well, including how they differ with the "C" standard, see http://haxe.org/manual/operators}
This the list of numeric operators in Haxe, grouped by descending priority.

\begin{center}
\begin{tabular}{| l | l | l | l | l |}
	\hline
	\multicolumn{5}{|c|}{Arithmetic} \\ \hline
	Operator & Operation & Operand 1 & Operand 2 & Return \\ \hline
	\expr{++}& increment & \type{Int} & N/A & \type{Int}\\
	& & \type{Float} & N/A & \type{Float}\\
	\expr{--} & decrement & \type{Int} & N/A & \type{Int}\\
	& & \type{Float} & N/A & \type{Float}\\
	\expr{+} & addition & \type{Float} & \type{Float} & \type{Float} \\
	& & \type{Float} & \type{Int} & \type{Float} \\
	& & \type{Int} & \type{Float} & \type{Float} \\
	& & \type{Int} & \type{Int} & \type{Int} \\
	\expr{-} & subtraction & \type{Float} & \type{Float} & \type{Float} \\
	& & \type{Float} & \type{Int} & \type{Float} \\
	& & \type{Int} & \type{Float} & \type{Float} \\
	& & \type{Int} & \type{Int} & \type{Int} \\
	\expr{*} & multiplication & \type{Float} & \type{Float} & \type{Float} \\
	& & \type{Float} & \type{Int} & \type{Float} \\
	& & \type{Int} & \type{Float} & \type{Float} \\
	& & \type{Int} & \type{Int} & \type{Int} \\	
	\expr{/} & division & \type{Float} & \type{Float} & \type{Float} \\
	& & \type{Float} & \type{Int} & \type{Float} \\
	& & \type{Int} & \type{Float} & \type{Float} \\
	& & \type{Int} & \type{Int} & \type{Float} \\
	\expr{\%} & modulo & \type{Float} & \type{Float} & \type{Float} \\
	& & \type{Float} & \type{Int} & \type{Float} \\
	& & \type{Int} & \type{Float} & \type{Float} \\
	& & \type{Int} & \type{Int} & \type{Int} \\	 \hline
	\multicolumn{5}{|c|}{Comparison} \\ \hline
	Operator & Operation & Operand 1 & Operand 2 & Return \\ \hline
	\expr{==} & equal & \type{Float/Int} & \type{Float/Int} & \type{Bool} \\
	\expr{!=} & not equal & \type{Float/Int} & \type{Float/Int} & \type{Bool} \\
	\expr{<} & less than & \type{Float/Int} & \type{Float/Int} & \type{Bool} \\
	\expr{<=} & less than or equal & \type{Float/Int} & \type{Float/Int} & \type{Bool} \\
	\expr{>} & greater than & \type{Float/Int} & \type{Float/Int} & \type{Bool} \\
	\expr{>=} & great than or equal & \type{Float/Int} & \type{Float/Int} & \type{Bool} \\ \hline
	\multicolumn{5}{|c|}{Bitwise} \\ \hline
	Operator & Operation & Operand 1 & Operand 2 & Return \\ \hline
	\expr{\textasciitilde} & bitwise negation & \type{Int} & N/A & \type{Int} \\	
	\expr{\&} & bitwise and & \type{Int} & \type{Int} & \type{Int} \\	
	\expr{|} & bitwise or & \type{Int} & \type{Int} & \type{Int} \\	
	\expr{\^} & bitwise xor & \type{Int} & \type{Int} & \type{Int} \\	
	\expr{<<} & shift left & \type{Int} & \type{Int} & \type{Int} \\
	\expr{>>} & shift right & \type{Int} & \type{Int} & \type{Int} \\
	\expr{>>>} & unsigned shift right & \type{Int} & \type{Int} & \type{Int} \\ \hline
\end{tabular}
\end{center}

\paragraph{Equality}

\emph{For enums:}
\begin{description}
	\item[Enum without parameters] Are always represent the same value, so \expr{MyEnum.A == MyEnum.A}. 
	\item[Enum with parameters] Can be compared with \expr{a.equals(b)} (which is a short for \expr{Type.enumEquals()}).
\end{description}

\emph{Dynamic:}
Comparison involving at least one Dynamic value is unspecifed and platform-specific.

\subsection{Bool}
\label{types-bool}

\define[Type]{Bool}{define-bool}{Represents a value which can be either \emph{true} or \emph{false}.}

Values of type \type{Bool} are a common occurrence in \emph{conditions} such as \tref{\expr{if}}{expression-if} and \tref{\expr{while}}{expression-while}. The following \emph{operators} accept and return \type{Bool} values:

\begin{itemize}
	\item \expr{\&\&} (and)
	\item \expr{||} (or)
	\item \expr{!} (not)
\end{itemize}

Haxe guarantees that compound boolean expressions are evaluated from left to right and only as far as necessary at run-time. For instance, an expression like \expr{A \&\& B} will evaluate \expr{A} first and evaluate \expr{B} only if the evaluation of \expr{A} yielded \expr{true}. Likewise, the expressions \expr{A || B} will not evaluate \expr{B} if the evaluation of \expr{A} yielded \expr{true}, because the value of \expr{B} is irrelevant in that case. This is important in cases such as this:

\begin{lstlisting}
if (object != null && object.field == 1) { }
\end{lstlisting}

Accessing \expr{object.field} if \expr{object} is \expr{null} would lead to a run-time error, but the check for \expr{object != null} guards against it.




\subsection{Void}
\label{types-void}

\define[Type]{Void}{define-void}{Void denotes the absence of a type. It is used to express that something (usually a function) has no value.}

\type{Void} is a special case in the type system because it is not actually a type. It is used to express the absence of a type, which applies mostly to function arguments and return types.
We have already ``seen'' Void in the initial ``Hello World'' example:
\todo{please review, doubled content}

\haxe{assets/HelloWorld.hx}

The function type will be explored in detail in the section \Fullref{types-function} but a quick preview helps here: The type of the function \expr{main} in the example above is \type{Void->Void}, which reads as ``it has no arguments and returns nothing''.
Haxe does not allow fields and variables of type \type{Void} and will complain if an attempt at declaring such is made:
\todo{review please, sounds weird}

\begin{lstlisting}
// Arguments and variables of type Void are not allowed
var x:Void;
\end{lstlisting}



\section{Nullability}
\label{types-nullability}

\define{nullable}{define-nullable}{A type in Haxe is considered \emph{nullable} if \expr{null} is a valid value for it.}

It is common for programming languages to have a single, clean definition for nullability. However, Haxe has to find a compromise in this regard due to the nature of Haxe's target languages: While some of them allow and; in fact, default to \expr{null} for anything, others do not even allow \expr{null} for certain types. This necessitates the distinction of two types of target languages:

\define{Static target}{define-static-target}{Static targets employ their own type system where \expr{null} is not a valid value for basic types. This is true for the \target{Flash}, \target{C++}, \target{Java} and \target{C\#} targets.}

\define{Dynamic target}{define-dynamic-target}{Dynamic targets are more lenient with their types and allow \expr{null} values for basic types. This applies to the \target{JavaScript}, \target{PHP}, \target{Neko} and \target{Flash 6-8} targets.}

There is nothing to worry about when working with \expr{null} on dynamic targets; however, static ones may require some thought. For starters, basic types are initialized to their default values.
\todo{for starters...please review}

\define{Default values}{define-default-value}{
	Basic types have the following default values on static targets:
	\begin{description}
		\item[\type{Int}:] \expr{0}
		\item[\type{Float}:] \expr{NaN} on \target{Flash}, \expr{0.0} on other static targets
		\item[\type{Bool}:] \expr{false}
	\end{description}
}

As a consequence, the Haxe Compiler does not allow the assignment of \expr{null} to a basic type on static targets. In order to achieve this, the basic type has to be wrapped as \type{Null$<$T$>$}:

\begin{lstlisting}
// error on static platforms
var a:Int = null;
var b:Null<Int> = null; // allowed
\end{lstlisting}

Similarly, basic types cannot be compared to \expr{null} unless wrapped:

\begin{lstlisting}
var a : Int = 0;
// error on static platforms
if( a == null ) { ... }
var b : Null<Int> = 0;
if( b != null ) { ... } // allowed
\end{lstlisting}

This restriction extends to all situations where \tref{unification}{type-system-unification} is performed.

\define[Type]{\expr{Null<T>}}{define-null-t}{On static targets the types \type{Null<Int>}, \type{Null<Float>} and \type{Null<Bool>} can be used to allow \expr{null} as a value. On dynamic targets this has no effect. \type{Null<T>} can also be used with other types in order to document that \expr{null} is an allowed value.}

If a \expr{null}-value is ``hidden'' in \type{Null$<$T$>$} or \type{Dynamic} and assigned to a basic type, the default value is used:

\begin{lstlisting}
var n : Null<Int> = null;
var a : Int = n;
trace(a); // 0 on static platforms
\end{lstlisting}



\subsection{Optional Arguments and Nullability}
\label{types-nullability-optional-arguments}

Optional arguments also have to be accounted for when considering nullability.

In particular, there must be a distinction between \emph{native} optional arguments which are not nullable and Haxe-specific optional arguments which might be. The distinction is made by using the question-mark optional argument:

\begin{lstlisting}
// x is a native Int (not nullable)
function foo(x : Int = 0) {}
// y is Null<Int> (nullable)
function bar( ?y : Int) {}
// z is also Null<Int>
function opt( ?z : Int = -1) {}
\end{lstlisting}
\todo{Is there a difference between \type{?y : Int} and \type{y : Null$<$Int$>$} or can you even do the latter? Some more explanation and examples with native optional and Haxe optional arguments and how they relate to nullability would be nice.}

\trivia{Argument vs. Parameter}{In some other programming languages, \emph{argument} and \emph{parameter} are used interchangeably.  In Haxe, \emph{argument} is used when referring to methods and \emph{parameter} refers to \Fullref{type-system-type-parameters}.}

\section{Class Instance}
\label{types-class-instance}

Similar to many object-oriented languages, classes are the primary data structure for the majority of programs in Haxe. Each Haxe class has an explicit name, an implied path and zero or more class fields. Here we will focus on the general structure of classes and their relations, while leaving the details of class fields for \Fullref{class-field}.
\todo{please review future tense}

The following code example serves as basis for the remainder of this section:

\haxe{assets/Point.hx}

Semantically, this class represents a point in discrete 2-dimensional space - but this is not important here. Let us instead describe the structure:

\begin{itemize}
	\item The keyword \expr{class} denotes that we are declaring a class.
	\item \type{Point} is the name of the class and could be anything conforming to the \tref{rules for type identifiers}{define-identifier}.
	\item Enclosed in curly braces \expr{$\left\{\right\}$} are the class fields,
	\item Which consist of two \emph{variable} fields \expr{x} and \expr{y} of type \type{Int},
	\item followed by a special \emph{function} field named \expr{new}, which is the \emph{constructor} of the class,
	\item as well as a normal function \expr{toString}.
\end{itemize}
There is a special type in Haxe which is compatible with all classes:

\define[Type]{\expr{Class$<$T$>$}}{define-class-t}{This type is compatible with all class types which means that all classes (not their instances) can be assigned to it. At compile-time, \type{Class<T>} is the common base type of all class types. However, this relation is not reflected in generated code.

This type is useful when an API requires a value to be \emph{a} class, but not a specific one. This applies to several methods of the \tref{Haxe reflection API}{std-reflection}.}

\subsection{Class constructor}
\label{types-class-constructor}

Instances of classes are created by calling the class constructor - a process commonly referred to as \emph{instantiation}. Another name for class instances is \emph{object}. Nevertheless, we prefer the term class instance to emphasize the analogy between classes/class instances and \tref{enums/enum instances}{types-enum-instance}. 

\begin{lstlisting}
var p = new Point(-1, 65);
\end{lstlisting}
This will yield an instance of class \type{Point}, which is assigned to a variable named \expr{p}. The constructor of \type{Point} receives the two arguments \expr{-1} and \expr{65} and assigns them to the instance variables \expr{x} and \expr{y} respectively (compare its definition in \Fullref{types-class-instance}). We will revisit the exact meaning of the \expr{new} expression later in the section \ref{expression-new}. For now, we just think of it as calling the class constructor and returning the appropriate object.



\subsection{Inheritance}
\label{types-class-inheritance}

Classes may inherit from other classes, which in Haxe is denoted by the \expr{extends} keyword:

\haxe{assets/Point3.hx}
This relation is often described as "is-a": Any instance of class \type{Point3} is also an instance of \type{Point}. \type{Point} is then known as the \emph{parent class} of \type{Point3}, which is a \emph{child class} of \type{Point}. A class may have many child classes, but only one parent class. The term ``a parent class of class X'' usually refers to its direct parent class, the parent class of its parent class and so on.

The code above is very similar to the original \type{Point} class, with two new constructs being shown:
\begin{itemize}
 \item \expr{extends Point} denotes that this class inherits from class \type{Point}
 \item \expr{super(x, y)} is the call to the constructor of the parent class, in this case \expr{Point.new}
\end{itemize}
It is not necessary for child classes to define their own constructors, but if they do, a call to \expr{super()} is mandatory. Unlike some other object-oriented languages, this call can appear anywhere in the constructor code and does not have to be the first expression.

A class may override \tref{methods}{class-field-method} of its parent class, which requires the explicit \expr{override} keyword. The effects and restrictions of this are detailed in \Fullref{class-field-overriding}.


\subsection{Interfaces}
\label{types-interfaces}

An interface can be understood as the signature of a class because it describes the public fields of a class. Interfaces do not provide implementations but pure structural information:

\begin{lstlisting}
interface Printable {
	public function toString():String;
}
\end{lstlisting}
The syntax is similar to classes, with the following exceptions:

\begin{itemize}
	\item \expr{interface} keyword is used instead of \expr{class} keyword
	\item functions do not have any \tref{expressions}{expression}
	\item every field must have an explicit type
\end{itemize}
Interfaces, unlike \tref{structural subtyping}{type-system-structural-subtyping}, describe a \emph{static relation} between classes. A given class is only considered to be compatible to an interface if it explicitly states so:

\begin{lstlisting}
class Point implements Printable { }
\end{lstlisting}
Here, the \expr{implements} keyword denotes that \type{Point} has a "is-a" relationship to \type{Printable}, i.e. each instance of \type{Point} is also an instance of \type{Printable}. While a class may only have one parent class, it may implement multiple interfaces through multiple \expr{implements} keywords:

\begin{lstlisting}
class Point implements Printable
  implements Serializable
\end{lstlisting}

The compiler checks if the \expr{implements} assumption holds. That is, it makes sure the class actually does implement all the fields required by the interface. A field is considered implemented if the class or any of its parent classes provide an implementation.

Interface fields are not limited to methods. They can be variables and properties as well:

\haxe{assets/InterfaceWithVariables.hx}

Interfaces can extend multiple other interfaces using the \expr{extends} keyword:
\begin{lstlisting}
interface Debuggable extends Printable extends Serializable
\end{lstlisting}


\trivia{Implements Syntax}{Haxe versions prior to 3.0 required multiple \expr{implements} keywords to be separated by a comma. We decided to adhere to the de-facto standard of Java and got rid of the comma. This was one of the breaking changes between Haxe 2 and 3.}


\section{Enum Instance}
\label{types-enum-instance}

Haxe provides powerful enumeration (short: enum) types, which are actually an \emph{algebraic data type} (ADT)\footnote{\url{http://en.wikipedia.org/wiki/Algebraic_data_type}}. While they cannot have any \tref{expressions}{expression}, they are very useful for describing data structures:

\haxe{assets/Color.hx}
Semantically, this enum describes a color which is either red, green, blue or a specified RGB value. The syntactic structure is as follows:
\begin{itemize}
	\item The keyword \expr{enum} denotes that we are declaring an enum.
	\item \type{Color} is the name of the enum and could be anything conforming to the rules for \tref{type identifiers}{define-identifier}.
	\item Enclosed in curly braces \expr{$\left\{\right\}$} are the \emph{enum constructors},
	\item which are \expr{Red}, \expr{Green} and \expr{Blue} taking no arguments,
	\item as well as \expr{Rgb} taking three \type{Int} arguments named \expr{r}, \expr{g} and \expr{b}.
\end{itemize}
The Haxe type system provides a type which unifies with all enum types:

\define[Type]{\expr{Enum$<$T$>$}}{define-enum-t}{This type is compatible with all enum types. At compile-time, \type{Enum<T>} can be seen as the common base type of all enum types. However, this relation is not reflected in generated code.} 
\todo{Same as in 2.2, what is \type{Enum$<$T$>$} syntax?}

\subsection{Enum Constructor}
\label{types-enum-constructor}

Similar to classes and their constructors, enums provide a way of instantiating them by using one of their constructors. However, unlike classes, enums provide multiple constructors which can easily be used through their name:

\begin{lstlisting}
var a = Red;
var b = Green;
var c = Rgb(255, 255, 0);
\end{lstlisting}
In this code the type of variables \expr{a}, \expr{b} and \expr{c} is \type{Color}. Variable \expr{c} is initialized using the \expr{Rgb} constructor with arguments.
\todo{list arguments}

All enum instances can be assigned to a special type named \type{EnumValue}.

\define[Type]{EnumValue}{define-enumvalue}{EnumValue is a special type which unifies with all enum instances. It is used by the Haxe Standard Library to provide certain operations for all enum instances and can be employed in user-code accordingly in cases where an API requires \emph{an} enum instance, but not a specific one.}

It is important to distinguish enum types and enum constructors, as this example demonstrates:

\haxe{assets/EnumUnification.hx}

If the commented line is uncommented, the program does not compile because \expr{Red} (an enum constructor) cannot be assigned to a variable of type \type{Enum<Color>} (an enum type). The relation is analogous to a class and its instance.

\trivia{Concrete type parameter for \type{Enum$<$T$>$}}{One of the reviewers of this manual was confused about the difference between \type{Color} and \type{Enum<Color>} in the example above. Indeed, using a concrete type parameter there is pointless and only serves the purpose of demonstration. Usually we would omit the type there and let \tref{type inference}{type-system-type-inference} deal with it.

However, the inferred type would be different from \type{Enum<Color>}. The compiler infers a pseudo-type which has the enum constructors as ``fields''. As of Haxe 3.2.0, it is not possible to express this type in syntax but also, it is never necessary to do so.}



\subsection{Using enums}
\label{types-enum-using}

Enums are a good choice if only a finite set of values should be allowed. The individual \tref{constructors}{types-enum-constructor} then represent the allowed variants and enable the compiler to check if all possible values are respected. This can be seen here:

\haxe{assets/Color2.hx}

After retrieving the value of \expr{color} by assigning the return value of \expr{getColor()} to it, a \tref{\expr{switch} expression}{expression-switch} is used to branch depending on the value. The first three cases \expr{Red}, \expr{Green} and \expr{Blue} are trivial and correspond to the constructors of \type{Color} that have no arguments. The final case \expr{Rgb(r, g, b)} shows how the argument values of a constructor can be extracted: they are available as local variables within the case body expression, just as if a \tref{\expr{var} expression}{expression-var} had been used.

Advanced information on using the \expr{switch} expression will be explored later in the section on \tref{pattern matching}{lf-pattern-matching}.


\section{Anonymous Structure}
\label{types-anonymous-structure}

Anonymous structures can be used to group data without explicitly creating a type. The following example creates a structure with two fields \expr{x} and \expr{name}, and initializes their values to \expr{12} and \expr{"foo"} respectively:

\haxe{assets/Structure.hx}
The general syntactic rules follow:

\begin{enumerate}
	\item A structure is enclosed in curly braces \expr{$\left\{\right\}$} and
	\item Has a \emph{comma-separated} list of key-value-pairs.
	\item A \emph{colon} separates the key, which must be a valid \tref{identifier}{define-identifier}, from the value.
	\item\label{valueanytype} The value can be any Haxe expression.
\end{enumerate}
Rule \ref{valueanytype} implies that structures can be nested and complex, e.g.:

\todo{please reformat}

\begin{lstlisting}
var user = {
  name : "Nicolas",
	age : 32,
	pos : [
	  { x : 0, y : 0 },
		{ x : 1, y : -1 }
  ],
};
\end{lstlisting}
Fields of structures, like classes, are accessed using a \emph{dot} (\expr{.}) like so:

\begin{lstlisting}
// get value of name, which is "Nicolas"
user.name;
// set value of age to 33
user.age = 33;
\end{lstlisting}
It is worth noting that using anonymous structures does not subvert the typing system. The compiler ensures that only available fields are accessed, which means the following program does not compile:

\begin{lstlisting}
class Test {
  static public function main() {
    var point = { x: 0.0, y: 12.0 };
    // { y : Float, x : Float } has no field z
    point.z;
  }
}
\end{lstlisting}
The error message indicates that the compiler knows the type of \expr{point}: It is a structure with fields \expr{x} and \expr{y} of type \type{Float}. Since it has no field \expr{z}, the access fails.
The type of \expr{point} is known through \tref{type inference}{type-system-type-inference}, which thankfully saves us from using explicit types for local variables. However, if \expr{point} was a field, explicit typing would be necessary:

\begin{lstlisting}
class Path {
    var start : { x : Int, y : Int };
    var target : { x : Int, y : Int };
    var current : { x : Int, y : Int };
}
\end{lstlisting}
To avoid this kind of redundant type declaration, especially for more complex structures, it is advised to use a \tref{typedef}{type-system-typedef}:

\begin{lstlisting}
typedef Point = { x : Int, y : Int }

class Path {
    var start : Point;
    var target : Point;
    var current : Point;
}
\end{lstlisting}

You may also use \tref{Extensions}{types-structure-extensions} to ``inherit'' fields from other structures.

\begin{lstlisting}
typedef Point3 = { > Point, z : Int }
\end{lstlisting}


\subsection{JSON for Structure Values}
\label{types-structure-json}

It is also possible to use \emph{JavaScript Object Notation} for structures by using \emph{string literals} for the keys:

\begin{lstlisting}
var point = { "x" : 1, "y" : -5 };
\end{lstlisting}
While any string literal is allowed, the field is only considered part of the type if it is a valid \tref{Haxe identifier}{define-identifier}. Otherwise, Haxe syntax does not allow expressing access to such a field, and \tref{reflection}{std-reflection} has to be employed through the use of \expr{Reflect.field} and \expr{Reflect.setField}.

\subsection{Class Notation for Structure Types}
\label{types-structure-class-notation}

When defining a structure type, Haxe allows using the same syntax as described in \Fullref{class-field}. The following \tref{typedef}{type-system-typedef} declares a \type{Point} type with variable fields \expr{x} and \expr{y} of type \type{Int}:

\begin{lstlisting}
typedef Point = {
    var x : Int;
    var y : Int;
}
\end{lstlisting}

\subsection{Optional Fields}
\label{types-structure-optional-fields}

Fields of a structure type can be made optional. In the standard notation, this is achieved by prefixing the field name with a \ic{?}:

\begin{lstlisting}
typedef User = {
  age : Int,
  name : String,
  ?phoneNumber : String
}
\end{lstlisting}

In class notation, the \expr{@:optional} metadata can be used instead:

\begin{lstlisting}
typedef User = {
  var age : Int;
  var name : String;
  @:optional var phoneNumber : String;
}
\end{lstlisting}




\subsection{Impact on Performance}
\label{types-structure-performance}

Using structures and, by extension, \tref{structural subtyping}{type-system-structural-subtyping} has no impact on performance when compiling to \tref{dynamic targets}{define-dynamic-target}. However, on \tref{static targets}{define-static-target} a dynamic lookup has to be performed which is typically slower than a static field access.

\subsection{Extensions}
\label{types-structure-extensions}

Extensions are used to express that a structure has all the fields of a given type as well as some additional fields of its own:

\haxe{assets/Extension.hx}
The greater-than operator \expr{>} denotes that an extension of \type{Iterable$<$T$>$} is being created, with the additional class fields following. In this case, a read-only \tref{property}{class-field-property} \expr{length} of type \type{Int} is required.

In order to be compatible with \type{IterableWithLength$<$T$>$}, a type then must be compatible with \type{Iterable$<$T$>$} and also provide a read-only \expr{length} property of type \type{Int}. The example assigns an \type{Array}, which happens to fulfill these requirements.

\since{3.1.0}

It is also possible to extend multiple structures:

\haxe{assets/Extension2.hx}



\section{Function Type}
\label{types-function}

\todo{It seems a bit convoluted explanations. Should we maybe start by "decoding" the meaning of  Void -> Void, then Int -> Bool -> Float, then maybe have samples using \$type}

The function type, along with the \tref{monomorph}{types-monomorph}, is a type which is usually well-hidden from Haxe users, yet present everywhere. We can make it surface by using \expr{\$type}, a special Haxe identifier which outputs the type its expression has during compilation :

\haxe{assets/FunctionType.hx}

There is a strong resemblance between the declaration of function \expr{test} and the output of the first \expr{\$type} expression, yet also a subtle difference:

\begin{itemize}
	\item \emph{Function arguments} are separated by the special arrow token \expr{->} instead of commas, and
	\item the \emph{function return type} appears at the end after another \expr{->}.
\end{itemize}

In either notation it is obvious that the function \expr{test} accepts a first argument of type \type{Int}, a second argument of type \type{String} and returns a value of type \type{Bool}. If a call to this function, such as \expr{test(1, "foo")}, is made within the second \expr{\$type} expression, the Haxe typer checks if \expr{1} can be assigned to \type{Int} and if \expr{"foo"} can be assigned to \type{String}. The type of the call is then equal to the type of the value \expr{test} returns, which is \type{Bool}.

If a function type has other function types as argument or return type, parentheses can be used to group them correctly. For example, \type{Int -> (Int -> Void) -> Void} represents a function which has a first argument of type \type{Int}, a second argument of function type \type{Int -> Void} and a return of \type{Void}.



\subsection{Optional Arguments}
\label{types-function-optional-arguments}

Optional arguments are declared by prefixing an argument identifier with a question mark \expr{?}:

\haxe[label=assets/OptionalArguments.hx]{assets/OptionalArguments.hx}
Function \expr{test} has two optional arguments: \expr{i} of type \type{Int} and \expr{s} of \type{String}. This is directly reflected in the function type output by line 3. 
This example program calls \expr{test} four times and prints its return value.

\begin{enumerate}
	\item The first call is made without any arguments.
	\item The second call is made with a singular argument \expr{1}.
	\item The third call is made with two arguments \expr{1} and \expr{"foo"}.
	\item The fourth call is made with a singular argument \expr{"foo"}.
\end{enumerate}
The output shows that optional arguments which are omitted from the call have a value of \expr{null}. This implies that the type of these arguments must admit \expr{null} as value, which raises the question of its \tref{nullability}{types-nullability}. The Haxe Compiler ensures that optional basic type arguments are nullable by inferring their type as \type{Null<T>} when compiling to a \tref{static target}{define-static-target}.

While the first three calls are intuitive, the fourth one might come as a surprise: It is indeed allowed to skip optional arguments if the supplied value is assignable to a later argument.


\subsection{Default values}
\label{types-function-default-values}

Haxe allows default values for arguments by assigning a \emph{constant value} to them:

\haxe{assets/DefaultValues.hx}
This example is very similar to the one from \Fullref{types-function-optional-arguments}, with the only difference being that the values \expr{12} and \expr{"bar"} are assigned to the function arguments \expr{i} and \expr{s} respectively. The effect is that the default values are used instead of \expr{null} should an argument be omitted from the call.

%TODO: Default values do not imply nullability, even if the value is \expr{null}. 

Default values in Haxe are not part of the type and are not replaced at call-site (unless the function is \tref{inlined}{class-field-inline}, which can be considered as a more typical approach. On some targets the compiler may still pass \expr{null} for omitted argument values and generate code similar to this into the function:
\begin{lstlisting}
	static function test(i = 12, s = "bar") {
		if (i == null) i = 12;
		if (s == null) s = "bar";
		return "i: " +i + ", s: " +s;
	}
\end{lstlisting}
This should be considered in performance-critical code where a solution without default values may sometimes be more viable.




\section{Dynamic}
\label{types-dynamic}

While Haxe has a static type system, this type system can, in effect, be turned off by using the \type{Dynamic} type. A \emph{dynamic value} can be assigned to anything; and anything can be assigned to it. This has several drawbacks:

\begin{itemize}
	\item The compiler can no longer type-check assignments, function calls and other constructs where specific types are expected.
	\item Certain optimizations, in particular when compiling to static targets, can no longer be employed.
	\item Some common errors, e.g. a typo in a field access, can not be caught at compile-time and likely cause an error at runtime.
	\item \Fullref{cr-dce} cannot detect used fields if they are used through \type{Dynamic}.
\end{itemize}
It is very easy to come up with examples where the usage of \type{Dynamic} can cause problems at runtime. Consider compiling the following two lines to a static target:

\begin{lstlisting}
var d:Dynamic = 1;
d.foo;
\end{lstlisting}

Trying to run a compiled program in the Flash Player yields an error \texttt{Property foo not found on Number and there is no default value}. Without \type{Dynamic}, this would have been detected at compile-time.

\trivia{Dynamic Inference before Haxe 3}{The Haxe 3 compiler never infers a type to \type{Dynamic}, so users must be explicit about it. Previous Haxe versions used to infer arrays of mixed types, e.g. \expr{[1, true, "foo"]}, as \type{Array<Dynamic>}. We found that this behavior introduced too many type problems and thus removed it for Haxe 3.}

Use of \type{Dynamic} should be minimized as there are better options in many situations but sometimes it is just practical to use it. Parts of the Haxe \Fullref{std-reflection} API use it and it is sometimes the best option when dealing with custom data structures that are not known at compile-time.

\type{Dynamic} behaves in a special way when being \tref{unified}{type-system-unification} with a \tref{monomorph}{types-monomorph}. Monomorphs are never bound to \type{Dynamic} which can have surprising results in examples such as this:

\haxe{assets/DynamicInferenceIssue.hx}

Although the return type of \expr{Json.parse} is \type{Dynamic}, the type of local variable \expr{json} is not bound to it and remains a monomorph. It is then inferred as an \tref{anonymous structure}{types-anonymous-structure} upon the \expr{json.length} field access, which causes the following \expr{json[0]} array access to fail. In order to avoid this, the variable \expr{json} can be explicitly typed as \type{Dynamic} by using \expr{var json:Dynamic}.

\trivia{Dynamic in the Standard Library}{Dynamic was quite frequent in the Haxe Standard Library before Haxe 3. With the continuous improvements of the Haxe type system the occurrences of Dynamic were reduced over the releases leading to Haxe 3.}

\subsection{Dynamic with Type Parameter}
\label{types-dynamic-with-type-parameter}

\type{Dynamic} is a special type because it allows explicit declaration with and without a \tref{type parameter}{type-system-type-parameters}. If such a type parameter is provided, the semantics described in \Fullref{types-dynamic} are constrained to all fields being compatible with the parameter type:

\begin{lstlisting}
var att : Dynamic<String> = xml.attributes;
// valid, value is a String
att.name = "Nicolas";
// dito (this documentation is quite old)
att.age = "26";
// error, value is not a String
att.income = 0;
\end{lstlisting}

\subsection{Implementing Dynamic}
\label{types-dynamic-implemented}

Classes can \tref{implement}{types-interfaces} \type{Dynamic} and \type{Dynamic$<$T$>$} which enables arbitrary field access. In the former case, fields can have any type, in the latter, they are constrained to be compatible with the parameter type:

\haxe{assets/ImplementsDynamic.hx}

Implementing \type{Dynamic} does not satisfy the requirements of other implemented interfaces. The expected fields still have to be implemented explicitly.

Classes that implement \type{Dynamic} (with or without type parameter) can also utilize a special method named \expr{resolve}. If a \tref{read access}{define-read-access} is made and the field in question does not exist, the \expr{resolve} method is called with the field name as argument:

\haxe{assets/DynamicResolve.hx}


\subsection{Dynamic access}
\label{types-dynamic-access}

\type{DynamicAccess} is an \tref{abstract type}{types-abstract} for working with \tref{anonymous structures}{types-anonymous-structure} that are intended to hold collections of objects by the string key. Basically \type{DynamicAccess} wraps \type{Reflect}{std-reflection} calls in a Map-like interface.

\haxe{assets/DynamicAccess.hx}

\subsection{Any type}
\label{types-dynamic-any}

\type{Any} is a type that is compatible with any other in both ways. 
It serves one purpose - to hold values of any type, but to actually use that values, an explicit casting is required. That way the code doesn't suddently become dynamically typed and we keep all the static typing goodness, like advanced type system features and optimizations.

The implementation is quite simple:

\begin{lstlisting}
abstract Any(Dynamic) from Dynamic to Dynamic {}
\end{lstlisting}

This type don't make any assumptions about what the value actually is and whether it supports fields or any operations - this is up to the user.

\haxe{assets/Any.hx}

It's a more type-safe alternative to \type{Dynamic}, because it doesn't support field access or operators and it's bound to monomorphs. So, to work with the actual value, it needs to be explicitly promoted to another type.



\section{Abstract}
\label{types-abstract}

An abstract type is a type which is actually a different type at run-time. It is a compile-time feature which defines types ``over'' concrete types in order to modify or augment their behavior:

\haxe[firstline=1,lastline=5]{assets/MyAbstract.hx}
We can derive the following from this example:

\begin{itemize}
	\item The keyword \expr{abstract} denotes that we are declaring an abstract type.
	\item \type{AbstractInt} is the name of the abstract and could be anything conforming to the rules for type identifiers.
	\item Enclosed in parenthesis \expr{()} is the \emph{underlying type} \type{Int}.
	\item Enclosed in curly braces \expr{$\left\{\right\}$} are the fields,
	\item which are a constructor function \expr{new} accepting one argument \expr{i} of type \type{Int}.
\end{itemize}

\define{Underlying Type}{define-underlying-type}{The underlying type of an abstract is the type which is used to represent said abstract at runtime. It is usually a concrete (i.e. non-abstract) type but could be another abstract type as well.}

The syntax is reminiscent of classes and the semantics are indeed similar. In fact, everything in the ``body'' of an abstract (that is everything after the opening curly brace) is parsed as class fields. Abstracts may have \tref{method}{class-field-method} fields and non-\tref{physical}{define-physical-field} \tref{property}{class-field-property} fields.

Furthermore, abstracts can be instantiated and used just like classes:

\haxe[firstline=7,lastline=12]{assets/MyAbstract.hx}
As mentioned before, abstracts are a compile-time feature, so it is interesting to see what the above actually generates. A suitable target for this is \target{JavaScript}, which tends to generate concise and clean code. Compiling the above (using \texttt{haxe -main MyAbstract -js myabstract.js}) shows this \target{JavaScript} code:

\lang{js}\begin{lstlisting}
var a = 12;
console.log(a);
\end{lstlisting}
The abstract type \type{Abstract} completely disappeared from the output and all that is left is a value of its underlying type, \type{Int}. This is because the constructor of \type{Abstract} is inlined - something we shall learn about later in the section \Fullref{class-field-inline} - and its inlined expression assigns a value to \expr{this}. This might be surprising when thinking in terms of classes. However, it is precisely what we want to express in the context of abstracts. Any \emph{inlined member method} of an abstract can assign to \expr{this}, and thus modify the ``internal value''.


A good question at this point is ``What happens if a member function is not declared inline'' because the code obviously has to go somewhere. Haxe creates a private class, known to be the \emph{implementation class}, which has all the abstract member functions as static functions accepting an additional first argument \expr{this} of the underlying type.



\trivia{Basic Types and abstracts}{Before the advent of abstract types, all basic types were implemented as extern classes or enums. While this nicely took care of some aspects such as \type{Int} being a ``child class'' of \type{Float}, it caused issues elsewhere. For instance, with \type{Float} being an extern class, it would unify with the empty structure \expr{\{\}}, making it impossible to constrain a type to accepting only real objects.}




\subsection{Implicit Casts}
\label{types-abstract-implicit-casts}

Unlike classes, abstracts allow defining implicit casts. There are two kinds of implicit casts:

\begin{description}
	\item[Direct:] Allows direct casting of the abstract type to or from another type. This is defined by adding \expr{to} and \expr{from} rules to the abstract type and is only allowed for types which unify with the underlying type of the abstract.
	\item[Class field:] Allows casting via calls to special cast functions. These functions are defined using \expr{@:to} and \expr{@:from} metadata. This kind of cast is allowed for all types.
\end{description}
The following code example shows an example of \emph{direct} casting:

\haxe{assets/ImplicitCastDirect.hx}
We declare \type{MyAbstract} as being \expr{from Int} and \expr{to Int}, meaning it can be assigned from \type{Int} and assigned to \type{Int}. This is shown in lines 9 and 10, where we first assign the \type{Int} \expr{12} to variable \expr{a} of type \type{MyAbstract} (this works due to the \expr{from Int} declaration) and then that abstract back to variable \expr{b} of type \type{Int} (this works due to the \expr{to Int} declaration).

Class field casts have the same semantics, but are defined completely differently:

\haxe{assets/ImplicitCastField.hx}
By adding \expr{@:from} to a static function, that function qualifies as implicit cast function from its argument type to the abstract. These functions must return a value of the abstract type. They must also be declared \expr{static}.

Similarly, adding \expr{@:to} to a function qualifies it as implicit cast function from the abstract to its return type.

In the example the method \expr{fromString} allows the assignment of value \expr{"3"} to variable \expr{a} of type \type{MyAbstract} while the method \expr{toArray} allows assigning that abstract to variable \expr{b} of type \type{Array<Int>}.

When using this kind of cast, calls to the cast-functions are inserted where required. This becomes obvious when looking at the \target{JavaScript} output:

\lang{js}\begin{lstlisting}
var a = _ImplicitCastField.MyAbstract_Impl_.fromString("3");
var b = _ImplicitCastField.MyAbstract_Impl_.toArray(a);
\end{lstlisting}
This can be further optimized by \tref{inlining}{class-field-inline} both cast functions, turning the output into the following:
\todo{please review your use of ``this'' and try to vary somewhat to avoid too much word repetition}

\begin{lstlisting}
var a = Std.parseInt("3");
var b = [a];
\end{lstlisting}
The \emph{selection algorithm} when assigning a type \expr{A} to a type \expr{B} with at least one of them being an abstract is simple:

\begin{enumerate}
	\item If \expr{A} is not an abstract, go to 3.
	\item If \expr{A} defines a \emph{to}-conversions that admits \expr{B}, go to 6.
	\item If \expr{B} is not an abstract, go to 5.
	\item If \expr{B} defines a \emph{from}-conversions that admits \expr{A}, go to 6.
	\item Stop, unification fails.
	\item Stop, unification succeeds.
\end{enumerate}

\input{assets/tikz/abstract-selection.tex}

By design, implicit casts are \emph{not transitive}, as the following example shows:

\haxe{assets/ImplicitTransitiveCast.hx}
While the individual casts from \type{A} to \type{B} and from \type{B} to \type{C} are allowed, a transitive cast from \type{A} to \type{C} is not. This is to avoid ambiguous cast-paths and retain a simple selection algorithm. 




\subsection{Operator Overloading}
\label{types-abstract-operator-overloading}

Abstracts allow overloading of unary and binary operators by adding the \expr{@:op} metadata to class fields:

\haxe{assets/AbstractOperatorOverload.hx}
By defining \expr{@:op(A * B)}, the function \expr{repeat} serves as operator method for the multiplication \expr{*} operator when the type of the left value is \type{MyAbstract} and the type of the right value is \type{Int}. The usage is shown in line 17, which turns into this when compiled to \target{JavaScript}:

\lang{js}\begin{lstlisting}
console.log(_AbstractOperatorOverload.
  MyAbstract_Impl_.repeat(a,3));
\end{lstlisting}
Similar to \tref{implicit casts with class fields}{types-abstract-implicit-casts}, a call to the overload method is inserted where required.

The example \expr{repeat} function is not commutative: While \expr{MyAbstract * Int} works, \expr{Int * MyAbstract} does not. If this should be allowed as well, the \expr{@:commutative} metadata can be added. If it should work \emph{only} for \expr{Int * MyAbstract}, but not for \expr{MyAbstract * Int}, the overload method can be made static, accepting \type{Int} and \type{MyAbstract} as first and second type respectively.

Overloading unary operators is analogous:

\haxe{assets/AbstractUnopOverload.hx}
Both binary and unary operator overloads can return any type.

\paragraph{Exposing underlying type operations}

It is also possible to omit the method body of a \expr{@:op} function, but only if the underlying type of the abstract allows the operation in question and if the resulting type can be assigned back to the abstract.

\haxe{assets/AbstractExposeTypeOperations.hx}

\todo{please review for correctness}


\subsection{Array Access}
\label{types-abstract-array-access}

Array access describes the particular syntax traditionally used to access the value in an array at a certain offset. This is usually only allowed with arguments of type \type{Int}. Nevertheless, with abstracts it is possible to define custom array access methods. The \tref{Haxe Standard Library}{std} uses this in its \type{Map} type, where the following two methods can be found:
\todo{You have marked ``Map'' for some reason}

\begin{lstlisting}
@:arrayAccess
public inline function get(key:K) {
  return this.get(key);
}
@:arrayAccess
public inline function arrayWrite(k:K, v:V):V {
	this.set(k, v);
	return v;
}
\end{lstlisting}
There are two kinds of array access methods:

\begin{itemize}
	\item If an \expr{@:arrayAccess} method accepts one argument, it is a getter.
	\item If an \expr{@:arrayAccess} method accepts two arguments, it is a setter.
\end{itemize}
The methods \expr{get} and \expr{arrayWrite} seen above then allow the following usage:

\haxe{assets/AbstractArrayAccess.hx}

At this point it should not be surprising to see that calls to the array access fields are inserted in the output:

\lang{js}\begin{lstlisting}
map.set("foo",1);
console.log(map.get("foo")); // 1
\end{lstlisting}

\paragraph{Order of array access resolving}
\label{types-abstract-array-access-order}

Due to a bug in Haxe versions before 3.2 the order of checked \expr{:arrayAccess} fields was undefined. This was fixed for Haxe 3.2 so that the fields are now consistently checked from top to bottom:

\haxe{assets/AbstractArrayAccessOrder.hx}

The array access \expr{a[0]} is resolved to the \expr{getInt1} field, leading to lower case \expr{f} being returned. The result might be different in Haxe versions before 3.2.

Fields which are defined earlier take priority even if they require an \tref{implicit cast}{types-abstract-implicit-casts}.



\subsection{Enum abstracts}
\label{types-abstract-enum}
\since{3.1.0}

By adding the \expr{:enum} metadata to an abstract definition, that abstract can be used to define finite value sets:

\haxe{assets/AbstractEnum.hx}

The Haxe Compiler replaces all field access to the \type{HttpStatus} abstract with their values, as evident in the \target{JavaScript} output:

\lang{js}\begin{lstlisting}
Main.main = function() {
	var status = 404;
	var msg = Main.printStatus(status);
};
Main.printStatus = function(status) {
	switch(status) {
	case 404:
		return "Not found";
	case 405:
		return "Method not allowed";
	}
};
\end{lstlisting}

This is similar to accessing \tref{variables declared as inline}{class-field-inline}, but has several advantages:

\begin{itemize}
	\item The typer can ensure that all values of the set are typed correctly.
	\item The pattern matcher checks for \tref{exhaustiveness}{lf-pattern-matching-exhaustiveness} when \tref{matching}{lf-pattern-matching} an enum abstract.
	\item Defining fields requires less syntax.
\end{itemize}


\subsection{Forwarding abstract fields}
\label{types-abstract-forward}
\since{3.1.0}

When wrapping an underlying type, it is sometimes desirable to ``keep'' parts of its functionality. Because writing forwarding functions by hand is cumbersome, Haxe allows adding the \expr{:forward} metadata to an abstract type:

\haxe{assets/AbstractExpose.hx}

The \type{MyArray} abstract in this example wraps \type{Array}. Its \expr{:forward} metadata has two arguments which correspond to the field names to be forwarded to the underlying type. In this example, the \expr{main} method instantiates \type{MyArray} and accesses its \expr{push} and \expr{pop} methods. The commented line demonstrates that the \expr{length} field is not available.

As usual we can look at the \target{JavaScript} output to see how the code is being generated:

\lang{js}\begin{lstlisting}
Main.main = function() {
	var myArray = [];
	myArray.push(12);
	myArray.pop();
};
\end{lstlisting}

It is also possible to use \expr{:forward} without any arguments in order to forward all fields. Of course the Haxe Compiler still ensures that the field actually exists on the underlying type.

\trivia{Implemented as macro}{Both the \expr{:enum} and \expr{:forward} functionality were originally implemented using \tref{build macros}{macro-type-building}. While this worked nicely in non-macro code, it caused issues if these features were used from within macros. The implementation was subsequently moved to the compiler.}


\subsection{Core-type abstracts}
\label{types-abstract-core-type}

The Haxe Standard Library defines a set of basic types as core-type abstracts. They are identified by the \expr{:coreType} metadata and the lack of an underlying type declaration. These abstracts can still be understood to represent a different type. Still, that type is native to the Haxe target. 

Introducing custom core-type abstracts is rarely necessary in user code as it requires the Haxe target to be able to make sense of it. However, there could be interesting use-cases for authors of macros and new Haxe targets.

In contrast to opaque abstracts, core-type abstracts have the following properties:

\begin{itemize}
	\item They have no underlying type.
	\item They are considered nullable unless they are annotated with \expr{:notNull} metadata.
	\item They are allowed to declare \tref{array access}{types-abstract-array-access} functions without expressions.
	\item \tref{Operator overloading fields}{types-abstract-operator-overloading} that have no expression are not forced to adhere to the Haxe type semantics.
\end{itemize}



\section{Monomorph}
\label{types-monomorph}

A monomorph is a type which may, through \tref{unification}{type-system-unification}, morph into a different type later. We shall see details about this type when talking about \tref{type inference}{type-system-type-inference}.

\chapter{Type System}
\label{type-system}

We learned about the different kinds of types in \Fullref{types} and it is now time to see how they interact with each other. We start off easy by introducing \tref{typedef}{type-system-typedef}, a mechanism to give a name (or alias) to a more complex type. Among other things, this will come in handy when working with types having \tref{type parameters}{type-system-type-parameters}.

A lot of type-safety is achieved by checking if two given types of the type groups above are compatible. Meaning, the compiler tries to perform \emph{unification} between them as detailed in \Fullref{type-system-unification}.

All types are organized in \emph{modules} and can be addressed through \emph{paths}. \Fullref{type-system-modules-and-paths} will give a detailed explanation of the related mechanics.

\section{Typedef}
\label{type-system-typedef}

We briefly looked at typedefs while talking about \tref{anonymous structures}{types-anonymous-structure} and saw how we could shorten a complex \tref{structure type}{types-anonymous-structure} by giving it a name. This is precisely what typedefs are good for. Giving names to structure types might even be considered their primary use. In fact, it is so common that the distinction appears somewhat blurry and many Haxe users consider typedefs to actually \emph{be} the structure.

A typedef can give a name to any other type:

\begin{lstlisting}
typedef IA = Array<Int>;
\end{lstlisting}
This enables us to use \expr{IA} in places where we would normally use \expr{Array$<$Int$>$}. While this saves only a few keystrokes in this particular case, it can make a much bigger difference for more complex, compound types. Again, this is why typedef and structures seem so connected:

\begin{lstlisting}
typedef User = {
  var age : Int;
  var name : String;
}
\end{lstlisting}
A typedef is not a textual replacement but actually a real type. It can even have \tref{type parameters}{type-system-type-parameters} as the \type{Iterable} type from the Haxe Standard Library demonstrates:

\begin{lstlisting}
typedef Iterable<T> = {
  function iterator() : Iterator<T>;
}
\end{lstlisting}

\paragraph{Optional fields}
Mark the field of a structure as optional using the \ic{@:optional} metadata.
\begin{lstlisting}
typedef User = {
  var age : Int;
  var name : String;
  @:optional var phoneNumber : String;
}
\end{lstlisting}



\section{Type Parameters}
\label{type-system-type-parameters}

Haxe allows parametrization of a number of types, as well as \tref{class fields}{class-field} and \tref{enum constructors}{types-enum-constructor}. Type parameters are defined by enclosing comma-separated type parameter names in angle brackets \expr{$<>$}. A simple example from the Haxe Standard Library is \type{Array}:

\begin{lstlisting}
class Array<T> {
  function push(x : T) : Int;
}
\end{lstlisting}
Whenever an instance of \type{Array} is created, its type parameter \type{T} becomes a \tref{monomorph}{types-monomorph}. That is, it can be bound to any type, but only one at a time. This binding can happen

\begin{description}
	\item[explicitly] by invoking the constructor with explicit types (\expr{new Array$<$String$>$()}) or
	\item[implicitly] by \tref{type inference}{type-system-type-inference}, e.g. when invoking \expr{arrayInstance.push("foo")}.
\end{description}
Inside the definition of a class with type parameters, these type parameters are an unspecific type. Unless \tref{constraints}{type-system-type-parameter-constraints} are added, the compiler has to assume that the type parameters could be used with any type. As a consequence, it is not possible to access fields of type parameters or \tref{cast}{expression-cast} to a type parameter type. It is also not possible to create a new instance of a type parameter type, unless the type parameter is \tref{generic}{type-system-generic} and constrained accordingly. 

The following table shows where type parameters are allowed:

\begin{center}
\begin{tabular}{| l | l | l |}
	\hline
	Parameter on & Bound upon & Notes \\ \hline
	Class & instantiation & Can also be bound upon member field access. \\
	Enum & instantiation & \\
	Enum Constructor & instantiation & \\
	Function & invocation & Allowed for methods and named local lvalue functions. \\
	Structure & instantiation & \\ \hline
\end{tabular}
\end{center}
With function type parameters being bound upon invocation, such a type parameter (if unconstrained) accepts any type. However, only one type per invocation is accepted. This can be utilized if a function has multiple arguments:

\haxe{assets/FunctionTypeParameter.hx}

Both arguments \expr{expected} and \expr{actual} of the \expr{equals} function have type \type{T}. This implies that for each invocation of \expr{equals} the two arguments must be of the same type. The compiler admits the first call (both arguments being of \type{Int}) and the second call (both arguments being of \type{String}) but the third attempt causes a compiler error.

\trivia{Type parameters in expression syntax}{We often get the question why a method with type parameters cannot be called as \expr{method<String>(x)}. The error messages the compiler gives are not very helpful. However, there is a simple reason for that: The above code is parsed as if both \expr{<} and \expr{>} were binary operators, yielding \expr{(method < String) > (x)}.}

\subsection{Constraints}
\label{type-system-type-parameter-constraints}

Type parameters can be constrained to multiple types:

\haxe{assets/Constraints.hx}
Type parameter \type{T} of method \expr{test} is constrained to the types \type{Iterable$<$String$>$} and \type{Measurable}. The latter is defined using a \tref{typedef}{type-system-typedef} for convenience and requires compatible types to have a read-only \tref{property}{class-field-property} named \expr{length} of type \type{Int}. The constraints then say that a type is compatible if

\begin{itemize}
	\item it is compatible with \type{Iterable$<$String$>$} and
	\item has a \expr{length}-property of type \type{Int}.
\end{itemize}
We can see that invoking \expr{test} with an empty array in line 7 and an \type{Array$<$String$>$} in line 8 works fine. This is because \type{Array} has both a \expr{length}-property and an \expr{iterator}-method. However, passing a \type{String} as argument in line 9 fails the constraint check because \type{String} is not compatible with \type{Iterable$<$T$>$}. 


\section{Generic}
\label{type-system-generic}

Usually, the Haxe Compiler generates only a single class or function even if it has type parameters. This results in a natural abstraction where the code generator for the target language has to assume that a type parameter could be of any type. The generated code then might have to perform some type checks which can be detrimental to performance.

A class or function can be made \emph{generic} by attributing it with the \expr{:generic} \tref{metadata}{lf-metadata}. This causes the compiler to emit a distinct class/function per type parameter combination with mangled names. A specification like this can yield a boost in performance-critical code portions on \tref{static targets}{define-static-target} at the cost of a larger output size:

\haxe{assets/GenericClass.hx}

It seems unusual to see the explicit type \type{MyValue<String>} here as we usually let \tref{type inference}{type-system-type-inference} deal with this. Nonetheless, it is indeed required in this case. The compiler has to know the exact type of a generic class upon construction. The \target{JavaScript} output shows the result:

\begin{lstlisting}
(function () { "use strict";
var Test = function() { };
Test.main = function() {
	var a = new MyValue_String("Hello");
	var b = new MyValue_Int(5);
};
var MyValue_Int = function(value) {
	this.value = value;
};
var MyValue_String = function(value) {
	this.value = value;
};
Test.main();
})();
\end{lstlisting}

We can identify that \type{MyValue<String>} and \type{MyValue<Int>} have become \type{MyValue_String} and \type{MyValue_Int} respectively. This is similar for generic functions:

\haxe{assets/GenericFunction.hx}

Again, the \target{JavaScript} output makes it obvious:

\begin{lstlisting}
(function () { "use strict";
var Main = function() { }
Main.method_Int = function(t) {
}
Main.method_String = function(t) {
}
Main.main = function() {
	Main.method_String("foo");
	Main.method_Int(1);
}
Main.main();
})();
\end{lstlisting}


\subsection{Construction of generic type parameters}
\label{type-system-generic-type-parameter-construction}

\define{Generic Type Parameter}{define-generic-type-parameter}{A type parameter is said to be generic if its containing class or method is generic.}

It is not possible to construct normal type parameters, e.g. \expr{new T()} is a compiler error. The reason for this is that Haxe generates only a single function and the construct makes no sense in that case. This is different when the type parameter is generic: Since we know that the compiler will generate a distinct function for each type parameter combination, it is possible to replace the \type{T} \expr{new T()} with the real type.

\haxe{assets/GenericTypeParameter.hx}

It should be noted that \tref{top-down inference}{type-system-top-down-inference} is used here to determine the actual type of \type{T}. There are two requirements for this kind of type parameter construction to work: The constructed type parameter must be

\begin{enumerate}
	\item generic and
	\item be explicitly \tref{constrained}{type-system-type-parameter-constraints} to having a \tref{constructor}{types-class-constructor}.
\end{enumerate}

Here, 1. is given by \expr{make} having the \expr{@:generic} metadata, and 2. by \type{T} being constrained to \type{Constructible}. The constraint holds for both \type{String} and \type{haxe.Template} as both have a constructor accepting a singular \type{String} argument. Sure enough, the relevant \target{JavaScript} output looks as expected:

\begin{lstlisting}
var Main = function() { }
Main.__name__ = true;
Main.make_haxe_Template = function() {
	return new haxe.Template("foo");
}
Main.make_String = function() {
	return new String("foo");
}
Main.main = function() {
	var s = Main.make_String();
	var t = Main.make_haxe_Template();
}
\end{lstlisting}

\section{Variance}
\label{type-system-variance}

While variance is also relevant in other places, it occurs particularly often with type parameters and comes as a surprise in this context. Additionally, it is very easy to trigger variance errors:

\haxe{assets/Variance.hx}

Apparently, an \type{Array<Child>} cannot be assigned to an \type{Array<Base>}, even though \type{Child} can be assigned to \type{Base}. The reason for this might be somewhat unexpected: It is not allowed because arrays can be written to, e.g. via their \expr{push()} method. It is easy to generate problems by ignoring variance errors:

\haxe{assets/Variance2.hx}

Here we subvert the type checker by using a \tref{cast}{expression-cast}, thus allowing the assignment after the commented line. With that we hold a reference \expr{bases} to the original array, typed as \type{Array<Base>}. This allows pushing another type compatible with \type{Base} (\type{OtherChild}) onto that array. However, our original reference \expr{children} is still of type \type{Array<Child>} and things go bad when we encounter the \type{OtherChild} instance in one of its elements while iterating.

If \type{Array} had no \expr{push()} method and no other means of modification, the assignment would be safe because no incompatible type could be added to it. In Haxe, we can achieve this by restricting the type accordingly using \tref{structural subtyping}{type-system-structural-subtyping}:

\haxe{assets/Variance3.hx}

We can safely assign with \expr{b} being typed as \type{MyArray<Base>} and \type{MyArray} only having a \expr{pop()} method. There is no method defined on \type{MyArray} which could be used to add incompatible types, it is thus said to be \emph{covariant}.

\define{Covariance}{define-covariance}{A \tref{compound type}{define-compound-type} is considered covariant if its component types can be assigned to less specific components, i.e. if they are only read, but never written.}

\define{Contravariance}{define-contravariance}{A \tref{compound type}{define-compound-type} is considered contravariant if its component types can be assigned to less generic components, i.e. if they are only written, but never read.}




\section{Unification}
\label{type-system-unification}

\todo{Mention toString()/String conversion somewhere in this chapter.}

Unification is the heart of the type system and contributes immensely to the robustness of Haxe programs. It describes the process of checking if a type is compatible to another type.

\define{Unification}{define-unification}{Unification between two types A and B is a directional process which answers the question if A \emph{can be assigned to} B. It may \emph{mutate} either type if it is or has a \tref{monomorph}{types-monomorph}.}

Unification errors are very easy to trigger:

\begin{lstlisting}
class Main {
  static public function main() {
    // Int should be String
    var s:String = 1;
  }
}
\end{lstlisting}
We try to assign a value of type \type{Int} to a variable of type \type{String}, which causes the compiler to try and \emph{unify Int with String}. This is, of course, not allowed and makes the compiler emit the error \expr{Int should be String}.

In this particular case, the unification is triggered by an \emph{assignment}, a context in which the ``is assignable to'' definition is intuitive. It is one of several cases where unification is performed:

\begin{description}
	\item[Assignment:] If \expr{a} is assigned to \expr{b}, the type of \expr{a} is unified with the type of \expr{b}.
	\item[Function call:] We have briefly seen this one while introducing the \tref{function}{types-function} type. In general, the compiler tries to unify the first given argument type with the first expected argument type, the second given argument type with the second expected argument type and so on until all argument types are handled.
	\item[Function return:] Whenever a function has a \expr{return e} expression, the type of \expr{e} is unified with the function return type. If the function has no explicit return type, it is inferred to the type of \expr{e} and subsequent \expr{return} expressions are inferred against it.
	\item[Array declaration:] The compiler tries to find a minimal type between all given types in an array declaration. Refer to \Fullref{type-system-unification-common-base-type} for details.
	\item[Object declaration:] If an object is declared ``against'' a given type, the compiler unifies each given field type with each expected field type.
	\item[Operator unification:] Certain operators expect certain types which the given types are unified against. For instance, the expression \expr{a \&\& b} unifies both \expr{a} and \expr{b} with \type{Bool} and the expression \expr{a == b} unifies \expr{a} with \expr{b}.
\end{description}


\subsection{Between Class/Interface}
\label{type-system-unification-between-classes-and-interfaces}

When defining unification behavior between classes, it is important to remember that unification is directional: We can assign a more specialized class (e.g. a child class) to a generic class (e.g. a parent class) but the reverse is not valid.

The following assignments are allowed:

\begin{itemize}
	\item child class to parent class
	\item class to implementing interface
	\item interface to base interface
\end{itemize}
These rules are transitive, meaning that a child class can also be assigned to the base class of its base class, an interface its base class implements, the base interface of an implementing interface and so on.
\todo{''parent class'' should probably be used here, but I have no idea what it means, so I will refrain from changing it myself.}

\subsection{Structural Subtyping}
\label{type-system-structural-subtyping}

\define{Structural Subtyping}{define-structural-subtyping}{Structural subtyping defines an implicit relation between types that have the same structure.}

Structural sub-typing in Haxe is allowed when unifying

\begin{itemize}
	\item a \tref{class}{types-class-instance} with a \tref{structure}{types-anonymous-structure} and
	\item a structure with another structure.
\end{itemize}

The following example is part of the \type{Lambda} class of the \tref{Haxe Standard Library}{std}:

\begin{lstlisting}
public static function empty<T>(it : Iterable<T>):Bool {
  return !it.iterator().hasNext();
}
\end{lstlisting}
The \expr{empty}-method checks if an \type{Iterable} has an element. For this purpose, it is not necessary to know anything about the argument type other than the fact that it is considered an iterable. This allows calling the \expr{empty}-method with any type that unifies with \type{Iterable$<$T$>$} which applies to a lot of types in the Haxe Standard Library.

This kind of typing can be very convenient but extensive use may be detrimental to performance on static targets, which  is detailed in \Fullref{types-structure-performance}.


\subsection{Monomorphs}
\label{type-system-monomorphs}

Unification of types having or being a \tref{monomorph}{types-monomorph} is detailed in \Fullref{type-system-type-inference}.


\subsection{Function Return}
\label{type-system-unification-function-return}

Unification of function return types may involve the \tref{\type{Void}-type}{types-void} and requires a clear definition of what unifies with \type{Void}. With \type{Void} describing the absence of a type, it is not assignable to any other type, not even \type{Dynamic}. This means that if a function is explicitly declared as returning \type{Dynamic}, it cannot return \type{Void}.

The opposite applies as well: If a function declares a return type of \type{Void}, it cannot return \type{Dynamic} or any other type. However, this direction of unification is allowed when assigning function types:

\begin{lstlisting}
var func:Void->Void = function() return "foo";
\end{lstlisting}

The right-hand function clearly is of type \type{Void->String}, yet we can assign it to the variable \expr{func} of type \type{Void->Void}. This is because the compiler can safely assume that the return type is irrelevant, given that it could not be assigned to any non-\type{Void} type.


\subsection{Common Base Type}
\label{type-system-unification-common-base-type}

Given a set of multiple types, a \emph{common base type} is a type which all types of the set unify against:

\haxe{assets/UnifyMin.hx}
Although \type{Base} is not mentioned, the Haxe Compiler manages to infer it as the common type of \type{Child1} and \type{Child2}. The Haxe Compiler employs this kind of unification in the following situations:

\begin{itemize}
	\item array declarations
	\item \expr{if}/\expr{else}
	\item cases of a \expr{switch}
\end{itemize}




\section{Type Inference}
\label{type-system-type-inference}

The effects of type inference have been seen throughout this document and will continue to be important. A simple example shows type inference at work:

\haxe{assets/TypeInference.hx}
The special construct \expr{\$type} was previously mentioned in order to simplify the explanation of the \Fullref{types-function} type, so let us now introduce it officially:

%TODO: $type
\define[Construct]{\expr{\$type}}{define-dollar-type}{\expr{\$type} is a compile-time mechanism being called like a function, with a single argument. The compiler evaluates the argument expression and then outputs the type of that expression.}

In the example above, the first \expr{\$type} prints \expr{Unknown<0>}. This is a \tref{monomorph}{types-monomorph}, a type that is not yet known. The next line \expr{x = "foo"} assigns a \type{String} literal to \expr{x}, which causes the \tref{unification}{type-system-unification} of the monomorph with \type{String}. We then see that the type of \expr{x} indeed has changed to \type{String}.

Whenever a type other than \Fullref{types-dynamic} is unified with a monomorph, that monomorph \emph{becomes} that type: it \emph{morphs} into that type. Therefore it cannot morph into a different type afterwards, a property expressed in the \emph{mono} part of its name.

Following the rules of unification, type inference can occur in compound types:

\haxe{assets/TypeInference2.hx}
Variable \expr{x} is first initialized to an empty \type{Array}. At this point we can tell that the type of \expr{x} is an array, but we do not yet know the type of the array elements. Consequentially, the type of \expr{x} is \type{Array<Unknown<0>>}. It is only after pushing a \type{String} onto the array that we know the type to be \type{Array<String>}.


\subsection{Top-down Inference}
\label{type-system-top-down-inference}

Most of the time, types are inferred on their own and may then be unified with an expected type. In a few places, however, an expected type may be used to influence inference. We then speak of \emph{top-down inference}.

\define{Expected Type}{define-expected-type}{Expected types occur when the type of an expression is known before that expression has been typed, e.g. because the expression is argument to a function call. They can influence typing of that expression through what is called \tref{top-down inference}{type-system-top-down-inference}.}

A good example are arrays of mixed types. As mentioned in \Fullref{types-dynamic}, the compiler refuses \expr{[1, "foo"]} because it cannot determine an element type. Employing top-down inference, this can be overcome:

\haxe{assets/TopDownInference.hx}

Here, the compiler knows while typing \expr{[1, "foo"]} that the expected type is \type{Array<Dynamic>}, so the element type is \type{Dynamic}. Instead of the usual unification behavior where the compiler would attempt (and fail) to determine a \tref{common base type}{type-system-unification-common-base-type}, the individual elements are typed against and unified with \type{Dynamic}.

We have seen another interesting use of top-down inference when \tref{construction of generic type parameters}{type-system-generic-type-parameter-construction} was introduced:

\haxe{assets/GenericTypeParameter.hx}

The explicit types \type{String} and \type{haxe.Template} are used here to determine the return type of \expr{make}. This works because the method is invoked as \expr{make()}, so we know the return type will be assigned to the variables. Utilizing this information, it is possible to bind the unknown type \type{T} to \type{String} and \type{haxe.Template} respectively.

% this is not really top down inference
%Top-down inference is also utilized when dealing with \tref{enum constructors}{types-enum-constructor}:

%\haxe{assets/TopDownInference2.hx}

%The constructors \expr{TObject} and \expr{TFunction} of type \expr{ValueType} are recognized even though their containing module \type{Type} is not \tref{imported}{Import}. This is possible because the return type of \expr{Type.typeof("foo")} is known to be \expr{ValueType}.


\subsection{Limitations}
\label{type-system-inference-limitations}

Type inference saves a lot of manual type hints when working with local variables, but sometimes the type system still needs some help. In fact, it does not even try to infer the type of a \tref{variable}{class-field-variable} or \tref{property}{class-field-property} field unless it has a direct initialization.

There are also some cases involving recursion where type inference has limitations. If a function calls itself recursively while its type is not (completely) known yet, type inference may infer a wrong, too specialized type.

A different kind of limitation involves the readability of code. If type inference is overused it might be difficult to understand parts of a program due to the lack of visible types. This is particularly true for method signatures. It is recommended to find a good balance between type inference and explicit type hints.


\section{Modules and Paths}
\label{type-system-modules-and-paths}

\define{Module}{define-module}{All Haxe code is organized in modules, which are addressed using paths. In essence, each .hx file represents a module which may contain several types. A type may be \expr{private}, in which case only its containing module can access it.}

The distinction of a module and its containing type of the same name is blurry by design. In fact, addressing \expr{haxe.ds.StringMap<Int>} can be considered shorthand for \expr{haxe.ds.StringMap.StringMap<Int>}. The latter version consists of four parts:

\begin{enumerate}
	\item the package \expr{haxe.ds}
	\item the module name \expr{StringMap}
	\item the type name \type{StringMap}
	\item the type parameter \type{Int}
\end{enumerate}
If the module and type name are equal, the duplicate can be removed, leading to the \expr{haxe.ds.StringMap<Int>} short version. However, knowing about the extended version helps with understanding how \tref{module sub-types}{type-system-module-sub-types} are addressed.

Paths can be shortened further by using an \tref{import}{type-system-import}, which typically allows omitting the package part of a path. This may lead to usage of unqualified identifiers, for which understanding the \tref{resolution order}{type-system-resolution-order} is required.

\define{Type path}{define-type-path}{The (dot-)path to a type consists of the package, the module name and the type name. Its general form is \expr{pack1.pack2.packN.ModuleName.TypeName}.} 


\subsection{Module Sub-Types}
\label{type-system-module-sub-types}

A module sub-type is a type declared in a module with a different name than that module. This allows a single .hx file to contain multiple types, which can be accessed unqualified from within the module, and by using \expr{package.Module.Type} from other modules:

\begin{lstlisting}
var e:haxe.macro.Expr.ExprDef;
\end{lstlisting}

Here the sub-type \type{ExprDef} within module \expr{haxe.macro.Expr} is accessed. 

The sub-type relation is not reflected at run-time. That is, public sub-types become a member of their containing package, which could lead to conflicts if two modules within the same package tried to define the same sub-type. Naturally, the Haxe compiler detects these cases and reports them accordingly. In the example above \type{ExprDef} is generated as \type{haxe.macro.ExprDef}.

Sub-types can also be made private:

\begin{lstlisting}
private class C { ... }
private enum E { ... }
private typedef T { ... }
private abstract A { ... }
\end{lstlisting}

\define{Private type}{define-private-type}{A type can be made private by using the \expr{private} modifier. As a result, the type can only be directly accessed from within the \tref{module}{define-module} it is defined in.

Private types, unlike public ones, do not become a member of their containing package.}

The accessibility of types can be controlled more fine-grained by using \tref{access control}{lf-access-control}.



\subsection{Import}
\label{type-system-import}

If a type path is used multiple times in a .hx file, it might make sense to use an \expr{import} to shorten it. This allows omitting the package when using the type:

\haxe{assets/Import.hx}

With \expr{haxe.ds.StringMap} being imported in the first line, the compiler is able to resolve the unqualified identifier \expr{StringMap} in the \expr{main} function to this package. The module \type{StringMap} is said to be \emph{imported} into the current file.

In this example, we are actually importing a \emph{module}, not just a specific type within that module. This means that all types defined within the imported module are available:

\haxe{assets/Import2.hx}

The type \type{Binop} is an \tref{enum}{types-enum-instance} declared in the module \type{haxe.macro.Expr}, and thus available after the import of said module. If we were to import only a specific type of that module, e.g. \expr{import haxe.macro.Expr.ExprDef}, the program would fail to compile with \expr{Class not found : Binop}.

There are several aspects worth knowing about importing:

\begin{itemize}
	\item The bottommost import takes priority (detailed in \Fullref{type-system-resolution-order}).
	\item The \tref{static extension}{lf-static-extension} keyword \expr{using} implies the effect of \expr{import}.
	\item If an enum is imported (directly or as part of a module import), all its \tref{enum constructors}{types-enum-constructor} are also imported (this is what allows the \expr{OpAdd} usage in the above example).
\end{itemize}

Furthermore, it is also possible to import \tref{static fields}{class-field} of a class and use them unqualified:

\haxe{assets/Import3.hx}

Special care has to be taken with field names or local variable names that conflict with a package name: Since they take priority over packages, a local variable named \expr{haxe} blocks off usage the entire \expr{haxe} package.

\paragraph{Wildcard import}

Haxe allows using \expr{.*} to allow import of all modules in a package, all types in a module or all static fields in a type. It is important to understand that this kind of import only crosses a single level as we can see in the following example:

\haxe{assets/ImportWildcard.hx}

Using the wildcard import on \expr{haxe.macro} allows accessing \type{Expr} which is a module in this package, but it does not allow accessing \type{ExprDef} which is a sub-type of the \type{Expr} module. This rule extends to static fields when a module is imported.

When using wildcard imports on a package the compiler does not eagerly process all modules in that package. This means that these modules are never actually seen by the compiler unless used explicitly and are then not part of the generated output.

\paragraph{Import with alias}

If a type or static field is used a lot in an importing module it might help to alias it to a shorter name. This can also be used to disambiguate conflicting names by giving them a unique identifier.

\haxe{assets/ImportAlias.hx}

Here we import \expr{String.fromCharCode} as \expr{f} which allows us to use \expr{f(65)} and \expr{f(66)}. While the same could be achieved with a local variable, this method is compile-time exclusive and guaranteed to have no run-time overhead.

\since{3.2.0}

Haxe also allows the more natural \expr{as} in place of \expr{in}.


\subsection{Resolution Order}
\label{type-system-resolution-order}

Resolution order comes into play as soon as unqualified identifiers are involved. These are \tref{expressions}{expression} in the form of \expr{foo()}, \expr{foo = 1} and \expr{foo.field}. The last one in particular includes module paths such as \expr{haxe.ds.StringMap}, where \expr{haxe} is an unqualified identifier.  

We describe the resolution order algorithm here, which depends on the following state:

\begin{itemize}
	\item the declared \tref{local variables}{expression-var} (including function arguments)
	\item the \tref{imported}{type-system-import} modules, types and statics
	\item the available \tref{static extensions}{lf-static-extension}
	\item the kind (static or member) of the current field
	\item the declared member fields on the current class and its parent classes
	\item the declared static fields on the current class
	\item the \tref{expected type}{define-expected-type}
	\item the expression being \expr{untyped} or not
\end{itemize}

\todo{proper label and caption + code/identifier styling for diagram}

\input{assets/tikz/resolution-order.tex}

Given an identifier \expr{i}, the algorithm is as follows:

\begin{enumerate}
	\item If i is \expr{true}, \expr{false}, \expr{this}, \expr{super} or \expr{null}, resolve to the matching constant and halt.
	\item If a local variable named \expr{i} is accessible, resolve to it and halt.
	\item If the current field is static, go to \ref{resolution:static-lookup}.
	\item If the current class or any of its parent classes has a field named \expr{i}, resolve to it and halt.
	\item\label{resolution:static-extension} If a static extension with a first argument of the type of the current class is available, resolve to it and halt.
	\item\label{resolution:static-lookup} If the current class has a static field named \expr{i}, resolve to it and halt.
	\item\label{resolution:enum-ctor} If an enum constructor named \expr{i} is declared on an imported enum, resolve to it and halt.
	\item If a static named \expr{i} is explicitly imported, resolve to it and halt.
	\item If \expr{i} starts with a lower-case character, go to \ref{resolution:untyped}.
	\item\label{resolution:type} If a type named \expr{i} is available, resolve to it and halt.
	\item\label{resolution:untyped} If the expression is not in untyped mode, go to \ref{resolution:failure}
	\item If \expr{i} equals \expr{__this__}, resolve to the \expr{this} constant and halt.
	\item Generate a local variable named \expr{i}, resolve to it and halt.
	\item\label{resolution:failure} Fail
\end{enumerate}

For step \ref{resolution:type}, it is also necessary to define the resolution order of types:

\begin{enumerate}
	\item\label{resolution:import} If a type named \expr{i} is imported (directly or as part of a module), resolve to it and halt.
	\item If the current package contains a module named \expr{i} with a type named \expr{i}, resolve to it and halt.
	\item If a type named \expr{i} is available at top-level, resolve to it and halt.
	\item Fail
\end{enumerate}

For step \ref{resolution:import} of this algorithm as well as steps \ref{resolution:static-extension} and \ref{resolution:enum-ctor} of the previous one, the order of import resolution is important:

\begin{itemize}
	\item Imported modules and static extensions are checked from bottom to top with the first match being picked.
	\item Within a given module, types are checked from top to bottom.
	\item For imports, a match is made if the name equals.
	\item For \tref{static extensions}{lf-static-extension}, a match is made if the name equals and the first argument \tref{unifies}{type-system-unification}. Within a given type being used as static extension, the fields are checked from top to bottom.
\end{itemize}

\chapter{Class Fields}
\label{class-field}

\define{Class Field}{define-class-field}{A class field is a variable, property or method of a class which can either be static or non-static. Non-static fields are referred to as \emph{member} fields, so we speak of e.g. a \emph{static method} or a \emph{member variable}.}

So far we have seen how types and Haxe programs in general are structured. This section about class fields concludes the structural part and at the same time bridges to the behavioral part of Haxe. This is because class fields are the place where \tref{expressions}{expression} are at home.

There are three kinds of class fields:

\begin{description}
	\item[Variable:] A \tref{variable}{class-field-variable} class field holds a value of a certain type, which can be read or written.
	\item[Property:] A \tref{property}{class-field-property} class field defines a custom access behavior for something that, outside the class, looks like a variable field.
	\item[Method:] A \tref{method}{class-field-method} is a function which can be called to execute code.
\end{description}
Strictly speaking, a variable could be considered to be a property with certain access modifiers. Indeed, the Haxe Compiler does not distinguish variables and properties during its typing phase, but they remain separated at syntax level.

Regarding terminology, a method is a (static or non-static) function belonging to a class. Other functions, such as a \tref{local functions}{expression-function} in expressions, are not considered methods.


\section{Variable}
\label{class-field-variable}

We have already seen variable fields in several code examples of previous sections. Variable fields hold values, a characteristic which they share with most (but not all) properties:

\haxe{assets/VariableField.hx}
We can learn from this that a variable

\begin{enumerate}
	\item has a name (here: \expr{member}),
	\item has a type (here: \type{String}),
	\item may have a constant initialization (here: \expr{"bar"}) and
	\item may have \tref{access modifiers}{class-field-access-modifier} (here: \expr{static})
\end{enumerate}

The example first prints the initialization value of \expr{member}, then sets it to \expr{"foo"} before printing its new value. The effect of access modifiers is shared by all three class field kinds and explained in a separate section.

It should be noted that the explicit type is not required if there is an initialization value. The compiler will \tref{infer}{type-system-type-inference} it in this case.

\input{assets/tikz/variable-initialization.tex}


\section{Property}
\label{class-field-property}

Next to \tref{variables}{class-field-variable}, properties are the second option for dealing with data on a class. Unlike variables however, they offer more control of which kind of field access should be allowed and how it should be generated. Common use cases include:

\begin{itemize}
	\item Have a field which can be read from anywhere, but only be written from within the defining class.
	\item Have a field which invokes a \emph{getter}-method upon read-access.
	\item Have a field which invokes a \emph{setter}-method upon write-access.
\end{itemize}

When dealing with properties, it is important to understand the two kinds of access:

\define{Read Access}{define-read-access}{A read access to a field occurs when a right-hand side \tref{field access expression}{expression-field-access} is used. This includes calls in the form of \expr{obj.field()}, where \expr{field} is accessed to be read.}

\define{Write Access}{define-write-access}{A write access to a field occurs when a \tref{field access expression}{expression-field-access} is assigned a value in the form of \expr{obj.field = value}. It may also occur in combination with \tref{read access}{define-read-access} for special assignment operators such as \expr{+=} in expressions like \expr{obj.field += value}.} 

Read access and write access are directly reflected in the syntax, as the following example shows:

\haxe{assets/Property.hx}

For the most part, the syntax is similar to variable syntax, and the same rules indeed apply. Properties are identified by

\begin{itemize}
	\item the opening parenthesis \expr{(} after the field name,
	\item followed by a special \emph{access identifier} (here: \expr{default}),
	\item with a comma \expr{,} separating
	\item another special access identifier (here: \expr{null})
	\item before a closing parenthesis \expr{)}.
\end{itemize}

The access identifiers define the behavior when the field is read (first identifier) and written (second identifier). The accepted values are:

\begin{description}
	\item[\expr{default}:] Allows normal field access if the field has public visibility, otherwise equal to \expr{null} access.
	\item[\expr{null}:] Allows access only from within the defining class.
	\item[\expr{get}/\expr{set}:] Access is generated as a call to an \emph{accessor method}. The compiler ensures that the accessor is available.
	\item[\expr{dynamic}:] Like \expr{get}/\expr{set} access, but does not verify the existence of the accessor field.
	\item[\expr{never}:] Allows no access at all.
\end{description}

\define{Accessor method}{define-accessor-method}{An \emph{accessor method} (or short \emph{accessor}) for a field named \expr{field} of type \type{T} is a \emph{getter} named \expr{get_field} of type \type{Void->T} or a \emph{setter} named \expr{set_field} of type \type{T->T}.}

\trivia{Accessor names}{In Haxe 2, arbitrary identifiers were allowed as access identifiers and would lead to custom accessor method names to be admitted. This made parts of the implementation quite tricky to deal with. In particular, \expr{Reflect.getProperty()} and \expr{Reflect.setProperty()} had to assume that any name could have been used, requiring the target generators to generate meta-information and perform lookups.\\
We disallowed these identifiers and went for the \expr{get_} and \expr{set_} naming convention which greatly simplified implementation. This was one of the breaking changes between Haxe 2 and 3.}

\subsection{Common accessor identifier combinations}
\label{class-field-property-common-combinations}

The next example shows common access identifier combinations for properties:

\haxe{assets/Property2.hx}

The \target{JavaScript} output helps understand what the field access in the \expr{main}-method is compiled to:

\lang{js}\begin{lstlisting}
var Main = function() {
	var v = this.get_x();
	this.set_x(2);
	var _g = this;
	_g.set_x(_g.get_x() + 1);
};
\end{lstlisting}

As specified, the read access generates a call to \expr{get_x()}, while the write access generates a call to \expr{set_x(2)} where \expr{2} is the value being assigned to \expr{x}. The way the \expr{+=} is being generated might look a little odd at first, but can easily be justified by the following example:

\haxe{assets/Property3.hx}

What happens here is that the expression part of the field access to \expr{x} in the \expr{main} method is \emph{complex}: It has potential side-effects, such as the construction of \type{Main} in this case. Thus, the compiler cannot generate the \expr{+=} operation as \expr{new Main().x = new Main().x + 1} and has to cache the complex expression in a local variable:

\lang{js}\begin{lstlisting}
Main.main = function() {
	var _g = new Main();
	_g.set_x(_g.get_x() + 1);
}
\end{lstlisting}



\subsection{Impact on the type system}
\label{class-field-property-type-system-impact}

The presence of properties has several consequences on the type system. Most importantly, it is necessary to understand that properties are a compile-time feature and thus \emph{require the types to be known}. If we were to assign a class with properties to \type{Dynamic}, field access would \emph{not} respect accessor methods. Likewise, access restrictions no longer apply and all access is virtually public.

When using \expr{get} or \expr{set} access identifier, the compiler ensures that the getter and setter actually exists. The following code snippet does not compile:

\haxe{assets/Property4.hx}

The method \expr{get_x} is missing, but it need not be declared on the class defining the property itself as long as a parent class defines it:

\haxe{assets/Property5.hx}

The \expr{dynamic} access modifier works exactly like \expr{get} or \expr{set}, but does not check for the existence



\subsection{Rules for getter and setter}
\label{class-field-property-rules}

Visibility of accessor methods has no effect on the accessibility of its property. That is, if a property is \expr{public} and defined to have a getter, that getter may be defined as \expr{private} regardless.

Both getter and setter may access their physical field for data storage. The compiler ensures that this kind of field access does not go through the accessor method if made from within the accessor method itself, thus avoiding infinite recursion:

\haxe{assets/GetterSetter.hx}

However, the compiler assumes that a physical field exists only if at least one of the access identifiers is \expr{default} or \expr{null}.

\define{Physical field}{define-physical-field}{A field is considered to be \emph{physical} if it is either
	\begin{itemize}
		\item a \tref{variable}{class-field-variable}
		\item a \tref{property}{class-field-property} with the read-access or write-access identifier being \expr{default} or \expr{null}
		\item a \tref{property}{class-field-property} with \expr{:isVar} \tref{metadata}{lf-metadata}
	\end{itemize}
}

If this is not the case, access to the field from within an accessor method causes a compilation error:

\haxe{assets/GetterSetter2.hx}

If a physical field is indeed intended, it can be forced by attributing the field in question with the \expr{:isVar} \tref{metadata}{lf-metadata}:

\haxe{assets/GetterSetter3.hx}


\trivia{Property setter type}{It is not uncommon for new Haxe users to be surprised by the type of a setter being required to be \type{T->T} instead of the seemingly more natural \type{T->Void}. After all, why would a \emph{setter} have to return something?\\
The rationale is that we still want to be able to use field assignments using setters as right-side expressions. Given a chain like \expr{x = y = 1}, it is evaluated as \expr{x = (y = 1)}. In order to assign the result of \expr{y = 1} to \expr{x}, the former must have a value. If \expr{y} had a setter returning \type{Void}, this would not be possible.}


\section{Method}
\label{class-field-method}

While \tref{variables}{class-field-variable} hold data, methods are defining behavior of a program by hosting \tref{expressions}{expression}. We have seen method fields in every code example of this document with even the initial \tref{Hello World}{introduction-hello-world} example containing a \expr{main} method:

\haxe{assets/HelloWorld.hx}

Methods are identified by the \expr{function} keyword. We can also learn that they

\begin{enumerate}
	\item have a name (here: \expr{main}),
	\item have an argument list (here: empty \expr{()}),
	\item have a return type (here: \type{Void}),
	\item may have \tref{access modifiers}{class-field-access-modifier} (here: \expr{static} and \expr{public}) and
	\item may have an expression (here: \expr{\{trace("Hello World");\}}).
\end{enumerate}

We can also look at the next example to learn more about arguments and return types:

\haxe{assets/MethodField.hx}

Arguments are given by an opening parenthesis \expr{(} after the field name, a comma \expr{,} separated list of argument specifications and a closing parenthesis \expr{)}. Additional information on the argument specification is described in \Fullref{types-function}.

The example demonstrates how \tref{type inference}{type-system-type-inference} can be used for both argument and return types. The method \expr{myFunc} has two arguments but only explicitly gives the type of the first one, \expr{f}, as \type{String}. The second one, \expr{i}, is not type-hinted and it is left to the compiler to infer its type from calls made to it. Likewise, the return type of the method is inferred from the \expr{return true} expression as \type{Bool}.

\subsection{Overriding Methods}
\label{class-field-overriding}

Overriding fields is instrumental for creating class hierarchies. Many design patterns utilize it, but here we will explore only the basic functionality. In order to use overrides in a class, it is required that this class has a \tref{parent class}{types-class-inheritance}. Let us consider the following example:

\haxe{assets/Override.hx}

The important components here are:

\begin{itemize}
	\item the class \type{Base} which has a method \expr{myMethod} and a constructor,
	\item the class \type{Child} which \expr{extends Base} and also has a method \expr{myMethod} being declared with \expr{override}, and
	\item the \type{Main} class whose \expr{main} method creates an instance of \expr{Child}, assigns it to a variable \expr{child} of explicit type \type{Base} and calls \expr{myMethod()} on it.
\end{itemize}

The variable \expr{child} is explicitly typed as \type{Base} to highlight an important difference: At compile-time the type is known to be \type{Base}, but the runtime still finds the correct method \expr{myMethod} on class \type{Child}. This is because field access is resolved dynamically at runtime.

The \type{Child} class can access methods it has overridden by calling \expr{super.methodName()}:

\haxe{assets/OverrideCallParent.hx}

The section on \Fullref{types-class-inheritance} explains the use of \expr{super()} from within a \expr{new} constructor.


\subsection{Effects of variance and access modifiers}
\label{class-field-override-effects}

Overriding adheres to the rules of \tref{variance}{type-system-variance}. That is, their argument types allow \emph{contravariance} (less specific types) while their return type allows \emph{covariance} (more specific types):

\haxe{assets/OverrideVariance.hx}

Intuitively, this follows from the fact that arguments are ``written to'' the function and the return value is ``read from'' it.

The example also demonstrates how \tref{visibility}{class-field-visibility} may be changed: An overriding field may be \expr{public} if the overridden field is \expr{private}, but not the other way around.

It is not possible to override fields which are declared as \tref{\expr{inline}}{class-field-inline}. This is due to the conflicting concepts: While inlining is done at compile-time by replacing a call with the function body, overriding fields necessarily have to be resolved at runtime.
	
	
	
\section{Access Modifier}
\label{class-field-access-modifier}
\state{NoContent}

\subsection{Visibility}
\label{class-field-visibility}

Fields are by default \emph{private}, meaning that only the class and its sub-classes may access them. They can be made \emph{public} by using the \expr{public} access modifier, allowing access from anywhere.

\haxe{assets/Visibility.hx}

Access to field \expr{available} of class \type{MyClass} is allowed from within \type{Main} because it is denoted as being \expr{public}. However, while access to field \expr{unavailable} is allowed from within class \type{MyClass}, it is not allowed from within class \type{Main} because it is \expr{private} (explicitly, although this identifier is redundant here).

The example demonstrates visibility through \emph{static} fields, but the rules for member fields are equivalent. The following example demonstrates visibility behavior for when \tref{inheritance}{types-class-inheritance} is involved.

\haxe{assets/Visibility2.hx}

We can see that access to \expr{child1.baseField()} is allowed from within \type{Child2} even though \expr{child1} is of a different type, \type{Child1}. This is because the field is defined on their common ancestor class \type{Base}, contrary to field \expr{child1Field} which can not be accessed from within \type{Child2}.

Omitting the visibility modifier usually defaults the visibility to \expr{private}, but there are exceptions where it becomes \expr{public} instead:

\begin{enumerate}
	\item If the class is declared as \expr{extern}.
	\item If the field is declared on an \tref{interface}{types-interfaces}.
	\item If the field \tref{overrides}{class-field-overriding} a public field.
	\item If the class has metadata \ic{@:publicFields}, which forces all class fields of inheriting classes to be public.
\end{enumerate}

\trivia{Protected}{Haxe has no notion of a \expr{protected} keyword known from Java, C++ and other object-oriented languages. However, its \expr{private} behavior is equal to those language's protected behavior, so Haxe actually lacks their real private behavior.}

\subsection{Inline}
\label{class-field-inline}

The \expr{inline} keyword allows function bodies to be directly inserted in place of calls to them. This can be a powerful optimization tool, but should be used judiciously as not all functions are good candidates for inline behavior. The following example demonstrates the basic usage:

\haxe{assets/Inline.hx}

The generated \target{JavaScript} output reveals the effect of inline:

\lang{js}\begin{lstlisting}
(function () { "use strict";
var Main = function() { }
Main.main = function() {
	var a = 1;
	var b = 2;
	var c = (a + b) / 2;
}
Main.main();
})();
\end{lstlisting}

As evident, the function body \expr{(s1 + s2) / 2} of field \expr{mid} was generated in place of the call to \expr{mid(a, b)}, with \expr{s1} being replaced by \expr{a} and \expr{s2} being replaced by \expr{b}. This avoids a function call which, depending on the target and frequency of occurrences, may yield noticeable performance improvements.

It is not always easy to judge if a function qualifies for being inline. Short functions that have no writing expressions (such as a \expr{=} assignment) are usually a good choice, but even more complex functions can be candidates. However, in some cases inlining can actually be detrimental to performance, e.g. because the compiler has to create temporary variables for complex expressions.

Inline is not guaranteed to be done. The compiler might cancel inlining for various reasons or a user could supply the \ic{--no-inline} command line argument to disable inlining. The only exception is if the class is \tref{extern}{lf-externs} or if the class field has the \expr{:extern} \tref{metadata}{lf-metadata}, in which case inline is forced. If it cannot be done, the compiler emits an error.

It is important to remember this when relying on inline:

\haxe{assets/InlineRelying.hx}

If the call to \expr{error} is inlined the program compiles correctly because the control flow checker is satisfied due to the inlined \tref{throw}{expression-throw} expression. If inline is not done, the compiler only sees a function call to \expr{error} and emits the error \expr{A return is missing here}.


\subsection{Dynamic}
\label{class-field-dynamic}

Methods can be denoted with the \expr{dynamic} keyword to make them (re-)bindable:

\haxe{assets/DynamicFunction.hx}

The first call to \expr{test()} invokes the original function which returns the \type{String} \expr{"original"}. In the next line, \expr{test} is \emph{assigned} a new function. This is precisely what \expr{dynamic} allows: Function fields can be assigned a new function. As a result, the next invocation of \expr{test()} returns the \type{String} \expr{"new"}.

Dynamic fields cannot be \expr{inline} for obvious reasons: While inlining is done at compile-time, dynamic functions necessarily have to be resolved at runtime.

%TODO: performance estimation %

\subsection{Override}
\label{class-field-override}

The access modifier \expr{override} is required when a field is declared which also exists on a \tref{parent class}{types-class-inheritance}. Its purpose is to ensure that the author of a class is aware of the override as this may not always be obvious in large class hierarchies. Likewise, having \expr{override} on a field which does not actually override anything (e.g. due to a misspelled field name) triggers an error.

The effects of overriding fields are detailed in \Fullref{class-field-overriding}. This modifier is only allowed on \tref{method}{class-field-method} fields.

\subsection{Static}
\label{class-field-static}

All fields are member fields unless the modifier \expr{static} is used. Static fields are used ``on the class'' whereas non-static fields are used ``on a class instance'':

\haxe{assets/StaticField.hx}

Static \tref{variable}{class-field-variable} and \tref{property}{class-field-property} fields can have arbitrary initialization \tref{expressions}{expression}.

\chapter{Expressions}
\label{expression}

Expressions in Haxe define what a program \emph{does}. Most expressions are found in the body of a \tref{method}{class-field-method}, where they are combined to express what that method should do. This section explains the different kinds of expressions. Some definitions help here:

\define{Name}{define-name}{A general name may refer to
\begin{itemize}
	\item a type,
	\item a local variable,
	\item a local function or
	\item a field.
\end{itemize}}

\define{Identifier}{define-identifier}{Haxe identifiers start with an underscore \expr{_}, a dollar \expr{\$}, a lower-case character \expr{a-z} or an upper-case character \expr{A-Z}. After that, any combination and number of \expr{_}, \expr{A-Z}, \expr{a-z} and \expr{0-9} may follow.\\
Further limitations follow from the usage context, which are checked upon typing:
\begin{itemize}
	\item Type names must start with an upper-case letter \expr{A-Z} or an underscore \expr{_}.
	\item Leading dollars are not allowed for any kind of \tref{name}{define-name} (dollar-names are mostly used for \tref{macro reification}{macro-reification}).
\end{itemize}}

\since{3.3.0}

Haxe reserves the identifier prefix \expr{_hx_} for internal use. This is not enforced by the parser or typer.

\paragraph{Keywords}
\label{expression-keywords}

The following Haxe keywords may not be used as identifiers:

\begin{itemize}
	\item abstract
	\item break
	\item case
	\item cast
	\item catch
	\item class
	\item continue
	\item default
	\item do
	\item dynamic
	\item else
	\item enum
	\item extends
	\item extern
	\item false
	\item for
	\item function
	\item if
	\item implements
	\item import
	\item in
	\item inline
	\item interface
	\item macro
	\item new
	\item null
	\item override
	\item package
	\item private
	\item public
	\item return
	\item static
	\item switch
	\item this
	\item throw
	\item true
	\item try
	\item typedef
	\item untyped
	\item using
	\item var
	\item while
\end{itemize}


\paragraph{Related content}
\begin{itemize}
	\item Haxe Code Cookbook  article: \href{http://code.haxe.org/category/principles/everything-is-an-expression.html}{Everything is an expression}.
\end{itemize}


\section{Blocks}
\label{expression-block}

A block in Haxe starts with an opening curly brace \expr{\{} and ends with a closing curly brace \expr{\}}. A block may contain several expressions, each of which is followed by a semicolon \expr{;}. The general syntax is thus:

\begin{lstlisting}
{
	expr1;
	expr2;
	...
	exprN;
}
\end{lstlisting}
The value and by extension the type of a block-expression is equal to the value and the type of the last sub-expression.

Blocks can contain local variables declared by \tref{\expr{var} expression}{expression-var}, as well as local functions declared by \tref{\expr{function} expressions}{expression-function}. These are available within the block and within sub-blocks, but not outside the block. Also, they are available only after their declaration. The following example uses \expr{var}, but the same rules apply to \expr{function} usage:

\begin{lstlisting}
{
	a; // error, a is not declared yet
	var a = 1; // declare a
	a; // ok, a was declared
	{
		a; // ok, a is available in sub-blocks
	}
  // ok, a is still available after
	// sub-blocks	
	a;
}
a; // error, a is not available outside
\end{lstlisting}
At runtime, blocks are evaluated from top to bottom. Control flow (e.g. \tref{exceptions}{expression-try-catch} or \tref{return expressions}{expression-return}) may leave a block before all expressions
are evaluated.

\paragraph{Variable Shadowing}

Haxe allows local variable shadowing within the same block. This means that
a \expr{var} or \expr{function} can be declared with the same name that was
previously available in a block, effectively hiding it from the further code:

\begin{lstlisting}
{
	var v = 42; // declare v
	$type(v); // Int
	var v = "hi"; // declare a new v
	$type(v); // String, previous declaration is not available
}
\end{lstlisting}

It might come as a surprise that this is allowed, but it's useful to avoid pollution of local name space and thus prevent accidental usage of a wrong variable.

Note, that the shadowing strictly follows syntax, so if a variable was captured
in a closure before it was shadowed, that closure would still reference the
original declaration:

\begin{lstlisting}
{
	var a = 1;
	function f() {
	    trace(a);
	}
	var a = 2;
	f(); // traces 1
}
\end{lstlisting}


\section{Constants}
\label{expression-constants}

The Haxe syntax supports the following constants:

\begin{description}
	\item[Int:] An \tref{integer}{define-int}, such as \expr{0}, \expr{1}, \expr{97121}, \expr{-12}, \expr{0xFF0000}.
	\item[Float:] A \tref{floating point number}{define-float}, such as \expr{0.0}, \expr{1.}, \expr{.3}, \expr{-93.2}.
	\item[String:] A \tref{string of characters}{define-string}, such as \expr{""}, \expr{"foo"}, \expr{'{'}}, \expr{'bar'}.
	\item[true,false:] A \tref{boolean}{define-bool} value.
	\item[null:] The null value.
\end{description}

Furthermore, the internal syntax structure treats \tref{identifiers}{define-identifier} as constants, which may be relevant when working with \tref{macros}{macro}.


\section{Binary Operators}
\label{expression-binops}

\section{Unary Operators}
\label{expression-unops}

\section{Array Declaration}
\label{expression-array-declaration}

Arrays are initialized by enclosing comma \expr{,} separated values in brackets \expr{[]}. A plain \expr{[]} represents the empty array, whereas \expr{[1, 2, 3]} initializes an array with three elements \expr{1}, \expr{2} and \expr{3}.

The generated code may be less concise on platforms that do not support array initialization. Essentially, such initialization code then looks like this:

\begin{lstlisting}
var a = new Array();
a.push(1);
a.push(2);
a.push(3);
\end{lstlisting}
This should be considered when deciding if a function should be \tref{inlined}{class-field-inline} as it may inline more code than visible in the syntax.

Advanced initialization techniques are described in \Fullref{lf-array-comprehension}.


\section{Object Declaration}
\label{expression-object-declaration}

Object declaration begins with an opening curly brace \expr{\{} after which \expr{key:value}-pairs separated by comma \expr{,} follow, and which ends in a closing curly brace \expr{\}}.

\begin{lstlisting}
{
	key1:value1,
	key2:value2,
	...
	keyN:valueN
}
\end{lstlisting}
Further details of object declaration are described in the section about \tref{anonymous structures}{types-anonymous-structure}.


\section{Field Access}
\label{expression-field-access}

Field access is expressed by using the dot \expr{.} followed by the name of the field.

\begin{lstlisting}
object.fieldName
\end{lstlisting}

This syntax is also used to access types within packages in the form of \expr{pack.Type}.

The typer ensures that an accessed field actually exist and may apply transformations depending on the nature of the field. If a field access is ambiguous, understanding the \tref{resolution order}{type-system-resolution-order} may help.


\section{Array Access}
\label{expression-array-access}

Array access is expressed by using an opening bracket \expr{[} followed by the index expression and a closing bracket \expr{]}.

\begin{lstlisting}
expr[indexExpr]
\end{lstlisting}

This notation is allowed with arbitrary expressions, but at typing level only certain combinations are admitted:

\begin{itemize}
	\item \expr{expr} is of \type{Array} or \type{Dynamic} and \expr{indexExpr} is of \type{Int}
	\item \expr{expr} is an \tref{abstract type}{types-abstract} which defines a matching \tref{array access}{types-abstract-array-access}
\end{itemize}


\section{Function Call}
\label{expression-function-call}

Functions calls consist of an arbitrary subject expression followed by an opening parenthesis \expr{(}, a comma \expr{,} separated list of expressions as arguments and a closing parenthesis \expr{)}.

\begin{lstlisting}
subject(); // call with no arguments
subject(e1); // call with one argument
subject(e1, e2); // call with two arguments
// call with multiple arguments
subject(e1, ..., eN);
\end{lstlisting}


\paragraph{Related content}
\begin{itemize}
	\item Haxe Code Cookbook article: \href{http://code.haxe.org/category/beginner/declare-functions.html}{How to declare functions}
	\item Class Methods: \Fullref{class-field-method}
\end{itemize}


\section{var}
\label{expression-var}

The \expr{var} keyword allows declaring multiple variables, separated by comma \expr{,}. Each variable has a valid \tref{identifier}{define-identifier} and optionally a value assignment following the assignment operator \expr{=}. Variables can also have an explicit type-hint.

\begin{lstlisting}
var a; // declare local a
var b:Int; // declare variable b of type Int
// declare variable c, initialized to value 1
var c = 1;
// declare an uninitialized variable d
// and variable e initialized to value 2
var d,e = 2;
\end{lstlisting}

The scoping behavior of local variables is described in \Fullref{expression-block}.


\section{Local functions}
\label{expression-function}

Haxe supports first-class functions and allows declaring local functions in expressions. The syntax follows \tref{class field methods}{class-field-method}:

\haxe{assets/LocalFunction.hx}

We declare \expr{myLocalFunction} inside the \tref{block expression}{expression-block} of the \expr{main} class field. It takes one argument \expr{i} and adds it to \expr{value}, which is defined in the outside scope.

The scoping is equivalent to that of \tref{variables}{expression-var} and for the most part writing a named local function can be considered equal to assigning an unnamed local function to a local variable:

\begin{lstlisting}
var myLocalFunction = function(a) { }
\end{lstlisting}

However, there are some differences related to type parameters and the position of the function. We speak of a ``lvalue'' function if it is not assigned to anything upon its declaration, and an ``rvalue'' function otherwise.

\begin{itemize}
	\item Lvalue functions require a name and can have \tref{type parameters}{type-system-type-parameters}.
	\item Rvalue functions may have a name, but cannot have type parameters.
\end{itemize}


\section{new}
\label{expression-new}

The \expr{new} keyword signals that a \tref{class}{types-class-instance} or an \tref{abstract}{types-abstract} is being instantiated. It is followed by the \tref{type path}{define-type-path} of the type which is to be instantiated. It may also list explicit \tref{type parameters}{type-system-type-parameters} enclosed in \expr{<>} and separated by comma \expr{,}. After an opening parenthesis \expr{(} follow the constructor arguments, again separated by comma \expr{,}, with a closing parenthesis \expr{)} at the end.

\haxe{assets/New.hx}

Within the \expr{main} method we instantiate an instance of \type{Main} itself, with an explicit type parameter \type{Int} and the arguments \expr{12} and \expr{"foo"}. As we can see, the syntax is very similar to the \tref{function call syntax}{expression-function-call} and it is common to speak of ``constructor calls''.



\section{for}
\label{expression-for}

Haxe does not support traditional for-loops known from C. Its \expr{for} keyword expects an opening parenthesis \expr{(}, then a variable identifier followed by the keyword \expr{in} and an arbitrary expression used as iterating collection. After the closing parenthesis \expr{)} follows an arbitrary loop body expression.

\begin{lstlisting}
for (v in e1) e2;
\end{lstlisting}

The typer ensures that the type of \expr{e1} can be iterated over, which is typically the case if it has an  \tref{iterator}{lf-iterators} method returning an \type{Iterator<T>}, or if it is an \type{Iterator<T>} itself.

Variable \expr{v} is then available within loop body \expr{e2} and holds the value of the individual elements of collection \expr{e1}.

Haxe has a special range operator to iterate over intervals. It is a binary operator taking two \type{Int} operands: \expr{min...max} returns an \href{http://api.haxe.org/IntIterator.html}{IntIterator} instance that iterates from \expr{min} (inclusive) to \expr{max} (exclusive). Note that \expr{max} may not be smaller than \expr{min}.

\begin{lstlisting}
for (i in 0...10) trace(i); // 0 to 9
\end{lstlisting}

The type of a \expr{for} expression is always \type{Void}, meaning it has no value and cannot be used as right-side expression.

The control flow of loops can be affected by \tref{\expr{break}}{expression-break} and \tref{\expr{continue}}{expression-continue} expressions.

\paragraph{Related content}
\begin{itemize}
	\item Manual: \tref{Haxe iterators documentation}{lf-iterators}, \tref{Haxe Data Structures documentation}{std-ds}
	\item Cookbook: \href{http://code.haxe.org/tag/iterator.html}{Haxe iterators examples}, \href{http://code.haxe.org/tag/data-structures.html}{Haxe data structures examples}
\end{itemize}

\section{while}
\label{expression-while}

A normal while loop starts with the \expr{while} keyword, followed by an opening parenthesis \expr{(}, the condition expression and a closing parenthesis \expr{)}. After that follows the loop body expression:

\begin{lstlisting}
while(condition) expression;
\end{lstlisting}

The condition expression has to be of type \type{Bool}.

Upon each iteration, the condition expression is evaluated. If it evaluates to \expr{false}, the loop stops, otherwise it evaluates the loop body expression.

\haxe{assets/While.hx}

This kind of while-loop is not guaranteed to evaluate the loop body expression at all: If the condition does not hold from the start, it is never evaluated. This is different for \tref{do-while loops}{expression-do-while}.

\section{do-while}
\label{expression-do-while}

A do-while loop starts with the \expr{do} keyword followed by the loop body expression. After that follows the \expr{while} keyword, an opening parenthesis \expr{(}, the condition expression and a closing parenthesis \expr{)}:

\begin{lstlisting}
do expression while(condition);
\end{lstlisting}

The condition expression has to be of type \type{Bool}.

As the syntax suggests, the loop body expression is always evaluated at least once, unlike \tref{while}{expression-while} loops.

\section{if}
\label{expression-if}

Conditional expressions come in the form of a leading \expr{if} keyword, a condition expression enclosed in parentheses \expr{()} and a expression to be evaluated in case the condition holds:

\begin{lstlisting}
if (condition) expression;
\end{lstlisting}

The condition expression has to be of type \type{Bool}.

Optionally, \expr{expression} may be followed by the \expr{else} keyword as well as another expression to be evaluated if the condition does not hold:

\begin{lstlisting}
if (condition) expression1 else expression2;
\end{lstlisting}

Here, \expr{expression2} may consist of another \expr{if} expression:

\begin{lstlisting}
if (condition1) expression1
else if(condition2) expression2
else expression3
\end{lstlisting}

If the value of an \expr{if} expression is required, e.g. for \expr{var x = if(condition) expression1 else expression2}, the typer ensures that the types of \expr{expression1} and \expr{expression2} \tref{unify}{type-system-unification}. If no \expr{else} expression is given, the type is inferred to be \type{Void}.


\section{switch}
\label{expression-switch}

A basic switch expression starts with the \expr{switch} keyword and the switch subject expression, as well as the case expressions between curly braces \expr{\{\}}. Case expressions either start with the \expr{case} keyword and are followed by a pattern expression, or consist of the \expr{default} keyword. In both cases a colon \expr{:} and an optional case body expression follows:

\begin{lstlisting}
switch subject {
	case pattern1: case-body-expression-1;
	case pattern2: case-body-expression-2;
	default: default-expression;
}
\end{lstlisting}

Case body expressions never ``fall through'', so the \tref{\expr{break}}{expression-break} keyword is not supported in Haxe.

Switch expressions can be used as value; in that case the types of all case body expressions and the default expression must \tref{unify}{type-system-unification}.

\paragraph{Related content}
\begin{itemize}
	\item Further details on syntax of pattern expressions are detailed in \Fullref{lf-pattern-matching}.
	\item \href{http://code.haxe.org/tag/pattern-matching.html}{Snippets and tutorials about pattern matching} in the Haxe Code Cookbook.
\end{itemize}


\section{try/catch}
\label{expression-try-catch}

Haxe allows catching values using its \expr{try/catch} syntax:

\begin{lstlisting}
try try-expr
catch(varName1:Type1) catch-expr-1
catch(varName2:Type2) catch-expr-2
\end{lstlisting}

If during runtime the evaluation of \expr{try-expression} causes a \tref{\expr{throw}}{expression-throw}, it can be caught by any subsequent \expr{catch} block. These blocks consist of

\begin{itemize}
	\item a variable name which holds the thrown value,
	\item an explicit type annotation which determines which types of values to catch, and
	\item the expression to execute in that case.
\end{itemize}

Haxe allows throwing and catching any kind of value, it is not limited to types inheriting from a specific exception or error class. Catch blocks are checked from top to bottom with the first one whose type is compatible with the thrown value being picked.

This process has many similarities to the compile-time \tref{unification}{type-system-unification} behavior. However, since the check has to be done at runtime there are several restrictions:

\begin{itemize}
	\item The type must exist at runtime: \tref{Class instances}{types-class-instance}, \tref{enum instances}{types-enum-instance}, \tref{abstract core types}{types-abstract-core-type} and \tref{Dynamic}{types-dynamic}.
	\item Type parameters can only be \tref{Dynamic}{types-dynamic}.
\end{itemize}



\section{return}
\label{expression-return}

A \expr{return} expression can come with or without an value expression:

\begin{lstlisting}
return;
return expression;
\end{lstlisting}

It leaves the control-flow of the innermost function it is declared in, which has to be distinguished when \tref{local functions}{expression-function} are involved:

\begin{lstlisting}
function f1() {
	function f2() {
		return;
	}
	f2();
	expression;
}
\end{lstlisting}

The \expr{return} leaves local function \expr{f2}, but not \expr{f1}, meaning \expr{expression} is still evaluated.

If \expr{return} is used without a value expression, the typer ensures that the return type of the function it returns from is of \type{Void}. If it has a value expression, the typer \tref{unifies}{type-system-unification} its type with the return type (explicitly given or inferred by previous \expr{return} expressions) of the function it returns from.


\section{break}
\label{expression-break}

The \expr{break} keyword leaves the control flow of the innermost loop (\expr{for} or \expr{while}) it is declared in, stopping further iterations:

\begin{lstlisting}
while(true) {
	expression1;
	if (condition) break;
	expression2;
}
\end{lstlisting}

Here, \expr{expression1} is evaluated for each iteration, but as soon as \expr{condition} holds, \expr{expression2} is not evaluated anymore.

The typer ensures that it appears only within a loop. The \expr{break} keyword in \tref{\expr{switch} cases}{expression-switch} is not supported in Haxe.


\section{continue}
\label{expression-continue}

The \expr{continue} keyword ends the current iteration of the innermost loop (\expr{for} or \expr{while}) it is declared in, causing the loop condition to be checked for the next iteration:

\begin{lstlisting}
while(true) {
	expression1;
	if(condition) continue;
	expression2;
}
\end{lstlisting}

Here, \expr{expression1} is evaluated for each iteration, but if \expr{condition} holds, \expr{expression2} is not evaluated for the current iteration. Unlike \expr{break}, iterations continue.

The typer ensures that it appears only within a loop.


\section{throw}
\label{expression-throw}

Haxe allows throwing any kind of value using its \expr{throw} syntax:

\begin{lstlisting}
throw expr
\end{lstlisting}

A value which is thrown like this can be caught by \tref{\expr{catch} blocks}{expression-try-catch}. If no such block catches it, the behavior is target-dependent.


\section{cast}
\label{expression-cast}

Haxe allows two kinds of casts:

\begin{lstlisting}
cast expr; // unsafe cast
cast (expr, Type); // safe cast
\end{lstlisting}

\subsection{unsafe cast}
\label{expression-cast-unsafe}

Unsafe casts are useful to subvert the type system. The compiler types \expr{expr} as usual and then wraps it in a \tref{monomorph}{types-monomorph}. This allows the expression to be assigned to anything.

Unsafe casts do not introduce any \tref{dynamic}{types-dynamic} types, as the following example shows:

\haxe{assets/UnsafeCast.hx}

Variable \expr{i} is typed as \type{Int} and then assigned to variable \expr{s} using the unsafe cast \expr{cast i}. This causes \expr{s} to be of an unknown type, a monomorph. Following the usual rules of \tref{unification}{type-system-unification}, it can then be bound to any type, such as \type{String} in this example.

These casts are called "unsafe" because the runtime behavior for invalid casts is not defined. While most \tref{dynamic targets}{define-dynamic-target} are likely to work, it might lead to undefined errors on \tref{static targets}{define-static-target}.

Unsafe casts have little to no runtime overhead.

\subsection{safe cast}
\label{expression-cast-safe}

Unlike \tref{unsafe casts}{expression-cast-unsafe}, the runtime behavior in case of a failing cast is defined for safe casts:

\haxe{assets/SafeCast.hx}

In this example we first cast a class instance of type \type{Child1} to \type{Base}, which succeeds because \type{Child1} is a \tref{child class}{types-class-inheritance} of \type{Base}. We then try to cast the same class instance to \type{Child2}, which is not allowed because instances of \type{Child2} are not instances of \type{Child1}.

The Haxe compiler guarantees that an exception of type \type{String} is \tref{thrown}{expression-throw} in this case. This exception can be caught using a \tref{\expr{try/catch} block}{expression-try-catch}.

Safe casts have a runtime overhead. It is important to understand that the compiler already generates type checks, so it is redundant to add manual checks, e.g. using \expr{Std.is}. The intended usage is to try the safe cast and catch the \type{String} exception.


\section{type check}
\label{expression-type-check}
\since{3.1.0}

It is possible to employ compile-time type checks using the following syntax:

\begin{lstlisting}
(expr : type)
\end{lstlisting}

The parentheses are mandatory. Unlike \tref{safe casts}{expression-cast-safe} this construct has no run-time impact. It has two compile-time implications:

\begin{enumerate}
\item \tref{Top-down inference}{type-system-top-down-inference} is used to type \expr{expr} with type \expr{type}.
\item The resulting typed expression is \tref{unified}{type-system-unification} with type \expr{type}.
\end{enumerate}

This has the usual effect of both operations such as the given type being used as expected type when performing \tref{unqualified identifier resolution}{type-system-resolution-order} and the unification checking for \tref{abstract casts}{types-abstract-implicit-casts}.

\chapter{Language Features}
\label{lf}

\emph{\tref{Abstract types}{types-abstract}:}

An abstract type is a compile-time construct which is represented in a different way at runtime. This allows giving a whole new meaning to existing types.

\emph{\tref{Extern classes}{lf-externs}:}

Externs can be used to describe target-specific interaction in a type-safe manner.

\emph{\tref{Anonymous structures}{types-anonymous-structure}:}

Data can easily be grouped in anonymous structures, minimizing the necessity of small data classes.

\begin{lstlisting}
var point = { x: 0, y: 10 };
point.x += 10;
\end{lstlisting}

\emph{\tref{Array Comprehension}{lf-array-comprehension}:}

Create and populate arrays quickly using for loops and logic.

\begin{lstlisting}
var evenNumbers = [ for (i in 0...100) if (i\%2==0) i ];
\end{lstlisting}

\emph{\tref{Classes, interfaces and inheritance}{types-class-instance}:}

Haxe allows structuring code in classes, making it an object-oriented language. Common related features known from languages such as Java are supported, including inheritance and interfaces.

\emph{\tref{Conditional compilation}{lf-condition-compilation}:}

Conditional Compilation allows compiling specific code depending on compilation parameters. This is instrumental for abstracting target-specific differences, but can also be used for other purposes, such as more detailed debugging.

\begin{lstlisting}
\#if js
    js.Browser.alert("Hello");
\#elseif sys
    Sys.println("Hello");
\#end
\end{lstlisting}

\emph{\tref{(Generalized) Algebraic Data Types}{types-enum-instance}:}

Structure can be expressed through algebraic data types (ADT), which are known as enums in the Haxe Language. Furthermore, Haxe supports their generalized variant known as GADT.

\begin{lstlisting}
enum Result {
    Success(data:Array<Int>);
    UserError(msg:String);
    SystemError(msg:String, position:PosInfos);
}
\end{lstlisting}

\emph{\tref{Inlined calls}{class-field-inline}:}

Functions can be designated as being inline, allowing their code to be inserted at call-site. This can yield significant performance benefits without resorting to code duplication via manual inlining.

\emph{\tref{Iterators}{lf-iterators}:}

Iterating over a set of values, e.g. the elements of an array, is very easy in Haxe courtesy of iterators. Custom classes can quickly implement iterator functionality to allow iteration.

\begin{lstlisting}
for (i in [1, 2, 3]) {
    trace(i);
}
\end{lstlisting}

\emph{\tref{Local functions and closures}{expression-function}:}

Functions in Haxe are not limited to class fields and can be declared in expressions as well, allowing powerful closures.

\begin{lstlisting}
var buffer = "";
function append(s:String) {
    buffer += s;
}
append("foo");
append("bar");
trace(buffer); // foobar
\end{lstlisting}

\emph{\tref{Metadata}{lf-metadata}:}

Add metadata to fields, classes or expressions. This can communicate information to the compiler, macros, or runtime classes.

\begin{lstlisting}
class MyClass {
    @range(1, 8) var value:Int;
}
trace(haxe.rtti.Meta.getFields(MyClass).value.range); // [1,8]
\end{lstlisting}

\emph{\tref{Static Extensions}{lf-static-extension}:}

Existing classes and other types can be augmented with additional functionality through using static extensions.

\begin{lstlisting}
using StringTools;
"  Me & You    ".trim().htmlEscape();
\end{lstlisting}

\emph{\tref{String Interpolation}{lf-string-interpolation}:}

Strings declared with a single quotes are able to access variables in the current context.

\begin{lstlisting}
trace('My name is $name and I work in ${job.industry}');
\end{lstlisting}

\emph{\tref{Partial function application}{lf-function-bindings}:} 

Any function can be applied partially, providing the values of some arguments and leaving the rest to be filled in later.

\begin{lstlisting}
var map = new haxe.ds.IntMap();
var setToTwelve = map.set.bind(_, 12);
setToTwelve(1);
setToTwelve(2);
\end{lstlisting}

\emph{\tref{Pattern Matching}{lf-pattern-matching}:} 

Complex structures can be matched against patterns, extracting information from an enum or a structure and defining specific operations for specific value combination.

\begin{lstlisting}
var a = { foo: 12 };
switch (a) {
    case { foo: i }: trace(i);
    default:
}
\end{lstlisting}

\emph{\tref{Properties}{class-field-property}:}

Variable class fields can be designed as properties with custom read and write access, allowing fine grained access control.
\begin{lstlisting}
public var color(get,set);
function get_color() {
    return element.style.backgroundColor;
}
function set_color(c:String) {
    trace('Setting background of element to $c');
    return element.style.backgroundColor = c;
}
\end{lstlisting}

\emph{\tref{Access control}{lf-access-control}:}

The access control language feature uses the Haxe metadata syntax to force or allow access classes or fields.

\emph{\tref{Type Parameters, Constraints and Variance}{type-system-type-parameters}:}

Types can be parametrized with type parameters, allowing typed containers and other complex data structures. Type parameters can also be constrained to certain types and respect variance rules.

\begin{lstlisting}
class Main<A> {
    static function main() {
        new Main<String>("foo");
        new Main(12); // use type inference
    }

    function new(a:A) { }
}
\end{lstlisting}

\section{Conditional Compilation}
\label{lf-condition-compilation}

Haxe allows conditional compilation by using \expr{\#if}, \expr{\#elseif} and \expr{\#else} and checking for \emph{compiler flags}.

\define{Compiler Flag}{define-compiler-flag}{A compiler flag is a configurable value which may influence the compilation process. Such a flag can be set by invoking the command line with \expr{-D key=value} or just \expr{-D key}, in which case the value defaults to \expr{"1"}. The compiler also sets several flags internally to pass information between different compilation steps.}

This example demonstrates usage of conditional compilation:

\haxe{assets/ConditionalCompilation.hx}

Compiling this without any flags will leave only the \expr{trace("ok");} line in the body of the \expr{main} method. The other branches are discarded while parsing the file. These other branches must still contain valid Haxe syntax, but the code is not type-checked.

The conditions after \expr{\#if} and \expr{\#elseif} allow the following expressions:

\begin{itemize}
	\item Any identifier is replaced by the value of the compiler flag by the same name. Note that \expr{-D some-flag} from command line leads to the flags \expr{some-flag} and \expr{some\_flag} to be defined.
	\item The values of \type{String}, \type{Int} and \type{Float} constants are used directly.
		\item The boolean operators \expr{\&\&} (and), \expr{||} (or) and \expr{!} (not) work as expected, however the full expression must be completely contained by parentheses.
	\item The operators \expr{==}, \expr{!=}, \expr{>}, \expr{>=}, \expr{<}, \expr{<=} can be used to compare values.
	\item Parentheses \expr{()} can be used to group expressions as usual.
\end{itemize}

The Haxe parser does not parse \expr{some-flag} as a single token and instead reads it as a subtraction binary operator \expr{some - flag}. In cases like this the underscore version \expr{some_flag} has to be used.

\paragraph{Working with compiler flags}
Compiler flags are available at compile time, the following methods only work in macro context:
\begin{itemize}
	\item To see if a compiler flag is set, use \expr{haxe.macro.Context.defined("any_flag")}. 
	\item To get the value of a compiler flag, use \expr{haxe.macro.Context.definedValue("any_flag")}. 
	\item To get a map of all compiler flags with its value use \expr{haxe.macro.Context.getDefines()}. 
\end{itemize}

\paragraph{Haxelibs}
By default, each used haxelib version is automatically added as flag, e.g. when you add \expr{-lib actuate}, the compiler adds \expr{-D actuate=1.8.7}. To test if a library exists in current context, use \expr{\#if actuate}. To check a specific haxelib version, use the operators, for example \expr{\#if (actuate <= "1.8.7")}

\paragraph{Built-in Compiler Flags}
An exhaustive list of all built-in defines can be obtained by invoking the Haxe Compiler with the \expr{--help-defines} argument. The Haxe Compiler allows multiple \expr{-D} flags per compilation.

\paragraph{Related content}
\begin{itemize}
	\item See also the \tref{Compiler Flags list}{compiler-usage-flags}.
\end{itemize} 


\section{Externs}
\label{lf-externs}

Externs can be used to describe target-specific interaction in a type-safe manner. They are defined like normal classes, except that

\begin{itemize}
	\item the \expr{class} keyword is preceded by the \expr{extern} keyword,
	\item \tref{methods}{class-field-method} have no expressions,
	\item all argument and return types are explicit, and
	\item the default \tref{visibility}{class-field-visibility} is \expr{public} (\expr{private} must be specified explicitly).
\end{itemize}

A common example from the \tref{Haxe Standard Library}{std} is the \type{Math} class, as an excerpt shows:

\begin{lstlisting}
extern class Math
{
	static var PI(default,null) : Float;
	static function floor(v:Float):Int;
}
\end{lstlisting}

We see that externs can define both methods and variables (actually, \expr{PI} is declared as a read-only \tref{property}{class-field-property}). Once this information is available to the compiler, it allows field access accordingly and also knows the types:

\haxe{assets/Extern.hx}

This works because the return type of method \expr{floor} is declared to be \type{Int}.

The Haxe Standard Library comes with many externs for the \target{Flash} and \target{JavaScript} target. They allow accessing the native APIs in a type-safe manner and are instrumental for designing higher-level APIs. There are also externs for many popular native libraries on \tref{haxelib}{haxelib}.

The \target{Flash}, \target{Java} and \target{C\#} targets allow direct inclusion of native libraries from \tref{command line}{compiler-usage}. Target-specific details are explained in the respective sections of \Fullref{target-details}.

Some targets such as \target{Python} or \target{JavaScript} may require generating additional "import" code that loads an \expr{extern} class from a native module. Haxe provides ways to declare such dependencies also described in respective sections \Fullref{target-details}.

\paragraph{Rest arguments and type choices}
\since{3.2.0}

The haxe.extern package provides two types that help mapping native semantics to Haxe:

\begin{description}
	\item[\type{Rest<T>}:] This type can be used as a final function argument to allow passing an arbitrary number of additional call arguments. The type parameter can be used to constrain these arguments to a specific type.
	\item[\type{EitherType<T1,T2>}:] This type allows using either of its parameter types, thus representing a type choice. It can be nested to allow more than two different types.
\end{description}

We demonstrate the usage in this code sample:

\haxe{assets/RestAndEitherType.hx}

\paragraph{Visibility}

Externs support the \expr{private} visibility modifier. However, because the default visibility in an extern class is \expr{public}, \expr{private} needs to be explicitly specified.

Specifying \expr{private} members is helpful when an API intends to allow overriding functions. Also, Haxe cannot prevent subclasses from reusing field names unless if the fields are included in the extern definition. This is important on targets such as JavaScript where reusing a super class’s field name as a new field in a subclass is not supported.

\begin{lstlisting}
extern class ExampleSuperClass
{
	private function new(); // Require subclassing to use.
	// Only allow subclasses access to this overridable function.
	private function overridableFunction():String;
	// This function is implicitly public:
	function doSomething():String;
}
\end{lstlisting}


\section{Static Extension}
\label{lf-static-extension}

\define{Static Extension}{define-static-extension}{A static extension allows pseudo-extending existing types without modifying their source. In Haxe this is achieved by declaring a static method with a first argument of the extending type and then bringing the defining class into context through \expr{using}.}

Static extensions can be a powerful tool which allows augmenting types without actually changing them. The following example demonstrates the usage:

\haxe{assets/StaticExtension.hx}

Clearly, \type{Int} does not natively provide a \expr{triple} method, yet this program compiles and outputs \expr{36} as expected. This is because the call to \expr{12.triple()} is transformed into \expr{IntExtender.triple(12)}. There are three requirements for this:

\begin{enumerate}
	\item Both the literal \expr{12} and the first argument of \expr{triple} are known to be of type \type{Int}.
	\item The class \type{IntExtender} is brought into context through \expr{using Main.IntExtender}.
	\item \type{Int} does not have a \expr{triple} field by itself (if it had, that field would take priority over the static extension).
\end{enumerate}

Static extensions are usually considered syntactic sugar and indeed they are, but it is worth noting that they can have a dramatic effect on code readability: Instead of nested calls in the form of \expr{f1(f2(f3(f4(x))))}, chained calls in the form of \expr{x.f4().f3().f2().f1()} can be used.

Following the rules previously described in \Fullref{type-system-resolution-order}, multiple \expr{using} expressions are checked from bottom to top, with the types within each module as well as the fields within each type being checked from top to bottom. Using a module (as opposed to a specific type of a module, see \Fullref{type-system-modules-and-paths}) as static extension brings all its types into context.


\paragraph{Related content}
\begin{itemize}
	\item \href{http://code.haxe.org/tag/static-extension.html}{Haxe snippets and tutorials about static extensions} in the Haxe Code Cookbook.
\end{itemize} 


\subsection{In the Haxe Standard Library}
\label{lf-static-extension-in-std}

Several classes in the Haxe Standard Library are suitable for static extension usage. The next example shows the usage of \type{StringTools}:

\haxe{assets/StaticExtension2.hx}

While \type{String} does not have a \expr{replace} functionality by itself, the \expr{using StringTools} static extension provides one. As usual, the \target{JavaScript} output nicely shows the transformation:

\lang{js}\begin{lstlisting}
Main.main = function() {
	StringTools.replace("adc","d","b");
}
\end{lstlisting}

The following classes from the Haxe Standard Library are designed to be used as static extensions:

\begin{description}
	\item[\type{StringTools}:] Provides extended functionality on strings, such as replacing or trimming.
	\item[\type{Lambda}:] Provides functional methods on iterables.
	\item[\type{haxe.EnumTools}:] Provides type information functionality on enums and their instances.
	\item[\type{haxe.macro.Tools}:] Provides different extensions for working with macros (see \Fullref{macro-tools}).
\end{description}



\trivia{``using'' using}{Since the \expr{using} keyword was added to the language, it has been common to talk about certain problems with ``using using'' or the effect of ``using using''. This makes for awkward English in many cases, so the author of this manual decided to call the feature by what it actually is: Static extension.}



\section{Pattern Matching}
\label{lf-pattern-matching}
\state{NoContent}

\subsection{Introduction}
\label{lf-pattern-matching-introduction}

Pattern matching is the process of branching depending on a value matching given, possibly deep patterns. In Haxe, all pattern matching is done within a \tref{\expr{switch} expression}{expression-switch} where the individual \expr{case} expressions represent the patterns. Here we will explore the syntax for different patterns using this data structure as running example:

\haxe[firstline=1,lastline=4]{assets/PatternMatching.hx}

Some pattern matcher basics include:

\begin{itemize}
	\item Patterns will always be matched from top to bottom.
	\item The topmost pattern that matches the input value has its expression executed.
	\item A \expr{_} pattern matches anything, so \expr{case _}: is equal to \expr{default:}
\end{itemize}

\paragraph{Related content}
\begin{itemize}
	\item More about the \tref{switch expression}{expression-switch}.
	\item \href{http://code.haxe.org/tag/pattern-matching.html}{Haxe snippets and tutorials about pattern matching} in the Haxe Code Cookbook.
\end{itemize} 

\subsection{Enum matching}
\label{lf-pattern-matching-enums}

Enums can be matched by their constructors in a natural way:

\haxe[firstline=8,lastline=21]{assets/PatternMatching.hx}

The pattern matcher will check each case from top to bottom and pick the first one that matches the input value. The following manual interpretation of each case rule helps understanding the process:

\begin{description}
	\item[\expr{case Leaf(_)}:] matching fails because \expr{myTree} is a \expr{Node}
	\item[\expr{case Node(_, Leaf(_))}:] matching fails because the right sub-tree of \expr{myTree} is not a \expr{Leaf}, but another \expr{Node}
	\item[\expr{case Node(_, Node(Leaf("bar"), _))}:] matching succeeds
	\item[\expr{case _}:] this is not checked here because the previous line matched
\end{description}

\subsection{Variable capture}
\label{lf-pattern-matching-variable-capture}

It is possible to catch any value of a sub-pattern by matching it against an identifier:

\haxe[firstline=24,lastline=30]{assets/PatternMatching.hx}

This would return one of the following:

\begin{itemize}
	\item If \expr{myTree} is a \expr{Leaf}, its name is returned.
	\item If \expr{myTree} is a \expr{Node} whose left sub-tree is a \expr{Leaf}, its name is returned (this will apply here, returning \expr{"foo"}).
	\item Otherwise \expr{"none"} is returned.
\end{itemize}

It is also possible to use = to capture values which are further matched:

\haxe[firstline=32,lastline=36]{assets/PatternMatching.hx}

Here, \expr{leafNode} is bound to \expr{Leaf("foo")} if the input matches that. In all other cases, \expr{myTree} itself is returned: \expr{case x} works similar to \expr{case _} in that it matches anything, but with an identifier name like \expr{x} it also binds the matched value to that variable.

\subsection{Structure matching}
\label{lf-pattern-matching-structure}

It is also possible to match against the fields of anonymous structures and instances:

\haxe[firstline=38,lastline=50]{assets/PatternMatching.hx}

In the second case we bind the matched \expr{name} field to identifier \expr{n} if \expr{rating} matches \expr{"awesome"}. Of course this structure could also be put into the \type{Tree} from the previous example to combine structure and enum matching.

\subsection{Array matching}
\label{lf-pattern-matching-array}

Arrays can be matched on fixed length:

\haxe[firstline=52,lastline=60]{assets/PatternMatching.hx}

This will trace \expr{1} because \expr{myArray[1]} matches \expr{6}, and \expr{myArray[0]} is allowed to be anything.

\subsection{Or patterns}
\label{lf-pattern-matching-or}

The \expr{|} operator can be used anywhere within patterns to describe multiple accepted patterns:

\haxe[firstline=63,lastline=68]{assets/PatternMatching.hx}

If there is a captured variable in an or-pattern, it must appear in both its sub-patterns.

\subsection{Guards}
\label{lf-pattern-matching-guards}

It is also possible to further restrict patterns with the \expr{case ... if(condition):} syntax:

\haxe[firstline=71,lastline=79]{assets/PatternMatching.hx}

The first case has an additional guard condition \expr{if (b > a)}. It will only be selected if that condition holds, otherwise matching continues with the next case.

\subsection{Match on multiple values}
\label{lf-pattern-matching-tuples}

Array syntax can be used to match on multiple values:

\haxe[firstline=82,lastline=87]{assets/PatternMatching.hx}

This is quite similar to usual array matching, but there are differences:

\begin{itemize}
	\item The number of elements is fixed, so patterns of different array length will not be accepted.
	\item It is not possible to capture the switch value in a variable, i.e. \expr{case x} is not allowed (\expr{case _} still is).
\end{itemize}




\subsection{Extractors}
\label{lf-pattern-matching-extractors}
\since{3.1.0}

Extractors allow applying transformations to values being matched. This is often useful when a small operation is required on a matched value before matching can continue:

\haxe{assets/Extractor2.hx}

Here we have to capture the argument value of the \expr{TString} enum constructor in a variable \expr{temp} and use a nested switch on \expr{temp.toLowerCase()}. Obviously, we want matching to succeed if \expr{TString} holds a value of \expr{"foo"} regardless of its casing. This can be simplified with extractors:

\haxe{assets/Extractor.hx}

Extractors are identified by the \expr{extractorExpression => match} expression. The compiler generates code which is similar to the previous example, but the original syntax was greatly simplified. Extractors consist of two parts, which are separated by the \expr{=>} operator:

\begin{enumerate}
\item The left side can be any expression, where all occurrences of underscore \expr{_} are replaced with the currently matched value.
\item The right side is a pattern which is matched against the result of the evaluation of the left side.
\end{enumerate}

Since the right side is a pattern, it can contain another extractor. The following example ``chains'' two extractors:

\haxe{assets/Extractor4.hx}

This traces \expr{12} as a result of the calls to \expr{add(3, 1)}, where \expr{3} is the matched value, and \expr{mul(4, 3)} where \expr{4} is the result of the \expr{add} call. It is worth noting that the \expr{a} on the right side of the second \expr{=>} operator is a \tref{capture variable}{lf-pattern-matching-variable-capture}.

It is currently not possible to use extractors within \tref{or-patterns}{lf-pattern-matching-or}:

\haxe{assets/Extractor5.hx}

However, it is possible to have or-patterns on the right side of an extractor, so the previous example would compile without the parentheses.


\subsection{Exhaustiveness checks}
\label{lf-pattern-matching-exhaustiveness}

The compiler ensures that no possible cases are forgotten:

\begin{lstlisting}
switch(true) {
    case false:
} // Unmatched patterns: true
\end{lstlisting}

The matched type \type{Bool} admits two values \expr{true} and \expr{false}, but only \expr{false} is checked. 

\todo{Figure out wtf our rules are now for when this is checked.}



\subsection{Useless pattern checks}
\label{lf-pattern-matching-unused}

In a similar fashion, the compiler detects patterns which will never match the input value:

\begin{lstlisting}
switch(Leaf("foo")) {
    case Leaf(_)
       | Leaf("foo"): // This pattern is unused
    case Node(l,r):
    case _: // This pattern is unused
}
\end{lstlisting}



\section{String Interpolation}
\label{lf-string-interpolation}

With Haxe 3 it is no longer necessary to manually concatenate parts of a string due to the introduction of \emph{String Interpolation}. Special identifiers, denoted by the dollar sign \expr{\$} within a String enclosed by single-quote \expr{'} characters, are evaluated as if they were concatenated identifiers:

\begin{lstlisting}
var x = 12;
// The value of x is 12
trace('The value of x is $x');
\end{lstlisting}
Furthermore, it is possible to include whole expressions in the string by using \expr{\$$\left\{expr\right\}$}, with \expr{expr} being any valid Haxe expression:

\begin{lstlisting}
var x = 12;
// The sum of 12 and 3 is 15
trace('The sum of $x and 3 is ${x + 3}');
\end{lstlisting}
String interpolation is a compile-time feature and has no impact on the runtime. The above example is equivalent to manual concatenation, which is exactly what the compiler generates:

\begin{lstlisting}
trace("The sum of " + x + " and 3 is " + (x + 3));
\end{lstlisting}
Of course the use of single-quote enclosed strings without any interpolation remains valid, but care has to be taken regarding the \$ character as it triggers interpolation. If an actual dollar-sign should be used in the string, \expr{\$\$} can be used.

\trivia{String Interpolation before Haxe 3}{String Interpolation has been a Haxe feature since version 2.09. Back then, the macro \expr{Std.format} had to be used, being both slower and less comfortable than the new string interpolation syntax.}


\section{Array Comprehension}
\label{lf-array-comprehension}

Array comprehension in Haxe uses existing syntax to allow concise initialization of arrays. It is identified by \expr{for} or \expr{while} constructs:

\haxe{assets/ArrayComprehension.hx}

Variable \expr{a} is initialized to an array holding the numbers 0 to 9. The compiler generates code which adds the value of each loop iteration to the array, so the following code would be equivalent:

\begin{lstlisting}
var a = [];
for (i in 0...10) a.push(i);
\end{lstlisting}

Variable \expr{b} is initialized to an array with the same values, but through a different comprehension style using \expr{while} instead of \expr{for}. Again, the following code would be equivalent:

\begin{lstlisting}
var i = 0;
var b = [];
while(i < 10) b.push(i++);
\end{lstlisting}

The loop expression can be anything, including conditions and nested loops, so the following works as expected:

\haxe{assets/AdvArrayComprehension.hx}

\section{Map Comprehension}
\label{lf-map-comprehension}

Map comprehension in Haxe uses existing syntax to allow concise initialization for \type{Map}. Maps are initialized like arrays, but use the map literal syntax with the \ic{=>} operator. It is identified by \expr{for} or \expr{while} constructs:

\haxe{assets/MapComprehension.hx}

Variable \expr{a} is initialized to an \type{Map} holding keys from 0 to 4 and string values. The compiler generates code which adds the value of each loop iteration to the map, so the following code would be equivalent:

\begin{lstlisting}
var a = new Map();
for (i in 0...5) a.set(i, 'number ${i}');
\end{lstlisting}

Variable \expr{b} is initialized to an \type{Map} with the same keys and values, but through a different comprehension style using \expr{while} instead of \expr{for}. Again, the following code would be equivalent:

\begin{lstlisting}
var i = 0;
var b = new Map();
while(i < 5) b.set(i, 'number ${i++}');
\end{lstlisting}

The loop expression can be anything, including conditions and nested loops, so the following works as expected:

\haxe{assets/AdvMapComprehension.hx}


\section{Iterators}
\label{lf-iterators}

With Haxe it is very easy to define custom iterators and iterable data types. These concepts are represented by the types \type{Iterator<T>} and \type{Iterable<T>} respectively:

\begin{lstlisting}
typedef Iterator<T> = {
	function hasNext() : Bool;
	function next() : T;
}

typedef Iterable<T> = {
	function iterator() : Iterator<T>;
}
\end{lstlisting}

Any \tref{class}{types-class-instance} which \tref{structurally unifies}{type-system-structural-subtyping} with one of these types can be iterated over using a \tref{for-loop}{expression-for}. That is, if the class defines methods \expr{hasNext} and \expr{next} with matching return types it is considered an iterator, if it defines a method \expr{iterator} returning an \type{Iterator<T>} it is considered an iterable type.

\haxe{assets/Iterator.hx}

The type \type{MyStringIterator} in this example qualifies as iterator: It defines a method \expr{hasNext} returning \type{Bool} and a method \expr{next} returning \type{String}, making it compatible with \type{Iterator<String>}. The \expr{main} method instantiates it, then iterates over it.

\haxe{assets/Iterable.hx}

Here we do not setup a full iterator like in the previous example, but instead define that the \type{MyArrayWrap<T>} has a method \expr{iterator}, effectively forwarding the iterator method of the wrapped \type{Array<T>} type. 

\paragraph{Related content}
\begin{itemize}
	\item See the \href{http://api.haxe.org/Iterator.html}{Iterator API documentation}. 
	\item \href{http://code.haxe.org/tag/iterator.html}{Haxe snippets and tutorials about iterators} in the Haxe Code Cookbook.
\end{itemize} 


\section{Function Bindings}
\label{lf-function-bindings}

Haxe 3 allows binding functions with partially applied arguments. Each function type can be considered to have a \expr{bind} field, which can be called with the desired number of arguments in order to create a new function. This is demonstrated here:

\haxe{assets/Bind.hx}
Line 4 binds the function \expr{map.set} to a variable named \expr{f}, and applies \expr{12} as second argument. The underscore \expr{_} is used to denote that this argument is not bound, which is shown by comparing the types of \expr{map.set} and \expr{f}: The bound \type{String} argument is effectively cut from the type, turning a \expr{Int->String->Void} type into \expr{Int->Void}.

A call to \expr{f(1)} then actually invokes \expr{map.set(1, "12")}, the calls to \expr{f(2)} and \expr{f(3)} are analogous. The last line proves that all three indices indeed are mapped to the value \expr{"12"}.

The underscore \expr{_} can be skipped for trailing arguments, so the first argument could be bound through \expr{map.set.bind(1)}, yielding a \expr{String->Void} function that sets a new value for index \expr{1} on invocation.

\paragraph{Optional arguments}

By default, trailing optional arguments are bound to their default values and do not become arguments of the result function. This can be changed by using an explicit underscore \expr{_} instead, in which case the optional argument of the original function becomes a non-optional argument of the result function.
\haxe{assets/BindOptional.hx}

\trivia{Callback}{Prior to Haxe 3, Haxe used to know a \expr{callback}-keyword which could be called with a function argument followed by any number of binding arguments. The name originated from a common usage were a callback-function is created with the this-object being bound.\\
Callback would allow binding of arguments only from left to right as there was no support for the underscore \expr{_}. The choice to use an underscore was controversial and several other suggestions were made, none of which were considered superior. After all, the underscore \expr{_} at least looks like it's saying ``fill value in here'', which nicely describes its semantics.}



\section{Metadata}
\label{lf-metadata}

Several constructs can be attributed with custom metadata:

\begin{itemize}
	\item \expr{class} and \expr{enum} declarations
	\item Class fields
	\item Enum constructors
	\item Expressions
\end{itemize}

These metadata information can be obtained at runtime through the \type{haxe.rtti.Meta} API:

\haxe{assets/Meta.hx}

We can easily identify metadata by the leading \expr{@} character, followed by the metadata name and, optionally, by a number of comma-separated constant arguments enclosed in parentheses.

\begin{itemize}
	\item Class \type{MyClass} has an \expr{author} metadata with a single String argument \expr{"Nicolas"}, as well as a \expr{:keep} metadata without arguments.
	\item The member variable \expr{value} has a \expr{range} metadata with two Int arguments \expr{1} and \expr{8}.
	\item The static method \expr{method} has a \expr{broken} metadata without arguments.
\end{itemize}

The \expr{main} method accesses these metadata values using their API. The output reveals the structure of the obtained data:

\begin{itemize}
	\item There is a field for each metadata, with the field name being the metadata name.
	\item The field values correspond to the metadata arguments. If there are no arguments, the field value is \expr{null}. Otherwise the field value is an array with one element per argument.
\end{itemize}

Allowed values for metadata arguments are:

\begin{itemize}
	\item \tref{Constants}{expression-constants}
	\item \tref{Arrays declarations}{expression-array-declaration} (if all their elements qualify)
	\item \tref{Object declarations}{expression-object-declaration} (if all their field values qualify)
\end{itemize}

\paragraph{Compile-time Metadata}

Metadata starting with \expr{:}, such as \expr{@:keep}, is available at compile time only; it is omitted at runtime. It may be used by macros or by the Haxe compiler itself. Unlike runtime metadata, arguments to compile-time metadata can be any valid expression.

\paragraph{Built-in Compiler Metadata}
An exhaustive list of all defined metadata can be obtained by running \expr{haxe --help-metas} from command line.


\paragraph{Related content}
\begin{itemize}
	\item See also the \tref{Compiler Metadata list}{cr-metadata}.
\end{itemize} 


\section{Access Control}
\label{lf-access-control}

Access control can be used if the basic \tref{visibility}{class-field-visibility} options are not sufficient. It is applicable at \emph{class-level} and at \emph{field-level} and knows two directions:

\begin{description}
	\item[Allowing access:] The target is granted access to the given class or field by using the \expr{:allow(target)} \tref{metadata}{lf-metadata}.
	\item[Forcing access:] A target is forced to allow access to the given class or field by using the \expr{:access(target)} \tref{metadata}{lf-metadata}.
\end{description}

In this context, a \emph{target} can be the \tref{dot-path}{define-type-path} to

\begin{itemize}
	\item a \emph{class field},
	\item a \emph{class} or \emph{abstract} type, or
	\item a \emph{package}.
\end{itemize}

Target does \emph{not} respect imports, so the fully qualified path has to be used.

If it is a class or abstract type, access modification extends to all fields of that type. Likewise, if it is a package, access modification extends to all types of that package and recursively to all fields of these types.

\haxe{assets/ACL.hx}

Here, \expr{MyClass.foo} can be accessed from the \expr{main}-method because \type{MyClass} is annotated with \expr{@:allow(Main)}. This would also work with \expr{@:allow(Main.main)} and both versions could alternatively be annotated to the field \expr{foo} instead of the class \type{MyClass}:

\haxe{assets/ACL2.hx}

If a type cannot be modified to allow this kind of access, the accessing method may force access:

\haxe{assets/ACL3.hx}

The \expr{@:access(MyClass.foo)} annotation effectively subverts the visibility of the \expr{foo} field within the \expr{main}-method. 

\trivia{On the choice of metadata}{The access control language feature uses the Haxe metadata syntax instead of additional language-specific syntax. There are several reasons for that:\\
\begin{itemize}
	\item Additional syntax often adds complexity to the language parsing, and also adds (too) many keywords.
	\item Additional syntax requires additional learning by the language user, whereas metadata syntax is something that is already known.
	\item The metadata syntax is flexible enough to allow extension of this feature.
	\item The metadata can be accessed/generated/modified by Haxe macros.
\end{itemize}
Of course, the main drawback of using metadata syntax is that you get no error report in case you misspell either the metadata key (@:access for instance) or the class/package name. However, with this feature you will get an error when you try to access a private field that you are not allowed to, therefore there is no possibility for silent errors.}

\since{3.1.0}

If access is allowed to an \tref{interface}{types-interfaces}, it extends to all classes implementing that interface:

\haxe{assets/ACL4.hx}

This is also true for access granted to parent classes, in which case it extends to all child classes.

\trivia{Broken feature}{Access extension to child classes and implementing classes was supposed to work in Haxe 3.0 and even documented accordingly. While writing this manual it was found that this part of the access control implementation was simply missing.}


\section{Inline Constructors}
\label{lf-inline-constructor}
\since{3.1.0}

If a constructor is declared to be \tref{inline}{class-field-inline}, the compiler may try to optimize it away in certain situations. There are several requirements for this to work:

\begin{itemize}
	\item The result of the constructor call must be directly assigned to a local variable.
	\item The expression of the constructor field must only contain assignments to its fields.
\end{itemize}

The following example demonstrates constructor inlining:

\haxe{assets/NewInline.hx}

A look at the \target{JavaScript} output reveals the effect:

\lang{js}\begin{lstlisting}
Main.main = function() {
	var pt_x = 1.2;
	var pt_y = 9.3;
};
\end{lstlisting}


\part{Compiler Reference}
\chapter{Compiler Usage}
\label{compiler-usage}

\paragraph{Basic Usage}

The Haxe Compiler is typically invoked from command line with several arguments which have to answer two questions:

\begin{itemize}
	\item What should be compiled?
	\item What should the output be?
\end{itemize}
	
To answer the first question, it is usually sufficient to provide a class path via the \ic{-cp path} argument, along with the main class to be compiled via the \ic{-main dot_path} argument. The Haxe Compiler then resolves the main class file and begins compilation.

The second question usually comes down to providing an argument specifying the desired target. Each Haxe target has a dedicated command line switch, such as \ic{-js file_name} for JavaScript and \ic{-php directory} for PHP. Depending on the nature of the target, the argument value is either a file name (for \ic{-js}, \ic{-swf} and \ic{neko}) or a directory path.

\paragraph{Common arguments}

\emph{Input:}

\begin{description}
	\item[\ic{-cp path}] Adds a class path where \ic{.hx} source files or packages (sub-directories) can be found.
	\item[\ic{-lib library_name}] Adds a \Fullref{haxelib} library. By default the most recent version in the local Haxelib repository is used. To require a specific library version use \ic{-lib library_name:version}. To require a version from git use \ic{-lib library_name:git:https://github.com/user/project.git#commit} where the optional \#commit can be a branch, tag or commit hash.
	\item[\ic{-main dot_path}] Sets the main class.
	\item[\ic{-D <var[=value]>}] Define a \tref{conditional compilation flag}{lf-condition-compilation}.
\end{description}

\emph{Output:}

\begin{description}
	\item[\ic{-js file_name.js}] Generates \tref{JavaScript}{target-javascript} source code in specified file.
	\item[\ic{-as3 directory}] Generates ActionScript 3 source code in specified directory.
	\item[\ic{-swf file_name.swf}] Generates the specified file as \tref{Flash}{target-flash} .swf.
	\item[\ic{-neko file_name.n}] Generates \tref{Neko}{target-neko} binary as specified file.
	\item[\ic{-php directory}] Generates \tref{PHP}{target-php} source code in specified directory. Use \ic{-D php7} for PHP7 source code.
	\item[\ic{-cpp directory}] Generates \tref{C++}{target-cpp} source code in specified directory and compiles it using native C++ compiler.
	\item[\ic{-cs directory}] Generates \tref{C\#}{target-cs} source code in specified directory.
	\item[\ic{-java directory}] Generates \tref{Java}{target-java} source code in specified directory and compiles it using the Java Compiler.
	\item[\ic{-python file_name.py}] Generates \tref{Python}{target-python} source code in the specified file.
	\item[\ic{-lua file_name.lua}] Generates \tref{Lua}{target-python} source code in the specified file.
	\item[\ic{-hl file_name.hl}] Generates \tref{HashLink}{target-hl} byte code in specified file.
	\item[\ic{-cppia file_name.cppia}] Generates the specified script as \tref{cppia}{target-cppia} file.
	\item[\ic{-x <file>}] Shortcut for compiling and executing a neko file.
	\item[\ic{--no-output}] compiles but does not generate any file.
	\item[\ic{--interp}] interpret the program using internal macro system.
\end{description}

\paragraph{Other global arguments}

\begin{description}
	\item[\ic{-xml <file>}] Generate XML types description. Useful for API documentation generation tools like \href{https://github.com/HaxeFoundation/dox}{Dox}.
	\item[\ic{-v}] Turn on verbose mode.
	\item[\ic{-dce <std|full|no>}] Set the \Fullref{cr-dce} mode (default std).
	\item[\ic{-debug}] Add debug information to the compiled code.
	\item[\ic{-resource <file>[@name]}] Add a named resource file.
	\item[\ic{-prompt}] Prompt on error.
	\item[\ic{-cmd}] Run the specified command after a successful compilation.
	\item[\ic{--no-traces}] Don't compile trace calls in the program.
	\item[\ic{--gen-hx-classes}] Generate hx headers for all input classes.
	\item[\ic{--display}] Display code tips to provide \tref{completion information for IDEs and editors}{cr-completion-overview}. 
	\item[\ic{--times}] Measure compilation times.
	\item[\ic{--no-inline}] Disable \Fullref{class-field-inline}.
	\item[\ic{--no-opt}] Disable code optimizations.
	\item[\ic{--remap <package:target>}] Remap a package to another one.
	\item[\ic{--macro}] Call the given \tref{initialization macro}{macro-initialization} before typing anything else.
	\item[\ic{--wait <host:port>}] Wait on the given port for commands to run).
	\item[\ic{--connect <host:port>}] Connect on the given port and run commands there).
	\item[\ic{--cwd <dir>}] Set current working directory.
\end{description}

\paragraph{Target specific arguments}

\begin{description}
	\item[\ic{--php-front <filename>}] Select the name for the php front file.
	\item[\ic{--php-lib <filename>}] Select the name for the php lib folder.
	\item[\ic{--php-prefix <name>}] Prefix all classes with given name.
	\item[\ic{-swf-version <version>}] Change the SWF version.
	\item[\ic{-swf-header <header>}] Define SWF header (width:height:fps:color).
	\item[\ic{-swf-lib <file>}] Add the SWF library to the compiled SWF.
	\item[\ic{-swf-lib-extern <file>}] Use the SWF library for type checking.
	\item[\ic{--flash-strict}] More type strict flash API.
	\item[\ic{-java-lib <file>}] Add an external JAR or class directory library.
	\item[\ic{-net-lib <file>[@std]}] Add an external .NET DLL file.
	\item[\ic{-net-std <file>}] Add a root std .NET DLL search path.
	\item[\ic{-c-arg <arg>}] Pass option \ic{arg} to the native Java/C\# compiler.
\end{description}

\trivia{Run commands after compilation}{Use \ic{-cmd} to run the specified command after a successful compilation. It can be used to run (testing) tools or to directly run the build, e.g. \ic{-cmd java -jar bin/Main.jar} (for Java), \ic{-cmd node main.js} (for Node.js) or \ic{-cmd neko Main.n} (for Neko) etcetera.}

\paragraph{Global compiler configuration macros:} 

In order to include single modules, their paths can be listed directly on command line or in hxml: \ic{haxe ... ModuleName pack.ModuleName}. For more specific includes or excludes, use these \tref{initialization macros}{macro-initialization}:

\begin{description}
	\item[\ic{--macro include(pack:String, recursive=true, ?ignore:Array<String>, ?classPaths:Array<String>, strict=false)}] 
		Includes all modules in package pack in the compilation.  If \ic{recursive} is true, the compiler recursively adds all sub-packages.
	\item[\ic{--macro exclude(pack:String, recursive=true}] 
		Exclude a specific class, enum, or all classes and enums in a package from being generated. Excluded types become \expr{extern}. If \ic{recursive} is true, the compiler recursively excludes all sub-packages.
	\item[\ic{--macro excludeFile(fileName:String)}] 
		Exclude classes and enums listed from given external file (one per line) from being generated.
	\item[\ic{--macro keep(?path:String, ?paths:Array<String>, recursive=true)}] 
		Marks a package, module or sub-type dot path to be kept by DCE. This also extends to the sub-types of resolved modules. If \ic{recursive} is true, the compiler recursively keeps all sub-packages for package paths.
	\item[\ic{--macro includeFile(file:String, position)}] 
		Embed a JavaScript file at compile time. \ic{position} can be either "top", "inline" or "closure".  
\end{description}

The full documentation of these methods can be found in the \href{http://api.haxe.org/haxe/macro/Compiler.html}{\expr{haxe.macro.Compiler}} API documentation.

\paragraph{Help}

\begin{description}
	\item[\ic{haxe -version}] Print the current Haxe compiler version.
	\item[\ic{haxe -help}] Display this list of options.
	\item[\ic{haxe --help-defines}] Print help for all \tref{compiler specific defines}{compiler-usage-flags}.
	\item[\ic{haxe --help-metas}] Print help for all \tref{compiler metadata}{lf-condition-compilation}.
\end{description}

\paragraph{Related content}
\begin{itemize}
	\item \href{http://code.haxe.org/category/compilation/}{Compilation tutorials} in the Haxe Code Cookbook.
\end{itemize}


\section{HXML}
\label{compiler-usage-hxml}

\tref{Compiler arguments}{compiler-usage} can be stored in a .hxml file and can be executed with \ic{haxe <file.hxml>}.
In hxml it is possible to use newlines and comments which makes it easier to maintain Haxe build configurations.
It is possible to supply more arguments after the hxml file, e.g. \ic{haxe build.hxml -debug}.

\emph{Example:}

This example has a configuration which compiles the class file \ic{website.HomePage.hx} to JavaScript into a file called \ic{bin/homepage.js}, which is located in the \ic{src} class path. And uses full dead code elimination.

\begin{lstlisting}
-cp src
-dce full
-js bin/homepage.js
-main website.HomePage
\end{lstlisting}

\paragraph{Multiple build compilations}

Hxml configurations allow multiple compilation using these arguments:

\begin{description}
	\item[\ic{--next}] Separate several Haxe compilations.
	\item[\ic{--each}] Append preceding parameters to all haxe compilations separated by \ic{--next}. This reduces the repeating params.
	
	\begin{description} 
		*Note that neither '--next' nor '--each' are recognized by any Haxe IDEs
	\end{description}
\end{description}

\emph{Example:}

This example has a configuration which compiles three different classes into their own JavaScript files. Each build uses \ic{src} as class path and uses full dead code elimination.

\begin{lstlisting}
-cp src
-dce full

--each

-js bin/homepage.js
-main website.HomePage

--next  

-js bin/gallery.js
-main website.GalleryPage

--next  

-js bin/contact.js
-main website.ContactPage
\end{lstlisting}

\paragraph{Comments inside hxml}

Inside .hxml files use a hash (i.e. \ic{\#}) to comment out the rest of the line. 

\emph{Calling build configurations inside HXML:}

It is possible to create a configuration that looks like this:

\begin{lstlisting}
build-server.hxml  
--next  
build-website.hxml  
--next  
build-game.hxml
\end{lstlisting}


\section{Global Compiler Flags}
\label{compiler-usage-flags}

Starting from Haxe 3.0, you can get the list of supported \tref{compiler flags}{lf-condition-compilation} by running \expr{haxe --help-defines}

\begin{center}
\begin{tabular}{| l | l |}
	\hline
	\multicolumn{2}{|c|}{Global Compiler Flags} \\ \hline
	Flag &  Description \\ \hline
	\expr{absolute-path} &  Print absolute file path in trace output \\
	\expr{advanced-telemetry}  &  Allow the SWF to be measured with Monocle tool \\
	\expr{analyzer}  &  Use static analyzer for optimization (experimental) \\
	\expr{as3} &  Defined when outputing flash9 as3 source code \\
	\expr{check-xml-proxy}  &  Check the used fields of the xml proxy \\
	\expr{core-api}  &  Defined in the core api context \\
	\expr{core-api-serialize}  &  Mark some generated core api classes with the Serializable attribute on C\# \\
	\expr{cppia}  &  Generate experimental cpp instruction assembly \\
	\expr{dce=<mode:std|full|no>}  &  Set the \tref{dead code elimination}{cr-dce} mode (default std) \\
	\expr{dce-debug}  &  Show \tref{dead code elimination}{cr-dce} log \\
	\expr{debug}  &  Activated when compiling with \expr{-debug} \\
	\expr{display}  &  Activated during completion \\
	\expr{dll-export}  &  GenCPP experimental linking \\
	\expr{dll-import}  &  GenCPP experimental linking \\
	\expr{doc-gen}  &  Do not perform any removal/change in order to correctly generate documentation \\
	\expr{dump}  &  Dump the complete typed AST for internal debugging in a dump subdirectory - use \expr{dump=pretty} for Haxe-like formatting \\
	\expr{dump-dependencies}  &  Dump the classes dependencies in a dump subdirectory \\
	\expr{dump-ignore-var-ids}  &  Remove variable IDs from non-pretty dumps (helps with diff) \\
	\expr{erase-generics}  &  Erase generic classes on C\# \\
	\expr{fdb}  &  Enable full flash debug infos for FDB interactive debugging \\
	\expr{file-extension}  &  Output filename extension for cpp source code \\
	\expr{flash-strict}  &  More strict typing for flash target \\
	\expr{flash-use-stage}  &  Keep the SWF library initial stage \\
	\expr{force-lib-check}  &  Force the compiler to check -net-lib and -java-lib added classes (internal) \\
	\expr{force-native-property}  &  Tag all properties with \expr{:nativeProperty} metadata for 3.1 compatibility \\
	\expr{format-warning}  &  Print a warning for each formated string, for 2.x compatibility \\
	\expr{gencommon-debug}  &  GenCommon internal \\
	\expr{haxe-boot}  &  Given the name 'haxe' to the flash boot class instead of a generated name \\
	\expr{haxe-ver}  &  The current Haxe version value. Version 3.1.3 is \ic{3.103}, version 3.2.0 is \ic{3.200}, 3.1.10 is \ic{3.110}. version 3.4.0 is \ic{3.400} etc. This allows doing \ic{#if (haxe_ver >= 3.103)}  \\
	\expr{hxcpp-api-level}  &  Provided to allow compatibility between hxcpp versions \\
	\expr{include-prefix}  &  prepend path to generated include files \\
	\expr{interp}  &  The code is compiled to be run with \expr{--interp} \\
	\expr{java-ver=[version:5-7]}  & Sets the Java version to be targeted \\
	\expr{js-classic}  &  Don't use a function wrapper and strict mode in JS output \\
	\expr{js-es5}  &  Generate JS for ES5-compliant runtimes \\
	\expr{js-unflatten}  & Generate nested objects for packages and types \\
	\expr{keep-old-output}  & Keep old source files in the output directory (for C\#/Java) \\
	\expr{loop-unroll-max-cost}  & Maximum cost (number of expressions * iterations) before loop unrolling is canceled (default 250) \\
	\expr{macro} & Defined when code is compiled in the \tref{macro context}{macro} \\
	\expr{macro-times} & Display per-macro timing when used with \expr{--times} \\
	\expr{net-ver=<version:20-45>}  &  Sets the .NET version to be targeted \\
	\expr{net-target=<name>}  &  Sets the .NET target. Defaults to net. xbox, micro \_(Micro Framework\_, compact \_(Compact Framework)\_ are some valid values  \\
	\expr{neko-source} & Output neko source instead of bytecode \\
	\expr{neko-v1} &  Keep Neko 1.x compatibility \\
	\expr{network-sandbox}  &  Use local network sandbox instead of local file access one \\
	\expr{no-compilation}  &  Disable CPP final compilation \\
	\expr{no-copt}  &  Disable completion optimization \_(for debug purposes)\_ \\
	\expr{no-debug}  &  Remove all debug macros from cpp output \\
	\expr{no-deprecation-warnings} & Do not warn if fields annotated with \expr{@:deprecated} are used \\
	\expr{no-flash-override}  &  Change overrides on some basic classes into HX suffixed methods flash only \\
	\expr{no-opt}  &  Disable optimizations \\
	\expr{no-pattern-matching}  &  Disable \tref{pattern matching}{lf-pattern-matching} \\
	\expr{no-inline}  &  Disable \tref{inlining}{class-field-inline} \\
	\expr{no-root}  &  GenCS internal \\
	\expr{no-macro-cache}  &  Disable macro context caching \\
	\expr{no-simplify}  &  Disable simplification filter \\
	\expr{no-swf-compress}  &  Disable SWF output compression \\
	\expr{no-traces}  &  Disable all \expr{trace} calls \\
	\expr{php-prefix}  &  Compiled with \expr{--php-prefix} \\
	\expr{real-position}  &  Disables haxe source mapping when targetting C\# \\
	\expr{replace-files}  &  GenCommon internal \\
	\expr{scriptable}  &  GenCPP internal \\
	\expr{shallow-expose}  &  Expose types to surrounding scope of Haxe generated closure without writing to window object \\
	\expr{source-map-content}  &  Include the hx sources as part of the JS source map \\
	\expr{swc}  &  Output a SWC instead of a SWF \\
	\expr{swf-compress-level=<level:1-9>}  &  Set the amount of compression for the SWF output \\
	\expr{swf-debug-password=<yourPassword>}  &  Set a password for debugging. The password field is encrypted by using the MD5 algorithm and prevents unauthorised debugging of your swf. Without this flag -D fdb will use no password. \\
	\expr{swf-direct-blit}  &  Use hardware acceleration to blit graphics \\
	\expr{swf-gpu}  &  Use GPU compositing features when drawing graphics \\
	\expr{swf-metadata=<file.xml>}  &  Include contents of \expr{<file.xml>} as metadata in the swf. \\
	\expr{swf-preloader-frame}  &  Insert empty first frame in swf. To be used together with \expr{-D flash-use-stage} and \expr{-swf-lib} \\
	\expr{swf-protected}  &  Compile Haxe private as protected in the SWF instead of public \\
	\expr{swf-script-timeout}  &  Maximum ActionScript processing time before script stuck dialog box displays (in seconds) \\
	\expr{swf-use-doabc}  &  Use DoAbc swf-tag instead of DoAbcDefine \\
	\expr{sys}  &  Defined for all system platforms \\
	\expr{unsafe}  &  Allow unsafe code when targeting C\# \\
	\expr{use-nekoc}  &  Use nekoc compiler instead of internal one \\
	\expr{use-rtti-doc}  &  Allows access to documentation during compilation \\
	\expr{vcproj}  &  GenCPP internal \\
\end{tabular}
\end{center}

\chapter{Compiler Features}
\label{cr-features}
\state{NoContent}

\section{Built-in Compiler Metadata}
\label{cr-metadata}

Starting from Haxe 3.0, you can get the list of defined compiler metadata by running \expr{haxe --help-metas}

\begin{center}
\begin{tabular}{| l | l | l |}
	\hline
	\multicolumn{3}{|c|}{Global metadata} \\ \hline
	Metadata &  Description  &  Platform \\ \hline
	@:abi & Function ABI/calling convention  & cpp \\
	@:abstract &  Sets the underlying class implementation as 'abstract'  &  cs  java \\
	@:access \_(Target path)\_  &   Forces private access to package  type or field,  see \tref{Access Control}{lf-access-control}  &  all \\
	@:allow \_(Target path)\_  &   Allows private access from package  type or field,  see \tref{Access Control}{lf-access-control}  &  all \\
	@:analyzer & Used to configure the static analyzer  &  all \\
	@:annotation  &  Annotation (\expr{@interface}) definitions on \expr{-java-lib} imports will be annotated with this metadata. Has no effect on types compiled by Haxe   &  java \\
	@:arrayAccess  &  Allows \tref{Array access}{types-abstract-array-access} on an abstract  &  all \\
	@:autoBuild \_(Build macro call)\_  &   Extends \expr{@:build} metadata to all extending and implementing classes. See \tref{Macro autobuild}{macro-auto-build}  &  all \\
	@:bind  &  Override Swf class declaration  &  flash \\
	@:bitmap \_(Bitmap file path)\_  &  \_Embeds given bitmap data into the class (must extend \expr{flash.display.BitmapData})   &  flash \\
	@:bridgeProperties  &  Creates native property bridges for all Haxe properties in this class  &  cs \\
	@:build \_(Build macro call)\_  &   Builds a class or enum from a macro. See \tref{Type Building}{macro-type-building}  &  all \\
	@:buildXml  &  Specify xml data to be injected into Build.xml  &  cpp \\
	@:callable  &  Abstract forwards call to its underlying type  &  all \\
	@:classCode  &  Used to inject platform-native code into a class  &  cs  java \\
	@:commutative  &  Declares an abstract operator as commutative  &  all \\
	@:compilerGenerated  &  Marks a field as generated by the compiler. Shouldn't be used by the end user  &  cs  java \\
	@:coreApi &  Identifies this class as a core api class (forces Api check)  &  all \\
	@:coreType  &  Identifies an abstract as \tref{core type}{types-abstract-core-type} so that it requires no implementation  &  all \\
	@:cppFileCode  &  Code to be injected into generated cpp file  &  cpp \\
	@:cppInclude  &  File to be included in generated cpp file  &  cpp \\
	@:cppNamespaceCode  &    &  cpp \\
	@:dce  &  Forces \tref{Dead Code Elimination}{cr-dce} even when not \expr{-dce full} is specified  &  all \\
	@:debug  &  Forces debug information to be generated into the Swf even without \expr{-debug}   &  flash \\
	@:decl   &     &  cpp \\
	@:delegate  &  Automatically added by \expr{-net-lib} on delegates   &  cs \\
	@:depend  &     &  cpp \\
	@:deprecated   &  Automatically added by \expr{-java-lib} on class fields annotated with \expr{@Deprecated} annotation. Has no effect on types compiled by Haxe  &  java \\
	@:event  &  Automatically added by \expr{-net-lib} on events. Has no effect on types compiled by Haxe   &  cs \\
	@:enum  &  Defines finite value sets to abstract definitions. See \tref{enum abstracts}{types-abstract-enum}  &  all \\
	@:expose \_(?Name=Class path)\_  &  Makes the class available on the \expr{window} object or \expr{exports} for node.js. See \tref{exposing Haxe classes for JavaScript}{target-javascript-expose} &  js \\
	@:extern  &  Marks the field as extern so it is not generated  &  all \\
	@:fakeEnum \_(Type name)\_  &  Treat enum as collection of values of the specified type  &  all \\
	@:file(File path)  &  Includes a given binary file into the target Swf and associates it with the class (must extend \expr{flash.utils.ByteArray})  &  flash \\
	@:final  &  Prevents a class from being extended  &  all \\
	@:font \_(TTF path Range String)\_  &  Embeds the given TrueType font into the class (must extend \expr{flash.text.Font})  &  flash \\
	@:forward \_(List of field names)\_  &  \tref{Forwards field access}{types-abstract-forward} to underlying type  &  all \\
	@:from   &  Specifies that the field of the abstract is a cast operation from the type identified in the function. See \tref{Implicit Casts}{types-abstract-implicit-casts}  &  all \\
	@:functionCode  &  Injects native code into the beginning of the function   &  cs cpp \\
	@:functionTailCode  &  Injects native code into the end of the function  &  cpp \\
	@:generic &  Marks a class or class field as \tref{generic}{type-system-generic} so each type parameter combination generates its own type/field  &  all \\
	@:genericBuild  &  Builds instances of a type using the specified macro   &  all \\
	@:getter \_(Class field name)\_  &  Generates a native getter function on the given field   &  flash \\
	@:hack   &  Allows extending classes marked as \expr{@:final}  &  all \\
	@:headerClassCode  &  Code to be injected into the generated class, in the header  &  cpp \\
	@:headerCode   &  Code to be injected into the generated header file  &  cpp \\
	@:headerNamespaceCode  &    &  cpp \\
	@:hxGen  &  Annotates that an extern class was generated by Haxe  &  cs  java \\
	@:ifFeature \_(Feature name)\_  &  Causes a field to be kept by \tref{DCE}{cr-dce} if the given feature is part of the compilation  &  all \\
	@:include &     &  cpp \\
	@:internal  &  Generates the annotated field/class with \expr{internal} access  &  cs  java \\
	@:isVar  &  Forces a physical field to be generated for properties that otherwise would not require one  &  all \\
	@:javaCanonical \_(Output type package,Output type name)\_ &  Used by the Java target to annotate the canonical path of the type  &  java \\
	@:jsRequire  &  Generate javascript module require expression for given extern  &  js \\
	@:keep   &  Causes a field or type to be kept by \tref{DCE}{cr-dce}  &  all \\
	@:keepInit  &  Causes a class to be kept by \tref{DCE}{cr-dce} even if all its field are removed  &  all \\
	@:keepSub &  Extends \expr{@:keep} metadata to all implementing and extending classes  &  all \\
	@:macro  &  \_(deprecated)\_  &  all \\
	@:mergeBlock  &  Merge the annotated block into the current scope  &  all \\
	@:meta   &  Internally used to mark a class field as being the metadata field  &  all \\
	@:multiType \_(Relevant type parameters)\_  &  Specifies that an abstract chooses its this-type from its \expr{@:to} functions  &  all \\
	@:native \_(Output type path)\_  &  Rewrites the path of a class or enum during generation  &  all \\
	@:native  &  Add the \expr{native} keyword to the generated field (JNI)  &  java \\
	@:nativeChildren  &  Annotates that all children from a type should be treated as if it were an extern definition - platform native  &  cs java \\
	@:nativeGen  &  Annotates that a type should be treated as if it were an extern definition - platform native  &  cs  java \\
	@:nativeProperty  &  Use native properties which will execute even with dynamic usage  &  cpp \\
	@:noCompletion  &  Prevents the compiler from suggesting \tref{completion}{cr-completion} on this field  &  all \\
	@:noDebug &  Does not generate debug information into the Swf even if \expr{-debug} is set   &  flash \\
	@:noDoc  &  Prevents a type from being included in documentation generation  &  all \\
	@:noImportGlobal  &  Prevents a static field from being imported with \expr{import Class.*}  &  all \\
	@:noPrivateAccess  &  Disallow private access to anything for the annotated expression  &  all \\
	@:noStack &     &  cpp \\
	@:noUsing &  Prevents a field from being used with \expr{using}  &  all \\
	@:nonVirtual &  Declares function to be non-virtual  &  cpp \\
	@:notNull &  Declares an abstract type as not accepting \tref{\expr{null} values}{types-nullability}  &  all \\
	@:ns  &  Internally used by the Swf generator to handle namespaces   &  flash \\
	@:op \_(The operation)\_  &   Declares an abstract field as being an \tref{operator overload}{types-abstract-operator-overloading}  &  all \\
	@:optional  &  Marks the field of a structure as optional. See \tref{Optional Arguments}{types-nullability-optional-arguments}  &  all \\
	@:overload \_(Function specification)\_  &  Allows the field to be called with different argument types. Function specification cannot be an expression  &  all \\
	@:privateAccess  &  Allow private access to anything for the annotated expression  &  all \\
	@:property  &  Marks a property field to be compiled as a native C\# property   &  cs \\
	@:protected  &  Marks a class field as being protected  &  all \\
	@:public  &  Marks a class field as being public  &  all \\
	@:publicFields  &  Forces all class fields of inheriting classes to be public  &  all \\
	@:pythonImport  &  Generates python import statement for extern classes  &  python \\
	@:readOnly  &  Generates a field with the \expr{readonly} native keyword   &  cs \\
	@:remove  &  Causes an interface to be removed from all implementing classes before generation  &  all \\
	@:require \_(Compiler flag to check)\_  &  Allows access to a field only if the specified \tref{compiler flag}{lf-condition-compilation} is set  &  all \\
	@:rtti   &  Adds runtime type informations. See \tref{RTTI}{cr-rtti}  &  all \\
	@:runtime  &    &  all \\
	@:runtimeValue  &  Marks an abstract as being a runtime value  &  all \\
	@:selfCall  &  Translates method calls into calling object directly  &  js \\
	@:setter \_(Class field name)\_  &  Generates a native setter function on the given field   &  flash \\
	@:sound \_(File path)\_  &  Includes a given \_.wav\_ or \_.mp3\_ file into the target Swf and associates it with the class (must extend \expr{flash.media.Sound})  &  flash \\
	@:sourceFile  &  Source code filename for external class  &  cpp \\
	@:strict  &  Used to declare a native C\# attribute or a native Java metadata. Is type checked  &  cs java \\
	@:struct  &  Marks a class definition as a struct   &  cs \\
	@:structAccess  &  Marks an extern class as using struct access('.') not pointer('->')  &  cpp \\
	@:suppressWarnings  &  Adds a SuppressWarnings annotation for the generated Java class  &  java \\
	@:throws \_(Type as String)\_  &  Adds a \expr{throws} declaration to the generated function   &  java \\
	@:to  &  Specifies that the field of the abstract is a cast operation to the type identified in the function. See \tref{Implicit Casts}{types-abstract-implicit-casts} & all \\
	@:transient  &  Adds the \expr{transient} flag to the class field  &  java \\
	@:unbound  &  Compiler internal to denote unbounded global variable  &  all \\
	@:unifyMinDynamic  &  Allows a collection of types to unify to Dynamic  &  all \\
	@:unreflective  &    &  cpp \\
	@:unsafe  &  Declares a class  or a method with the C\#'s \expr{unsafe} flag   &  cs \\
	@:value  &  Used to store default values for fields and function arguments  &  all \\
	@:void  &  Use Cpp native 'void' return type  &  cpp \\
	@:volatile  &    &  cs  java \\
\end{tabular}
\end{center}

\section{Dead Code Elimination}
\label{cr-dce}

Dead Code Elimination or \emph{DCE} is a compiler feature which removes unused code from the output. After typing, the compiler evaluates the DCE entry-points (usually the main-method) and recursively determines which fields and types are used. Used fields are marked accordingly and unmarked fields are then removed from their classes.

DCE has three modes which are set when invoking the command line:

\begin{description}
	\item[-dce std:] Only classes in the Haxe Standard Library are affected by DCE. This is the default setting on all targets.
	\item[-dce no:] No DCE is performed.
	\item[-dce full:] All classes are affected by DCE.
\end{description}
The DCE-algorithm works well with typed code, but may fail when \tref{dynamic}{types-dynamic} or \tref{reflection}{std-reflection} is involved. This may require explicit marking of fields or classes as being used by attributing the following metadata:

\begin{description}
	\item[\expr{@:keep}:] If used on a class, the class along with all fields is unaffected by DCE. If used on a field, that field is unaffected by DCE.
	\item[\expr{@:keepSub}:] If used on a class, it works like \expr{@:keep} on the annotated class as well as all subclasses.
	\item[\expr{@:keepInit}:] Usually, a class which had all fields removed by DCE (or is empty to begin with) is removed from the output. By using this metadata, empty classes are kept.
\end{description}

If a class needs to be marked with \expr{@:keep} from the command line instead of editing its source code, there is a compiler macro available for doing so: \expr{--macro keep('type dot path')} See the \href{http://api.haxe.org/haxe/macro/Compiler.html#keep}{haxe.macro.Compiler.keep API} for details of this macro. It will mark package, module or sub-type to be kept by DCE and includes them for compilation.
 
The compiler automatically defines the flag \expr{dce} with a value of either \expr{"std"}, \expr{"no"} or \expr{"full"} depending on the active mode. This can be used in \tref{conditional compilation}{lf-condition-compilation}.

\trivia{DCE-rewrite}{DCE was originally implemented in Haxe 2.07. This implementation considered a function to be used when it was explicitly typed. The problem with that was that several features, most importantly interfaces, would cause all class fields to be typed in order to verify type-safety. This effectively subverted DCE completely, prompting the rewrite for Haxe 2.10.}

\trivia{DCE and try.haxe.org}{DCE for the \type{JavaScript} target saw vast improvements when the website \url{http://try.haxe.org} was published. Initial reception of the generated \target{JavaScript} code was mixed, leading to a more fine-grained selection of which code to eliminate.}




\section{Compiler Services}
\label{cr-completion}
\state{NoContent}

\subsection{Overview}
\label{cr-completion-overview}

The rich \tref{type system}{type-system} of the Haxe Compiler makes it difficult for IDEs and editors to provide accurate completion information. Between \tref{type inference}{type-system-type-inference} and \tref{macros}{macro}, it would require a substantial amount of work to replicate the required processing. This is why the Haxe Compiler comes with a built-in completion mode for third-party software to use.

All completion is triggered using the \ic{--display file@position[@mode]} compiler argument. The required arguments are:

\begin{description}
	\item[file:] The file to check for completion. This must be an absolute or relative path to a .hx file. It does not respect any class paths or libraries.
	\item[position:] The byte position (not character position) of where to check for completion in the given file.
	\item[mode:] The completion mode to use (see below).
\end{description}

We will look into the following completion modes in detail:

\begin{description}
	\item[\tref{Field access}{cr-completion-field-access}:] Provides a list of fields that can be accessed on a given type.
	\item[\tref{Call argument}{cr-completion-call-argument}:] Reports the type of the function which is currently being called.
	\item[\tref{Type path}{cr-completion-type-path}:] Lists sub-packages, sub-types and static fields.
	\item[\tref{Usage}{cr-completion-usage}:] Lists all occurrences of a given type, field or variable in all compiled files. (mode: \ic{"usage"})
	\item[\tref{Position}{cr-completion-position}:] Reports the position of where a given type, field or variable is defined. (mode: \ic{"position"})
	\item[\tref{Top-level}{cr-completion-top-level}:] Lists all identifiers which are available at a given position. (mode: \ic{"toplevel"})
\end{description}

Due to Haxe being a very fast compiler, it is often sufficient to rely on the normal compiler invocation for completion. For bigger projects Haxe provides a \tref{server mode}{cr-completion-server} which ensures that only those files are re-compiled that actually changed or had any of their dependencies changes.

\paragraph{General notes on the interface}
\label{cr-completion-interface-notes}

\begin{itemize}
	\item The position-argument can be set to 0 if the file in question contains a pipeline \ic{|} character at the position of interest. This is very useful for demonstration and testing as it allows us to ignore the byte-counting process a real IDE would have to do. The examples in this section are making use of this feature. Note that this only works in places where \ic{|} is not valid syntax otherwise, e.g. after dots (\ic{.|}) and opening parentheses (\ic{(|}).
	\item The output is HTML-escaped so that \ic{\&}, \ic{<} and \ic{>} become \ic{\&amp;amp;}, \ic{\&amp;lt;} and \ic{\&amp;gt;} respectively.
	\item Otherwise any documentation output is preserved which means longer documentation might include new-line and tab-characters as it does in the source files.
	\item When run in completion mode, the compiler does not display errors but instead tries to ignore them or recover from them.  If a critical error occurs while getting completion, the Haxe Compiler prints the error message instead of the completion output. Any non-XML output can be treated as a critical error message.
\end{itemize}

\subsection{Field access completion}
\label{cr-completion-field-access}

Field completion is triggered after a dot \ic{.} character to list the fields that are available on the given type. The compiler parses and types everything up to the point of completion and then outputs the relevant information to stderr:

\begin{lstlisting}
class Main {
  public static function main() {
    trace("Hello".|
  }
}
\end{lstlisting}

If this file is saved to Main.hx, the completion can be invoked using the command \ic{haxe --display Main.hx@0}. The output looks similar to this (we omit several fields for brevity and improve the formatting for readability):

\lang{xml}\begin{lstlisting}
<list>
<i n="length">
  <t>Int</t>
  <d>
    The number of characters in `this` String.
  </d>
</i>
<i n="charAt">
  <t>index : Int -&gt; String</t>
  <d>
    Returns the character at position `index` of `this` String.
    If `index` is negative or exceeds `this.length`, the empty String
    "" is returned.
  </d>
</i>
<i n="charCodeAt">
  <t>index : Int -&gt; Null&lt;Int&gt;</t>
  <d>
    Returns the character code at position `index` of `this` String.
    If `index` is negative or exceeds `this.length`, null is returned.
    To obtain the character code of a single character, "x".code can
    be used instead to inline the character code at compile time.
    Note that this only works on String literals of length 1.
  </d>
</i>
</list>
\end{lstlisting}

The XML structure follows:

\begin{itemize}
	\item The document node \ic{list} encloses several nodes \ic{i}, each representing a field.
	\item The \ic{n} attribute contains the name of the field.
	\item The \ic{t} node contains the type of the field.
	\item the \ic{d} node contains the documentation of the field.
\end{itemize}

\since{3.2.0}

When compiling with \ic{-D display-details}, each field additionally has a \ic{k} attribute which can either be \ic{var} or \ic{method}. This allows distinguishing method fields from variable fields that have a function type.



\subsection{Call argument completion}
\label{cr-completion-call-argument}

Call argument completion is triggered after an opening parenthesis character \ic{(} to show the type of the function that is currently being called. It works for any function call as well as constructor calls:

\begin{lstlisting}
class Main {
  public static function main() {
    trace("Hello".split(|
  }
}
\end{lstlisting}

If this file is saved to Main.hx, the completion can be invoked using the command \ic{haxe --display Main.hx@0}. The output looks like this:

\lang{xml}\begin{lstlisting}
<type>
delimiter : String -&gt; Array&lt;String&gt;
</type>
\end{lstlisting}

IDEs can parse this to recognize that the called function requires one argument named \ic{delimiter} of type \type{String} and returns an \type{Array<String>}.

\trivia{Problems with the output structure}{We acknowledge that the current format requires a bit of manual parsing which can be annoying. In the future we might look into providing a more structured output, especially for functions.}

\subsection{Type path completion}
\label{cr-completion-type-path}

Type path completion can occur in \tref{import declarations}{type-system-import}, \tref{using declarations}{lf-static-extension} or any place a type is referenced. We can identify three different cases:

\paragraph{package completion}

This lists all sub-packages of the haxe package as well as all modules in that package:

\begin{lstlisting}
import haxe.|
\end{lstlisting}

\lang{xml}\begin{lstlisting}
<list>
<i n="CallStack"><t></t><d></d></i>
<i n="Constraints"><t></t><d></d></i>
<i n="DynamicAccess"><t></t><d></d></i>
<i n="EnumFlags"><t></t><d></d></i>
<i n="EnumTools"><t></t><d></d></i>
<i n="Http"><t></t><d></d></i>
<i n="Int32"><t></t><d></d></i>
<i n="Int64"><t></t><d></d></i>
<i n="Json"><t></t><d></d></i>
<i n="Log"><t></t><d></d></i>
<i n="PosInfos"><t></t><d></d></i>
<i n="Resource"><t></t><d></d></i>
<i n="Serializer"><t></t><d></d></i>
<i n="Template"><t></t><d></d></i>
<i n="Timer"><t></t><d></d></i>
<i n="Ucs2"><t></t><d></d></i>
<i n="Unserializer"><t></t><d></d></i>
<i n="Utf8"><t></t><d></d></i>
<i n="crypto"><t></t><d></d></i>
<i n="ds"><t></t><d></d></i>
<i n="extern"><t></t><d></d></i>
<i n="format"><t></t><d></d></i>
<i n="io"><t></t><d></d></i>
<i n="macro"><t></t><d></d></i>
<i n="remoting"><t></t><d></d></i>
<i n="rtti"><t></t><d></d></i>
<i n="unit"><t></t><d></d></i>
<i n="web"><t></t><d></d></i>
<i n="xml"><t></t><d></d></i>
<i n="zip"><t></t><d></d></i>
</list>
\end{lstlisting}


\paragraph{import module completion}

This lists all \tref{sub-types}{type-system-module-sub-types} of the module \type{haxe.Unserializer} as well as all its public static fields (because these can be imported too):

\begin{lstlisting}
import haxe.Unserializer.|
\end{lstlisting}

\lang{xml}\begin{lstlisting}
<list>
<i n="DEFAULT_RESOLVER">
  <t>haxe.TypeResolver</t>
  <d>
    This value can be set to use custom type resolvers.

    A type resolver finds a Class or Enum instance from a given String.
    By default, the haxe Type Api is used.

    A type resolver must provide two methods:

    1. resolveClass(name:String):Class&lt;Dynamic&gt; is called to
      determine a Class from a class name
    2. resolveEnum(name:String):Enum&lt;Dynamic&gt; is called to
      determine an Enum from an enum name

    This value is applied when a new Unserializer instance is created.
    Changing it afterwards has no effect on previously created
    instances.
  </d>
</i>
<i n="run">
  <t>v : String -&gt; Dynamic</t>
  <d>
    Unserializes `v` and returns the according value.

    This is a convenience function for creating a new instance of
    Unserializer with `v` as buffer and calling its unserialize()
    method once.
  </d>
</i>
<i n="TypeResolver"><t></t><d></d></i>
<i n="Unserializer"><t></t><d></d></i>
</list>
\end{lstlisting}


\begin{lstlisting}
using haxe.Unserializer.|
\end{lstlisting}


\paragraph{other module completion}

This lists all \tref{sub-types}{type-system-module-sub-types} of the module \type{haxe.Unserializer}:

\begin{lstlisting}
using haxe.Unserializer.|
\end{lstlisting}

\begin{lstlisting}
class Main {
  static public function main() {
    var x:haxe.Unserializer.|
  }
}
\end{lstlisting}

\lang{xml}\begin{lstlisting}
<list>
<i n="TypeResolver"><t></t><d></d></i>
<i n="Unserializer"><t></t><d></d></i>
</list>
\end{lstlisting}


\subsection{Usage completion}
\label{cr-completion-usage}
\since{3.2.0}

Usage completion is enabled by using the \ic{"usage"} mode argument (see \Fullref{cr-completion-overview}). We demonstrate it here using a local variable. Note that it would work with fields and types the same way:

\begin{lstlisting}
class Main {
  public static function main() {
    var a = 1;
    var b = a + 1;
    trace(a);
    a.|
  }
}
\end{lstlisting}

If this file is saved to Main.hx, the completion can be invoked using the command \ic{haxe --display Main.hx@0@usage}. The output looks like this:

\lang{xml}\begin{lstlisting}
<list>
<pos>main.hx:4: characters 9-10</pos>
<pos>main.hx:5: characters 7-8</pos>
<pos>main.hx:6: characters 1-2</pos>
</list>
\end{lstlisting}



\subsection{Position completion}
\label{cr-completion-position}
\since{3.2.0}

Position completion is enabled by using the \ic{"position"} mode argument (see \Fullref{cr-completion-overview}). We demonstrate it using a field. Note that it would work with local variables and types the same way:

\begin{lstlisting}
class Main {
  static public function main() {
    "foo".split.|
}
\end{lstlisting}

If this file is saved to Main.hx, the completion can be invoked using the command \ic{haxe --display Main.hx@0@position}. The output looks like this:

\lang{xml}\begin{lstlisting}
<list>
<pos>std/string.hx:124: characters 1-54</pos>
</list>
\end{lstlisting}

\trivia{Effects of omitting a target specifier}{In this example the compiler reports the standard String.hx definition which does not actually have an implementation. This happens because we did not specify any target, which is allowed in completion-mode. If the command line arguments included, say, \ic{-neko neko.n}, the reported position would instead be \ic{std/neko/_std/string.hx:84: lines 84-98}.}


\subsection{Top-level completion}
\label{cr-completion-top-level}
\since{3.2.0}

Top-level completion displays all identifiers which the Haxe Compiler knows about at a given compilation position. This is the only completion method for which we need a real position argument in order to demonstrate its effect:

\begin{lstlisting}
class Main {
  static public function main() {
    var a = 1;
  }
}

enum MyEnum {
  MyConstructor1;
  MyConstructor2(s:String);
}
\end{lstlisting}

If this file is saved to Main.hx, the completion can be invoked using the command \ic{haxe --display Main.hx@63@toplevel}. The output looks similar to this (we omit several entries for brevity):

\lang{xml}\begin{lstlisting}
<il>
<i k="local" t="Int">a</i>
<i k="static" t="Void -&gt; Unknown&lt;0&gt;">main</i>
<i k="enum" t="MyEnum">MyConstructor1</i>
<i k="enum" t="s : String -&gt; MyEnum">MyConstructor2</i>
<i k="package">sys</i>
<i k="package">haxe</i>
<i k="type" p="Int">Int</i>
<i k="type" p="Float">Float</i>
<i k="type" p="MyEnum">MyEnum</i>
<i k="type" p="Main">Main</i>
</il>
\end{lstlisting}

The structure of the XML depends on the \ic{k} attribute of each entry. In all cases the node value of the \ic{i} nodes contains the relevant name.

\begin{description}
	\item[\ic{local}, \ic{member}, \ic{static}, \ic{enum}, \ic{global}:] The \ic{t} attribute holds the type of the variable or field.
	\item[\ic{global}, \ic{type}:] The \ic{p} attribute holds the path of the module which contains the type or field.
\end{description}



\subsection{Completion server}
\label{cr-completion-server}

To get the best speed for both compilation and completion, you can use the \ic{--wait} command-line parameter to start a Haxe compilation server. You can also use \ic{-v} to have the server print the log. Here's an example:

\lang{hxml}\begin{lstlisting}
haxe -v --wait 6000
\end{lstlisting}

You can then connect to the Haxe server, send command-line parameters followed by a 0 byte and, then, read the response (either completion result or errors).

Use the \ic{--connect} command-line parameter to have Haxe send its compilation commands to the server instead of executing them directly :

\lang{hxml}\begin{lstlisting}
haxe --connect 6000 myproject.hxml
\end{lstlisting}

Please note that you can use the parameter \ic{--cwd} at the first sent command line to change the Haxe server's current working directory. Usually class paths and other files are relative to your project.

\paragraph{How it works}
The compilation server will cache the following things:

\begin{description}
	\item[parsed files] the files will only get parsed again if they are modified or if there was a parse error
	\item[haxelib calls] the previous results of haxelib calls will be reused (only for completion : they are ignored when doing a compilation)
	\item[typed modules] compilation modules will be cached after a successful compilation and can be reused in later compilation/completions if none of their dependencies have been modified
\end{description}

You can get precise reading of the times spent by the compiler and how using the compilation server affects them by adding \ic{--times} to the command line.

\paragraph{Protocol}
As the following Haxe/Neko example shows, you can simply connect on the server port and send all commands (one per line) ending with a 0 binary char. You can, then, read the results.

Macros and other commands can log events which are not errors. From the command line, we can see the difference between what is printed on \ic{stdout} and what is print on \ic{stderr}. This is not the case in socket mode. In order to differentiate between the two, log messages (not errors) are prefixed with a \ic{\\x01} character and all newline-characters in the message are replaced by the same \ic{\\x01} character.

Warnings and other messages can also be considered errors, but are not fatal ones. If a fatal error occurred, it will send a single \ic{\\x02} message-line.

Here's some code that will treat connection to the server and handle the protocol details:

\haxe{assets/CompletionServer.hx}

\paragraph{Effect on macros}
The compilation server can have some side effects on \tref{macro execution}{macro}.



\section{Resources}
\label{cr-resources}
\flag{fold}{true}

Haxe provides a simple resource embedding system that can be used for embedding files directly into the compiled application.

While it may be not optimal to embed large assets such as images or music in the application file, it comes in very handy for embedding smaller resources like configuration or XML data.

\subsection{Embedding resources}
\label{cr-resources-embed}

External files are embedded using the \emph{-resource} compiler argument:

\lang{hxml}\begin{lstlisting}
-resource hello_message.txt@welcome
\end{lstlisting}

The string after the \emph{@} symbol is the \emph{resource identifier} which is used in the code for retrieving the resource. If it is omitted (together with the \emph{@} symbol) then the file name will become the resource identifier.

\subsection{Retrieving text resources}
\label{cr-resources-getString}

To retrieve the content of an embedded resource we use the static method \emph{getString} of \type{haxe.Resource}, passing a \emph{resource identifier} to it:

\haxe{assets/ResourceGetString.hx}

The code above will display the content of the \emph{hello_message.txt} file that we included earlier using \emph{welcome} as the identifier.

\subsection{Retrieving binary resources}
\label{cr-resources-getBytes}

While it's not recommended to embed large binary files in the application, it still may be useful to embed binary data. The binary representation of an embedded resource can be accessed using the static method \emph{getBytes} of \type{haxe.Resource}:

\haxe{assets/ResourceGetBytes.hx}

The return type of \emph{getBytes} method is \type{haxe.io.Bytes}, which is an object providing access to individual bytes of the data.

\subsection{Implementation details}
\label{cr-resources-impl}

Haxe uses the target platform's native resource embedding if there is one, otherwise it provides its own implementation.

\begin{itemize}
\item \emph{Flash} resources are embedded as ByteArray definitions
\item \emph{C\#} resources are included in the compiled assembly
\item \emph{Java} resources are packed in the resulting JAR file
\item \emph{C++} resources are stored in global byte array constants.
\item \emph{JavaScript} resources are serialized in Haxe serialization format and stored in a static field of \type{haxe.Resource} class.
\item \emph{Neko} resources are stored as strings in a static field of \type{haxe.Resource} class.
\end{itemize}



\section{Runtime Type Information}
\label{cr-rtti}

The Haxe compiler generates runtime type information (RTTI) for classes that are annotated or extend classes that are annotated with the \expr{:rtti} metadata. This information is stored as a XML string in a static field \expr{__rtti} and can be processed through \type{haxe.rtti.XmlParser}. The resulting structure is described in \Fullref{cr-rtti-structure}.

\since{3.2.0}

The type \type{haxe.rtti.Rtti} has been introduced in order to simplify working with RTTI. Retrieving this information is now very easy:

\haxe{assets/RttiUsage.hx}

\subsection{RTTI structure}
\label{cr-rtti-structure}

\paragraph{General type information}

\begin{description}
	\item[path:] The \tref{type path}{define-type-path} of the type.
	\item[module:] The type path of the \tref{module}{define-module} containing the type.
	\item[file:] The full slash path of the .hx file containing the type. This might be \expr{null} in case there is no such file, e.g. if the type is defined through a \tref{macro}{macro}.
	\item[params:] An array of strings representing the names of the \tref{type parameters}{type-system-type-parameters} the type has. As of Haxe 3.2.0, this does not include the \tref{constraints}{type-system-type-parameter-constraints}.
	\item[doc:] The documentation of the type. This information is only available if the \tref{compiler flag}{define-compiler-flag} \expr{-D use_rtti_doc} was in place. Otherwise, or if the type has no documentation, the value is \expr{null}.
	\item[isPrivate:] Whether or not the type is \tref{private}{define-private-type}.
	\item[platforms:] A list of strings representing the targets where the type is available.
	\item[meta:] The meta data the type was annotated with.
\end{description}
	
\paragraph{Class type information}
\label{cr-rtti-class-type-information}

\begin{description}
	\item[isExtern:] Whether or not the class is \tref{extern}{lf-externs}.
	\item[isInterface:] Whether or not the class is actually an \tref{interface}{types-interfaces}.
	\item[superClass:] The class' parent class defined by its type path and list of type parameters.
	\item[interfaces:] The list of interfaces defined by their type path and list of type parameters.
	\item[fields:] The list of member \tref{class fields}{class-field}, described in \Fullref{cr-rtti-class-field-information}.
	\item[statics:] The list of static class fields, described in \Fullref{cr-rtti-class-field-information}.
	\item[tdynamic:] The type which is \tref{dynamically implemented}{types-dynamic-implemented} by the class or \expr{null} if no such type exists.
\end{description}

\paragraph{Enum type information}

\begin{description}
	\item[isExtern:] Whether or not the enum is \tref{extern}{lf-externs}.
	\item[constructors:] The list of enum constructors.
\end{description}

\paragraph{Abstract type information}

\begin{description}
	\item[to:] An array containing the defined \tref{implicit to casts}{types-abstract-implicit-casts}.
	\item[from:] An array containing the defined \tref{implicit from casts}{types-abstract-implicit-casts}.
	\item[impl:] The \tref{class type information}{cr-rtti-class-type-information} of the implementing class.
	\item[athis:] The \tref{underlying type}{define-underlying-type} of the abstract.
\end{description}
	
	
\paragraph{Class field information}
\label{cr-rtti-class-field-information}

\begin{description}
	\item[name:] The name of the field.
	\item[type:] The type of the field.
	\item[isPublic:] Whether or not the field is \tref{public}{class-field-visibility}.
	\item[isOverride:] Whether or not the field \tref{overrides}{class-field-override} another field.
	\item[doc:] The documentation of the field. This information is only available if the \tref{compiler flag}{define-compiler-flag} \expr{-D use_rtti_doc} was in place. Otherwise, or if the field has no documentation, the value is \expr{null}.
	\item[get:] The \tref{read access behavior}{define-read-access} of the field.
	\item[set:] The \tref{write access behavior}{define-write-access} of the field.
	\item[params:] An array of strings representing the names of the \tref{type parameters}{type-system-type-parameters} the field has. As of Haxe 3.2.0, this does not include the \tref{constraints}{type-system-type-parameter-constraints}.
	\item[platforms:] A list of strings representing the targets where the field is available.
	\item[meta:] The meta data the field was annotated with.
	\item[line:] The line number where the field is defined. This information is only available if the field has an expression. Otherwise the value is \expr{null}.
	\item[overloads:] The list of available overloads for the fields or \expr{null} if no overloads exists.
\end{description}

\paragraph{Enum constructor information}
\label{cr-rtti-enum-constructor-information}

\begin{description}
	\item[name:] The name of the constructor.
	\item[args:] The list of arguments the constructor has or \expr{null} if no arguments are available.
	\item[doc:] The documentation of the constructor. This information is only available if the \tref{compiler flag}{define-compiler-flag} \expr{-D use_rtti_doc} was in place. Otherwise, or if the constructor has no documentation, the value is \expr{null}.
	\item[platforms:] A list of strings representing the targets where the constructor is available.
	\item[meta:] The meta data the constructor was annotated with.
\end{description}



\section{Static Analyzer}
\label{cr-static-analyzer}
\since{3.3.0}

Haxe 3.3.0 introduces a new static analyzer for code optimizations. It is enabled by using the \ic{-D analyzer-optimize} \tref{compiler flag}{compiler-usage-flags} and consists of multiple \tref{modules}{cr-static-analyzer-modules} which can be configured globally with a \tref{compiler flag}{compiler-usage-flags} as well as at type-level and field-level with a \tref{compiler metadata}{cr-metadata}:

\paragraph{Global configuration}
\label{cr-static-analyzer-global-configuration}

To globally enable an analyzer module \ic{-D analyzer-module} is used. To globally disable a module \ic{-D analyzer-no-module} is used. In both cases ``module'' represents the name of the module to be disabled or enabled:

\lang{hxml}\begin{lstlisting}
# Global enable from command line
haxe -D analyzer-module
# Global disable from command line
haxe -D analyzer-no-module
\end{lstlisting}

\paragraph{Local configuration}
\label{cr-static-analyzer-local-configuration}
To enable an analyzer module for a given type or field \ic{@:analyzer(module)} is used. To disable a module \ic{@:analyzer(no_module)} is used. In both cases ``module'' represents the name of the module to be disabled or enabled:

\begin{lstlisting}
@:analyzer(module)
class C {
	@:analyzer(module) function f() { } // Field-level enable
	@:analyzer(no_module) function f() { } // Field-level disable
}
@:analyzer(no_module)
class D { } // Type-level disable
\end{lstlisting}

\paragraph{Modules}
\label{cr-static-analyzer-modules}

The static analyzer currently comes with the following modules. Unless noted otherwise they are enabled if \ic{-D analyzer} is used.

\begin{description}
\item[\ic{const_propagation}]: Implements sparse conditional constant propagation to promote values that are known at compile-time to usage places. Also detects dead branches.
\item[\ic{copy_propagation}]: Detects local variables that alias other local variables and replaces them accordingly.
\item[\ic{local_dce}]: Detects and removes unused local variables.
\item[\ic{fusion}]: Moves variable expressions to its usage in case of single-occurrence. By default, only compiler-generated variables are handled. This can be changed by using the compiler flag `\ic{-D analyzer-user-var-fusion} or the metadata \ic{@:analyzer(user_var_fusion)}.
\item[\ic{purity_inference}]: Infers if fields are ``pure'', i.e. do not have any side-effects. This can improve the effect of the fusion module.
\end{description}

\chapter{Macros}
\label{macro}

Macros are without a doubt the most advanced feature in Haxe. They are often perceived as dark magic that only a select few are capable of mastering, yet there is nothing magical (and certainly nothing dark) about them.

\define{Abstract Syntax Tree (AST)}{define-ast}{The AST is the result of \emph{parsing} Haxe code into a typed structure. This structure is exposed to macros through the types defined in the file \expr{haxe/macro/Expr.hx} of the Haxe Standard Library.}

\input{assets/tikz/macro-compilation-role.tex}

A basic macro is a \emph{syntax-transformation}. It receives zero or more \tref{expressions}{expression} and also returns an expression. If a macro is called, it effectively inserts code at the place it was called from. In that respect, it could be compared to a preprocessor like \expr{\#define} in C++, but a Haxe macro is not a textual replacement tool.

We can identify different kinds of macros, which are run at specific compilation stages:

\begin{description}
	\item[Initialization Macros:] These are provided by command line using the \ic{--macro} compiler parameter. They are executed after the compiler arguments were processed and the \emph{typer context} has been created, but before any typing was done (see \Fullref{macro-initialization}).
	\item[Build Macros:] These are defined for classes, enums and abstracts through the \expr{@:build} or \expr{@:autoBuild} \tref{metadata}{lf-metadata}. They are executed per-type, after the type has been set up (including its relation to other types, such as inheritance for classes) but before its fields are typed (see \Fullref{macro-type-building}).
	\item[Expression Macros:] These are normal functions which are executed as soon as they are typed.
\end{description}
	
\paragraph{Related content}
\begin{itemize}
	\item See the \href{http://api.haxe.org/haxe/macro}{macro API documentation} for details about its tools, classes an methods. 
	\item See the \href{http://code.haxe.org/category/macros/}{macro snippets and tutorials} section in the Haxe Code Cookbook.
\end{itemize}

\section{Macro Context}
\label{macro-context}

\define{Macro Context}{define-macro-context}{The macro context is the environment in which the macro is executed. Depending on the macro type, it can be considered to be a class being built or a function being typed. Contextual information can be obtained through the \ic{haxe.macro.Context} API.}

Haxe macros have access to different contextual information depending on the macro type. Other than querying such information, the context also allows some modifications such as defining a new type or registering certain callbacks. It is important to understand that not all information is available for all macro kinds, as the following examples demonstrate:

\begin{itemize}
	\item Initialization macros will find that the \expr{Context.getLocal*()} methods return \expr{null}. There is no local type or method in the context of an initialization macro.
	\item Only build macros get a proper return value from \expr{Context.getBuildFields()}. There are no fields being built for the other macro kinds.
	\item Build macros have a local type (if incomplete), but no local method, so \expr{Context.getLocalMethod()} returns \expr{null}.
\end{itemize}

The context API is complemented by the \expr{haxe.macro.Compiler} API detailed in \Fullref{macro-initialization}. While this API is available to all macro kinds, care has to be taken for any modification outside of initialization macros. This stems from the natural limitation of undefined \tref{build order}{macro-limitations-build-order}, which could cause e.g. a flag definition through \expr{Compiler.define()} to take effect before or after a \tref{conditional compilation}{lf-condition-compilation} check against that flag.

\paragraph{Related content}
\begin{itemize}
	\item See the \href{http://api.haxe.org/haxe/macro/Context.html}{macro Context API documentation}. 
	\item See the \href{http://code.haxe.org/category/macros/}{macro snippets and tutorials} section in the Haxe Code Cookbook.
\end{itemize}

\section{Arguments}
\label{macro-arguments}

Most of the time, arguments to macros are expressions represented as an instance of enum \type{Expr}. As such, they are parsed but not typed, meaning they can be anything conforming to Haxe's syntax rules. The macro can then inspect their structure, or (try to) get their type using \expr{haxe.macro.Context.typeof()}.

It is important to understand that arguments to macros are not guaranteed to be evaluated, so any intended side-effect is not guaranteed to occur. On the other hand, it is also important to understand that an argument expression may be duplicated by a macro and used multiple times in the returned expression:

\haxe{assets/MacroArguments.hx}

The macro \expr{add} is called with \expr{x++} as argument and thus returns \expr{x++ + x++} using \tref{expression reification}{macro-reification-expression}, causing \expr{x} to be incremented twice.

\subsection{ExprOf}
\label{macro-ExprOf}

Since \type{Expr} is compatible with any possible input, Haxe provides the type \type{haxe.macro.ExprOf<T>}. For the most part, this type is identical to \type{Expr}, but it allows constraining the type of accepted expressions. This is useful when combining macros with \tref{static extensions}{lf-static-extension}:

\haxe{assets/ExprOf.hx}

The two direct calls to \expr{identity} are accepted, even though the argument is declared as \expr{ExprOf<String>}. It might come as a surprise that the \type{Int} \expr{1} is accepted, but it is a logical consequence of what was explained about \tref{macro arguments}{macro-arguments}: The argument expressions are never typed, so it is not possible for the compiler to check their compatibility by \tref{unifying}{type-system-unification}.

This is different for the next two lines which are using static extensions (note the \expr{using Main}): For these it is mandatory to type the left side (\expr{"foo"} and \expr{1}) first in order to make sense of the \expr{identity} field access. This makes it possible to check the types against the argument types, which causes \expr{1.identity()} to not consider \expr{Main.identity()} as a suitable field.

\subsection{Constant Expressions}
\label{macro-constant-arguments}

A macro can be declared to expect \tref{constant}{expression-constants} arguments:

\haxe{assets/MacroArgumentsConst.hx}

With these it is not necessary to detour over expressions as the compiler can use the provided constants directly.

\subsection{Rest Argument}
\label{macro-rest-argument}

If the final argument of a macro is of type \type{Array<Expr>}, the macro accepts an arbitrary number of extra arguments which are available from that array:

\haxe{assets/MacroArgumentsRest.hx}




\section{Reification}
\label{macro-reification}

The Haxe Compiler allows \emph{reification} of expressions, types and classes to simplify working with macros. The syntax for reification is \expr{macro expr}, where \expr{expr} is any valid Haxe expression.

\subsection{Expression Reification}
\label{macro-reification-expression}

Expression reification is used to create instances of \type{haxe.macro.Expr} in a convenient way. The Haxe Compiler accepts the usual Haxe syntax and translates it to an expression object. It supports several escaping mechanisms, all of which are triggered by the \expr{\$} character:

\begin{description}
	\item[\expr{\$\{\}} and \expr{\$e\{\}}:] \type{Expr -> Expr} This can be used to compose expressions. The expression within the delimiting \expr{\{ \}} is executed, with its value being used in place.
	\item[\expr{\$a\{\}}:] \type{Array<Expr> -> Array<Expr>} or \type{Array<Expr> -> Expr} If used in a place where an \type{Array<Expr>} is expected (e.g. call arguments, block elements), \expr{\$a\{\}} treats its value as that array. Otherwise it generates an array declaration.
	\item[\expr{\$b\{\}}:] \type{Array<Expr> -> Expr} Generates a block expression from the given expression array.
	\item[\expr{\$i\{\}}:] \type{String -> Expr} Generates an identifier from the given string.
	\item[\expr{\$p\{\}}:] \type{Array<String> -> Expr} Generates a field expression from the given string array.
	\item[\expr{\$v\{\}}:] \type{Dynamic -> Expr} Generates an expression depending on the type of its argument. This is only guaranteed to work for \tref{basic types}{types-basic-types} and \tref{enum instances}{types-enum-instance}.
\end{description}

Additionally the \tref{metadata}{lf-metadata} \expr{@:pos(p)} can be used to map the position of the annotated expression to \expr{p} instead of the place it is reified at.

This kind of reification only works in places where the internal structure expects an expression. This disallows \expr{object.\$\{fieldName\}}, but \expr{object.\$fieldName} works. This is true for all places where the internal structure expects a string:

\begin{itemize}
	\item field access \expr{object.\$name}
	\item variable name \expr{var \$name = 1;}
\end{itemize}
\since{3.1.0}
\begin{itemize}
	\item field name \expr{\{ \$name: 1\} }
	\item function name \expr{function \$name() \{ \}}
	\item catch variable name \expr{try e() catch(\$name:Dynamic) \{ \}}
\end{itemize}

Furthermore, a \expr{new} expression can be reified by providing \href{http://api.haxe.org/haxe/macro/TypePath.html}{haxe.macro.TypePath} argument: \expr{new \$typePath()}

\subsection{Type Reification}
\label{macro-reification-type}

Type reification is used to create instances of \type{haxe.macro.Expr.ComplexType} in a convenient way. It is identified by a \expr{macro : Type}, where \expr{Type} can be any valid type path expression. This is similar to explicit type hints in normal code, e.g. for variables in the form of \expr{var x:Type}.

Each constructor of \type{ComplexType} has a distinct syntax:

\begin{description}
	\item[\expr{TPath}:] \expr{macro : pack.Type}
	\item[\expr{TFunction}:] \expr{macro : Arg1 -> Arg2 -> Return}
	\item[\expr{TAnonymous}:] \expr{macro : \{ field: Type \}}
	\item[\expr{TParent}:] \expr{macro : (Type)}
	\item[\expr{TExtend}:] \expr{macro : \{> Type, field: Type \}}
	\item[\expr{TOptional}:] \expr{macro : ?Type}
\end{description}

\subsection{Class Reification}
\label{macro-reification-class}

It is also possible to use reification to obtain an instance of \type{haxe.macro.Expr.TypeDefinition}. This is indicated by the \expr{macro class} syntax as shown here:

\haxe{assets/ClassReification.hx}

The generated \type{TypeDefinition} instance is typically passed to \expr{haxe.macro.Context.defineType} in order to add a new type to the calling context (not the macro context itself).

This kind of reification can also be useful to obtain instances of \expr{haxe.macro.Expr.Field}, which are available from the \expr{fields} array of the generated \type{TypeDefinition}. 

\section{Tools}
\label{macro-tools}

The Haxe Standard Library comes with a set of tool-classes to simplify working with macros. These classes work best as \tref{static extensions}{lf-static-extension} and can be brought into context either individually or as a whole through \expr{using haxe.macro.Tools}. These classes are:

\begin{description}
	\item[\type{ComplexTypeTools}:] Allows printing \type{ComplexType} instances in a human-readable way. Also allows determining the \type{Type} corresponding to a \type{ComplexType}.
	\item[\type{ExprTools}:] Allows printing \type{Expr} instances in a human-readable way. Also allows iterating and mapping expressions.
	\item[\type{MacroStringTools}:] Offers useful operations on strings and string expressions in macro context.
	\item[\type{TypeTools}:] Allows printing \type{Type} instances in a human-readable way. Also offers several useful operations on types, such as \tref{unifying}{type-system-unification} them or getting their corresponding \type{ComplexType}.
\end{description}

Furthermore the \type{haxe.macro.Printer} class has public methods for printing various types as a human-readable format. This can be helpful when debugging macros.

\trivia{The tinkerbell library and why Tools.hx works}{We learned about static extensions that using a \emph{module} implies that all its types are brought into static extension context. As it turns out, such a type can also be a \tref{typedef}{type-system-typedef} to another type. The compiler then considers this type part of the module, and extends static extension accordingly.\\
This ``trick'' was first used in Juraj Kirchheim's \emph{tinkerbell}\footnote{https://github.com/back2dos/tinkerbell} library for exactly the same purpose. Tinkerbell provided many useful macro tools long before they made it into the Haxe Compiler and Haxe Standard Library. It remains the primary library for additional macro tools and offers other useful functionality as well.} 



\section{Type Building}
\label{macro-type-building}

Type-building macros are different from expression macros in several ways:

\begin{itemize}
	\item They do not return expressions, but an array of class fields. Their return type must be set explicitly to \type{Array<haxe.macro.Expr.Field>}.
	\item Their \tref{context}{macro-context} has no local method and no local variables.
	\item Their context does have build fields, available from \expr{haxe.macro.Context.getBuildFields()}.
	\item They are not called directly, but are argument to a \expr{@:build} or \expr{@:autoBuild} \tref{metadata}{lf-metadata} on a \tref{class}{types-class-instance} or \tref{enum}{types-enum-instance} declaration.
\end{itemize}

The following example demonstrates type building. Note that it is split up into two files for a reason: If a module contains a \expr{macro} function, it has to be typed into macro context as well. This is often a problem for type-building macros because the type to be built could only be loaded in its incomplete state, before the building macro has run. We recommend to always define type-building macros in their own module.

\haxe{assets/TypeBuildingMacro.hx}
\haxe{assets/TypeBuilding.hx}

The \expr{build} method of \type{TypeBuildingMacro} performs three steps:

\begin{enumerate}
	\item It obtains the build fields using \expr{Context.getBuildFields()}.
	\item It declares a new \type{haxe.macro.expr.Field} field using the \expr{funcName} macro argument as field name. This field is a \type{String} \tref{variable}{class-field-variable} with a default value \expr{"my default"} (from the \expr{kind} field) and is public and static (from the \expr{access} field).
	\item It adds the new field to the build field array and returns it.
\end{enumerate}

This macro is argument to the \expr{@:build} metadata on the \type{Main} class. As soon as this type is required, the compiler does the following:

\begin{enumerate}
	\item It parses the module file, including the class fields.
	\item It sets up the type, including its relation to other types through \tref{inheritance}{types-class-inheritance} and \tref{interfaces}{types-interfaces}.
	\item It executes the type-building macro according to the \expr{@:build} metadata.
	\item It continues typing the class normally with the fields returned by the type-building macro.
\end{enumerate}

This allows adding and modifying class fields at will in a type-building macro. In our example, the macro is called with a \expr{"myFunc"} argument, making \expr{Main.myFunc} a valid field access.

If a type-building macro should not modify anything, the macro can return \expr{null}. This indicates to the compiler that no changes are intended and is preferable to returning \expr{Context.getBuildFields()}.



\subsection{Enum building}
\label{macro-enum-building}

Building \tref{enums}{types-enum-instance} is analogous to building classes with a simple mapping:

\begin{itemize}
	\item Enum constructors without arguments are variable fields \expr{FVar}.
	\item Enum constructors with arguments are method fields \expr{FFun}.
\end{itemize}

\todo{Check if we can build GADTs this way.}

\haxe{assets/EnumBuildingMacro.hx}
\haxe{assets/EnumBuilding.hx}

Because enum \type{E} is annotated with a \expr{:build} metadata, the called macro builds two constructors \expr{A} and \expr{B} ``into'' it. The former is added with the kind being \expr{FVar(null, null)}, meaning it is a constructor without argument. For the latter, we use \tref{reification}{macro-reification-expression} to obtain an instance of \type{haxe.macro.Expr.Function} with a singular \type{Int} argument.

The \expr{main} method proves the structure of our generated enum by \tref{matching}{lf-pattern-matching} it. We can see that the generated type is equivalent to this:

\begin{lstlisting}
enum E {
	A;
	B(value:Int);
}
\end{lstlisting}


\subsection{@:autoBuild}
\label{macro-auto-build}

If a class has the \expr{:autoBuild} metadata, the compiler generates \expr{:build} metadata on all extending classes. If an interface has the \expr{:autoBuild} metadata, the compiler generates \expr{:build} metadata on all implementing classes and all extending interfaces. Note that \expr{:autoBuild} does not imply \expr{:build} on the class/interface itself.

\haxe{assets/AutoBuildingMacro.hx}
\haxe{assets/AutoBuilding.hx}

This outputs during compilation:

\lang{none}\begin{lstlisting}
AutoBuildingMacro.hx:6:
  fromInterface: TInst(I2,[])
AutoBuildingMacro.hx:6:
  fromInterface: TInst(Main,[])
AutoBuildingMacro.hx:11:
  fromBaseClass: TInst(Main,[])
\end{lstlisting}

It is important to keep in mind that the order of these macro executions is undefined, which is detailed in \Fullref{macro-limitations-build-order}.

\paragraph{Related content}
\begin{itemize}
	\item \href{http://code.haxe.org/tag/build-macro.html}{Haxe snippets and tutorials about build macros} in the Haxe Code Cookbook.
\end{itemize}

\subsection{@:genericBuild}
\label{macro-generic-build}
\since{3.1.0}

Normal \tref{build-macros}{macro-type-building} are run per-type and are already very powerful. In some cases it is useful to run a build macro per type \emph{usage} instead, i.e. whenever it actually appears in the code. Among other things this allows accessing the concrete type parameters in the macro.

\expr{@:genericBuild} is used just like \expr{@:build} by adding it to a type with the argument being a macro call:

\haxe{assets/GenericBuildMacro1.hx}

\haxe{assets/GenericBuild1.hx}

When running this example the compiler outputs \ic{TAbstract(Int,[])} and \ic{TInst(String,[])}, indicating that it is indeed aware of the concrete type parameters of \type{MyType}. The macro logic could use this information to generate a custom type (using \expr{haxe.macro.Context.defineType}) or refer to an existing one. For brevity we return \expr{null} here which asks the compiler to \tref{infer}{type-system-type-inference} the type.

In Haxe 3.1 the return type of a \expr{@:genericBuild} macro has to be a \type{haxe.macro.Type}. Haxe 3.2 allows (and prefers) returning a \type{haxe.macro.ComplexType} instead, which is the syntactic representation of a type. This is easier to work with in many cases because types can simply be referenced by their paths.

\paragraph{Const type parameter}

Haxe allows passing \tref{constant expression}{expression-constants} as a type parameter if the type parameter name is \expr{Const}. This can be utilized in the context of \expr{@:genericBuild} macros to pass information from the syntax directly to the macro:

\haxe{assets/GenericBuildMacro2.hx}

\haxe{assets/GenericBuild2.hx}

Here the macro logic could load a file and use its contents to generate a custom type.

\paragraph{Related content}
\begin{itemize}
	\item \href{http://code.haxe.org/tag/build-macro.html}{Haxe snippets and tutorials about build macros} in the Haxe Code Cookbook.
\end{itemize}

\section{Limitations}
\label{macro-limitations}
\state{NoContent}

\subsection{Macro-in-Macro}
\label{macro-limitations-macro-in-macro}

\subsection{Static extension}
\label{macro-limitations-static-extension}

The concepts of \tref{static extensions}{lf-static-extension} and macros are somewhat conflicting: While the former requires a known type in order to determine used functions, macros execute before typing on plain syntax. It is thus not surprising that combining these two features can lead to issues. Haxe 3.0 would try to convert the typed expression back to a syntax expression, which is not always possible and may lose important information. We recommend that it is used with caution.

\since{3.1.0}

The combination of static extensions and macros was reworked for the 3.1.0 release. The Haxe Compiler does not even try to find the original expression for the macro argument and instead passes a special \expr{@:this this} expression. While the structure of this expression conveys no information, the expression can still be typed correctly:

\haxe{assets/MacroStaticExtension.hx}



\subsection{Build Order}
\label{macro-limitations-build-order}

The build order of types is unspecified and this extends to the execution order of \tref{build-macros}{macro-type-building}. While certain rules can be determined, we strongly recommend to not rely on the execution order of build-macros. If type building requires multiple passes, this should be reflected directly in the macro code. In order to avoid multiple build-macro execution on the same type, state can be stored in static variables or added as \tref{metadata}{lf-metadata} to the type in question:

\haxe{assets/MacroBuildOrder.hx}

With both interfaces \type{I1} and \type{I2} having \expr{:autoBuild} metadata, the build macro is executed twice for class \type{C}. We guard against duplicate processing by adding a custom \expr{:processed} metadata to the class, which can be checked during the second macro execution.


\subsection{Type Parameters}
\label{macro-limitations-type-parameters}


\section{Initialization Macros}
\label{macro-initialization}

Initialization macros are invoked from command line by using the \expr{--macro callExpr(args)} command. This registers a callback which the compiler invokes after creating its context, but before typing what was argument to \expr{-main}. This then allows configuring the compiler in some ways.

If the argument to \expr{--macro} is a call to a plain identifier, that identifier is looked up in the class \type{haxe.macro.Compiler} which is part of the Haxe Standard Library. It comes with several useful initialization macros which are detailed in its \href{http://api.haxe.org//haxe/macro/Compiler.html}{API}.

As an example, the \expr{include} macro allows inclusion of an entire package for compilation, recursively if necessary. The command line argument for this would then be \expr{--macro include('some.pack', true)}.

Of course it is also possible to define custom initialization macros to perform various tasks before the real compilation. A macro like this would then be invoked via \expr{--macro some.Class.theMacro(args)}. For instance, as all macros share the same \tref{context}{macro-context}, an initialization macro could set the value of a static field which other macros use as configuration.


\part{Standard Library}
\chapter{Standard Library}
\label{std}
\state{NoContent}

\section{String}
\label{std-String}

\define[Type]{String}{define-string}{A String is a sequence of characters.}

%TODO: utf8 crap %

\paragraph{Character code}
Use the \ic{.code} property on a constant single-char string in order to compile its ASCII character code:

\begin{lstlisting}
"#".code // will compile as 35
\end{lstlisting}

\paragraph{Related content}
\begin{itemize}
	\item See the \href{http://api.haxe.org/String.html}{String API} for details about its methods.
\end{itemize} 

\section{Data Structures}
\label{std-ds}
\state{NoContent}

\subsection{Array}
\label{std-Array}

An \type{Array} is a collection of elements. It has one \tref{type parameter}{type-system-type-parameters} which corresponds to the type of these elements. Arrays can be created in three ways:

\begin{enumerate}
	\item By using their constructor: \expr{new Array()}
	\item By using \tref{array declaration syntax}{expression-array-declaration}: \expr{[1, 2, 3]}
	\item By using \tref{array comprehension}{lf-array-comprehension}: \expr{[for (i in 0...10) if (i \% 2 == 0) i]}
\end{enumerate}

Arrays come with an \href{http://api.haxe.org/Array.html}{API} to cover most use-cases. Additionally they allow read and write \tref{array access}{expression-array-access}:

\haxe{assets/ArrayAccess.hx}

Since array access in Haxe is unbounded, i.e. it is guaranteed to not throw an exception, this requires further discussion:

\begin{itemize}
	\item If a read access is made on a non-existing index, a target-dependent value is returned.
	\item If a write access is made with a positive index which is out of bounds, \expr{null} (or the \tref{default value}{define-default-value} for \tref{basic types}{types-basic-types} on \tref{static targets}{define-static-target}) is inserted at all positions between the last defined index and the newly written one.
	\item If a write access is made with a negative index, the result is unspecified.
\end{itemize}

Arrays define an \tref{iterator}{lf-iterators} over their elements. This iteration is typically optimized by the compiler to use a \tref{\expr{while} loop}{expression-while} with array index:

\haxe{assets/ArrayIterator.hx}

Haxe generates this optimized \target{JavaScript} output:

\lang{js}\begin{lstlisting}
Main.main = function() {
	var scores = [110,170,35];
	var sum = 0;
	var _g = 0;
	while(_g < scores.length) {
		var score = scores[_g];
		++_g;
		sum += score;
	}
	console.log(sum);
};
\end{lstlisting}

Haxe does not allow arrays of mixed types unless the parameter type is forced to \tref{\type{Dynamic}}{types-dynamic}:

\haxe{assets/ArrayDynamic.hx}

\trivia{Dynamic Arrays}{In Haxe 2, mixed type array declarations were allowed. In Haxe 3, arrays can have mixed types only if they are explicitly declared as \expr{Array<Dynamic>}.}

\paragraph{Related content}
\begin{itemize}
	\item See the \href{http://api.haxe.org/Array.html}{Array API} for details about its methods. 
	\item \href{http://code.haxe.org/category/data-structures/}{Data structures tutorials and examples} in the Haxe Code Cookbook.
\end{itemize} 

\subsection{Vector}
\label{std-vector}

A \type{Vector} is an optimized fixed-length \emph{collection} of elements. Much like \tref{Array}{std-Array}, it has one \tref{type parameter}{type-system-type-parameters} and all elements of a vector must be of the specified type, it can be \emph{iterated over} using a \tref{for loop}{expression-for} and accessed using \tref{array access syntax}{types-abstract-array-access}. However, unlike \type{Array} and \type{List}, vector length is specified on creation and cannot be changed later.

\haxe{assets/Vector.hx}

\type{haxe.ds.Vector} is implemented as an abstract type (\ref{types-abstract}) over a native array implementation for given target and can be faster for fixed-size collections, because the memory for storing its elements is pre-allocated.

\paragraph{Related content}
\begin{itemize}
	\item See the \href{http://api.haxe.org/haxe/ds/Vector.html}{Vector API} for details about the vector methods. 
	\item \href{http://code.haxe.org/category/data-structures/}{Data structures tutorials and examples} in the Haxe Code Cookbook.
\end{itemize}


\subsection{List}
\label{std-List}
A \type{List} is a \emph{collection} for storing elements.  On the surface, a list is similar to an \Fullref{std-Array}.  However, the underlying implementation is very different.  This results in several functional differences:

\begin{enumerate}
	\item A list can not be indexed using square brackets, i.e. \expr{[0]}.
	\item A list can not be initialized.
	\item There are no list comprehensions.
	\item A list can freely modify/add/remove elements while iterating over them.
\end{enumerate}

A simple example for working with lists:
\haxe{assets/ListExample.hx}

\paragraph{Related content}
\begin{itemize}
	\item See the \href{http://api.haxe.org/List.html}{List API} for details about the list methods. 
	\item \href{http://code.haxe.org/category/data-structures/}{Data structures tutorials and examples} in the Haxe Code Cookbook.
\end{itemize}

\subsection{GenericStack}
\label{std-GenericStack}
A \type{GenericStack}, like \type{Array} and \type{List} is a container for storing elements.  It has one \tref{type parameter}{type-system-type-parameters} and all elements of the stack must be of the specified type.   Here is a small example program for initializing and working with a \type{GenericStack}.
\haxe{assets/GenericStackExample.hx}
\trivia{FastList}{In Haxe 2, the GenericStack class was known as FastList.  Since its behavior more closely resembled a typical stack, the name was changed for Haxe 3.}
The \emph{Generic} in \type{GenericStack} is literal.  It is attributed with the \expr{:generic} metadata.  Depending on the target, this can lead to improved performance on static targets.  See \Fullref{type-system-generic} for more details.

\paragraph{Related content}
\begin{itemize}
	\item See the \href{http://api.haxe.org/haxe/ds/GenericStack.html}{GenericStack API} for details about its methods. 
	\item \href{http://code.haxe.org/category/data-structures/}{Data structures tutorials and examples} in the Haxe Code Cookbook.
\end{itemize}

\subsection{Map}
\label{std-Map}

A \type{Map} is a container composed of \emph{key}, \emph{value} pairs.  A \type{Map} is also commonly referred to as an associative array, dictionary, or symbol table. The following code gives a short example of working with maps:

\haxe{assets/MapExample.hx}

Under the hood, a \type{Map} is an \tref{abstract}{types-abstract} type. At compile time, it gets converted to one of several specialized types depending on the \emph{key} type:
\begin{itemize}
	\item \type{String}: \type{haxe.ds.StringMap}
	\item \type{Int}: \type{haxe.ds.IntMap}
	\item \type{EnumValue}: \type{haxe.ds.EnumValueMap}
	\item \type{\{\}}: \type{haxe.ds.ObjectMap}
\end{itemize}

The \type{Map} type does not exist at runtime and has been replaced with one of the above objects. 

Map defines \tref{array access}{types-abstract-array-access} using its key type.


\paragraph{Related content}
\begin{itemize}
	\item See the \href{http://api.haxe.org/Map.html}{Map API} for details of its methods.
	\item \href{http://code.haxe.org/category/data-structures/}{Data structures tutorials and examples} in the Haxe Code Cookbook.
\end{itemize}

\subsection{Option}
\label{std-Option}

An \href{http://api.haxe.org/haxe/ds/Option.html}{Option} is an \tref{enum}{types-enum-instance} in the Haxe Standard Library which is defined like so:

\begin{lstlisting}
enum Option<T> {
	Some(v:T);
	None;
}
\end{lstlisting}

It can be used in various situations, such as communicating whether or not a method had a valid return and if so, what value it returned:

\haxe{assets/OptionUsage.hx}


\section{Regular Expressions}
\label{std-regex}

Haxe has built-in support for \emph{regular expressions}\footnote{http://en.wikipedia.org/wiki/Regular_expression}. They can be used to verify the format of a string, transform a string or extract some regular data from a given text.

Haxe has special syntax for creating regular expressions. We can create a regular expression object by typing it between the \expr{\textasciitilde/} combination and a single \expr{/} character:

\begin{lstlisting}
var r = ~/haxe/i;
\end{lstlisting}

Alternatively, we can create regular expression with regular syntax:

\begin{lstlisting}
var r = new EReg("haxe", "i");
\end{lstlisting}

First argument is a string with regular expression pattern, second one is a string with \emph{flags} (see below).

We can use standard regular expression patterns such as:
\begin{itemize}
	\item \expr{.} any character
	\item \expr{*} repeat zero-or-more
	\item \expr{+} repeat one-or-more
	\item \expr{?} optional zero-or-one
	\item \expr{[A-Z0-9]} character ranges
	\item \expr{[\textasciicircum\textbackslash r\textbackslash n\textbackslash t]} character not-in-range
	\item \expr{(...)} parenthesis to match groups of characters
	\item \expr{\textasciicircum} beginning of the string (beginning of a line in multiline matching mode)
	\item \expr{\$} end of the string (end of a line in multiline matching mode)
	\item \expr{|} "OR" statement.
\end{itemize}

For example, the following regular expression matches valid email addresses:
\begin{lstlisting}
~/[A-Z0-9._\%-]+@[A-Z0-9.-]+\.[A-Z][A-Z][A-Z]?/i;
\end{lstlisting}

Please notice that the \expr{i} at the end of the regular expression is a \emph{flag} that enables case-insensitive matching.

The possible flags are the following:
\begin{itemize}
	\item \expr{i} case insensitive matching
	\item \expr{g} global replace or split, see below
	\item \expr{m} multiline matching, \expr{\textasciicircum} and \expr{\$} represent the beginning and end of a line
	\item \expr{s} the dot \expr{.} will also match newlines \emph{(Neko, C++, PHP, Flash and Java targets only)}
	\item \expr{u} use UTF-8 matching \emph{(Neko and C++ targets only)}
\end{itemize}

\paragraph{Related content}
\begin{itemize}
	\item See the \href{http://api.haxe.org/EReg.html}{EReg API} for details about its methods. 
	\item \href{http://code.haxe.org/tag/ereg.html}{Haxe snippets and tutorials about regular expressions} in the Haxe Code Cookbook.
\end{itemize}

\subsection{Matching}
\label{std-regex-match}

Probably one of the most common uses for regular expressions is checking whether a string matches the specific pattern. The \expr{match} method of a regular expression object can be used to do that:
\haxe{assets/ERegMatch.hx}

\subsection{Groups}
\label{std-regex-groups}

Specific information can be extracted from a matched string by using \emph{groups}. If \expr{match()} returns true, we can get groups using the \expr{matched(X)} method, where X is the number of a group defined by regular expression pattern:

\haxe{assets/ERegGroups.hx}

Note that group numbers start with 1 and \expr{r.matched(0)} will always return the whole matched substring.

The \expr{r.matchedPos()} will return the position of this substring in the original string:

\haxe{assets/ERegMatchPos.hx}

Additionally, \expr{r.matchedLeft()} and \expr{r.matchedRight()} can be used to get substrings to the left and to the right of the matched substring:

\haxe{assets/ERegMatchLeftRight.hx}

\subsection{Replace}
\label{std-regex-replace}

A regular expression can also be used to replace a part of the string:

\haxe{assets/ERegReplace.hx}

We can use \expr{\$X} to reuse a matched group in the replacement:

\haxe{assets/ERegReplaceGroups.hx}

\subsection{Split}
\label{std-regex-split}

A regular expression can also be used to split a string into several substrings:

\haxe{assets/ERegSplit.hx}

\subsection{Map}
\label{std-regex-map}

The \expr{map} method of a regular expression object can be used to replace matched substrings using a custom function. This function takes a regular expression object as its first argument so we may use it to get additional information about the match being done and do conditional replacement. For example:

\haxe{assets/ERegMap.hx}


\subsection{Implementation Details}
\label{std-regex-implementation-details}

Regular Expressions are implemented:

\begin{itemize}
	\item in JavaScript, the runtime is providing the implementation with the object RegExp.
	\item in Neko and C++, the PCRE library is used
	\item in Flash, PHP, C\# and Java, native implementations are used
	\item in Flash 6/8, the implementation is not available
\end{itemize}


\section{Math}
\label{std-math}

Haxe includes a floating point math library for some common mathematical operations. Most of the functions operate on and return \type{floats}. However, an \type{Int} can be used where a \type{Float} is expected, and Haxe also converts \type{Int} to \type{Float} during most numeric operations  (see \Fullref{types-numeric-operators} for more details).

Here are some example uses of the math library:  

\haxe{assets/MathExample.hx}

\paragraph{Related content}
\begin{itemize}
	\item See the \href{http://api.haxe.org/Math.html}{Math API documentation} for all available functions.
	\item \href{http://code.haxe.org/tag/math.html}{Haxe snippets and tutorials about math} in the Haxe Code Cookbook.
\end{itemize}

\subsection{Special Numbers}
\label{std-math-special-numbers}

The math library has definitions for several special numbers:

\begin{itemize}
	\item NaN (Not a Number): returned when a mathematically incorrect operation is executed, e.g. Math.sqrt(-1)
	\item POSITIVE_INFINITY: e.g. divide a positive number by zero
	\item NEGATIVE_INFINITY: e.g. divide a negative number by zero
	\item PI : 3.1415...
\end{itemize}

\subsection{Mathematical Errors}
\label{std-math-mathematical-errors}
Although neko can fluidly handle mathematical errors, like division by zero, this is not true for all targets.  Depending on the target, mathematical errors may produce exceptions and ultimately errors.

\subsection{Integer Math}
\label{std-math-integer-math}

If you are targeting a platform that can utilize integer operations, e.g. integer division, it should be wrapped in \emph{Std.int()} for improved performance.  The Haxe Compiler can then optimize for integer operations.  An example:

\begin{lstlisting}
	var intDivision = Std.int(6.2/4.7);
\end{lstlisting}

\todo{I think C++ can use integer operatins, but I don't know about any other targets. Only saw this mentioned in an old discussion thread, still true?}

\subsection{Extensions}
\label{std-math-extensions}
It is common to see \Fullref{lf-static-extension} used with the math library.  This code shows a simple example:  
\haxe{assets/MathStaticExtension.hx}
\haxe{assets/MathExtensionUsage.hx}


\section{Lambda}
\label{std-Lambda}

\define{Lambda}{define-lambda}{Lambda is a functional language concept within Haxe that allows you to apply a function to a list or \tref{iterators}{lf-iterators}. The Lambda class is a collection of functional methods in order to use functional-style programming with Haxe.}

It is ideally used with \expr{using Lambda} (see \tref{Static Extension}{lf-static-extension}) and then acts as an extension to \type{Iterable} types. 

On static platforms, working with the \type{Iterable} structure might be slower than performing the operations directly on known types, such as \type{Array} and \type{List}.

\paragraph{Lambda Functions}
The Lambda class allows us to operate on an entire \type{Iterable} at once.
This is often preferable to looping routines since it is less error prone and easier to read. 
For convenience, the \type{Array} and \type{List} class contains some of the frequently used methods from the Lambda class.

It is helpful to look at an example. The exists function is specified as:

\begin{lstlisting}
static function exists<A>( it : Iterable<A>, f : A -> Bool ) : Bool
\end{lstlisting}

Most Lambda functions are called in similar ways. The first argument for all of the Lambda functions is the \type{Iterable} on which to operate. Many also take a function as an argument.

\begin{description}
	\item[\expr{Lambda.array}, \expr{Lambda.list}] Convert Iterable to \type{Array} or \type{List}. It always returns a new instance.
	\item[\expr{Lambda.count}] Count the number of elements.  If the Iterable is a \type{Array} or \type{List} it is faster to use its length property.
	\item[\expr{Lambda.empty}] Determine if the Iterable is empty. For all Iterables it is best to use the this function; it's also faster than compare the length (or result of Lambda.count) to zero.
	\item[\expr{Lambda.has}] Determine if the specified element is in the Iterable.
	\item[\expr{Lambda.exists}] Determine if criteria is satisfied by an element.
	\item[\expr{Lambda.indexOf}] Find out the index of the specified element.
	\item[\expr{Lambda.find}] Find first element of given search function.
	\item[\expr{Lambda.foreach}] Determine if every element satisfies a criteria.
	\item[\expr{Lambda.iter}] Call a function for each element.
	\item[\expr{Lambda.concat}] Merge two Iterables, returning a new List.
	\item[\expr{Lambda.filter}] Find the elements that satisfy a criteria, returning a new List.
	\item[\expr{Lambda.map}, \expr{Lambda.mapi}] Apply a conversion to each element, returning a new List.
	\item[\expr{Lambda.fold}] Functional fold, which is also known as reduce, accumulate, compress or inject.
\end{description}

This example demonstrates the Lambda filter and map on a set of strings:

\begin{lstlisting}
using Lambda;
class Main {
    static function main() {
        var words = ['car', 'boat', 'cat', 'frog'];

		var isThreeLetters = function(word) return word.length == 3;
		var capitalize = function(word) return word.toUpperCase();
		
		// Three letter words and capitalized. 
		trace(words.filter(isThreeLetters).map(capitalize)); // [CAR,CAT]
    }
}
\end{lstlisting} 

This example demonstrates the Lambda count, has, foreach and fold function on a set of ints.

\begin{lstlisting}
using Lambda;
class Main {
    static function main() {
        var numbers = [1, 3, 5, 6, 7, 8];
		
		trace(numbers.count()); // 6
		trace(numbers.has(4)); // false
		
        // test if all numbers are greater/smaller than 20
		trace(numbers.foreach(function(v) return v < 20)); // true
        trace(numbers.foreach(function(v) return v > 20)); // false
		
        // sum all the numbers
		var sum = function(num, total) return total += num;
		trace(numbers.fold(sum, 0)); // 30
    }
}
\end{lstlisting} 

\paragraph{Related content}
\begin{itemize}
	\item See the \href{http://api.haxe.org/Lambda.html}{Lambda API documentation} for all available functions.
\end{itemize}


\section{Template}
\label{std-template}

Haxe comes with a standard template system with an easy to use syntax which is interpreted by a lightweight class called \href{http://api.haxe.org/haxe/Template.html}{haxe.Template}.

A template is a string or a file that is used to produce any kind of string output depending on the input. Here is a small template example:

\haxe{assets/Template.hx}

The console will trace \ic{My name is Mark, 30 years old}.

\paragraph{Expressions}
An expression can be put between the \ic{::}, the syntax allows the current possibilities:

\begin{description}
	\item[\ic{::name::}] the variable name
	\item[\ic{::expr.field::}] field access
	\item[\ic{::(expr)::}] the expression expr is evaluated
	\item[\ic{::(e1 op e2)::}] the operation op is applied to e1 and e2
	\item[\ic{::(135)::}] the integer 135. Float constants are not allowed
\end{description}

\paragraph{Conditions}
It is possible to test conditions using \ic{::if flag1::}. Optionally, the condition may be followed by \ic{::elseif flag2::} or \ic{::else::}. Close the condition with \ic{::end::}.

\lang{none}\begin{lstlisting}
::if isValid:: valid ::else:: invalid ::end::
\end{lstlisting} 

Operators can be used but they don't deal with operator precedence. Therefore it is required to enclose each operation in parentheses \ic{()}. Currently, the following operators are allowed: \ic{+}, \ic{-}, \ic{*}, \ic{/}, \ic{>}, \ic{<},  \ic{>=}, \ic{<=}, \ic{==}, \ic{!=}, \ic{\&\&} and \ic{||}.

For example \ic{::((1 + 3) == (2 + 2))::} will display true. 

\lang{none}\begin{lstlisting} 
::if (points == 10):: Great! ::end::
\end{lstlisting} 

To compare to a string, use double quotes \ic{"} in the template.
\lang{none}\begin{lstlisting}
::if (name == "Mark"):: Hi Mark ::end::
\end{lstlisting} 

\paragraph{Iterating}
Iterate on a structure by using \ic{::foreach::}. End the loop with \ic{::end::}.
\lang{xml}\begin{lstlisting}
<table>
	<tr>
		<th>Name</th>
		<th>Age</th>
	</tr>
	::foreach users::
		<tr>
			<td>::name::</td>
			<td>::age::</td>
		</tr>
	::end::
</table>
\end{lstlisting} 

\paragraph{Sub-templates}
To include templates in other templates, pass the sub-template result string as a parameter.
\begin{lstlisting} 
var users = [{name:"Mark", age:30}, {name:"John", age:45}];

var userTemplate = new haxe.Template("::foreach users:: ::name::(::age::) ::end::");
var userOutput = userTemplate.execute({users: users});

var template = new haxe.Template("The users are ::users::");
var output = template.execute({users: userOutput});
trace(output);
\end{lstlisting} 
The console will trace \ic{The users are Mark(30) John(45)}.

\paragraph{Template macros}
To call custom functions while parts of the template are being rendered, provide a \expr{macros} object to the argument of \href{http://api.haxe.org/haxe/Template.html#execute}{Template.execute}. The key will act as the template variable name, the value refers to a callback function that should return a \type{String}. The first argument of this macro function is always a \expr{resolve()} method, followed by the given arguments. The resolve function can be called to retrieve values from the template context. If \expr{macros} has no such field, the result is unspecified.

The following example passes itself as macro function context and executes \ic{display} from the template.
\haxe{assets/TemplateMacros.hx}
The console will trace \ic{The results: Mark ran 3.5 kilometers in 15 minutes}.

\paragraph{Globals}
Use the \href{http://api.haxe.org/haxe/Template.html#globals}{Template.globals} object to store values that should be applied across all \type{haxe.Template} instances. This has lower priority than the context argument of \expr{Template.execute}.

\paragraph{Using resources}

To separate the content from the code, consider using the \tref{resource embedding system}{cr-resources}. 
Place the template-content in a new file called \ic{sample.mtt}, add \ic{-resource sample.mtt@my_sample} to the compiler arguments and retrieve the content using \expr{haxe.Resource.getString}.
\haxe{assets/TemplateResource.hx}

When running the template system on the server side, you can simply use \expr{neko.Lib.print} or \expr{php.Lib.print} instead of trace to display the HTML template to the user.

\paragraph{Related content}
\begin{itemize}
	\item See the \href{http://api.haxe.org/haxe/Template.html}{Template API} for details about its methods.
\end{itemize}

\section{Reflection}
\label{std-reflection}

Haxe supports runtime reflection of types and fields. Special care has to be taken here because runtime representation generally varies between targets. In order to use reflection correctly it is necessary to understand what kind of operations are supported and what is not. Given the dynamic nature of reflection, this can not always be determined at compile-time.

The reflection API consists of two classes:

\begin{description}
	\item[Reflect:] A lightweight API which work best on \tref{anonymous structures}{types-anonymous-structure}, with limited support for \tref{classes}{types-class-instance}. 
	\item[Type:] A more robust API for working with classes and \tref{enums}{types-enum-instance}.
\end{description}

The available methods are detailed in the API for \href{http://api.haxe.org/Reflect.html}{Reflect} and \href{http://api.haxe.org/Type.html}{Type}.

Reflection can be a powerful tool, but it is important to understand why it can also cause problems. As an example, several functions expect a \tref{String}{std-String} argument and try to resolve it to a type or field. This is vulnerable to typing errors:

\haxe{assets/ReflectionTypo.hx}

However, even if there are no typing errors it is easy to come across unexpected behavior:

\haxe{assets/ReflectionMissingType.hx}

The problem here is that the compiler never actually ``sees'' the type \type{haxe.Template}, so it does not compile it into the output. Furthermore, even if it were to see the type there could be issues arising from \tref{dead code elimination}{cr-dce} eliminating types or fields which are only used via reflection.

Another set of problems comes from the fact that, by design, several reflection functions expect arguments of type \tref{Dynamic}{types-dynamic}, meaning the compiler cannot check if the passed in arguments are correct. The following example demonstrates a common mistake when working with \expr{callMethod}:

\haxe{assets/ReflectionWrongUsage.hx}

The commented out call would be accepted by the compiler because it assigns the string \expr{"f"} to the function argument \expr{func} which is specified to be \expr{Dynamic}.

A good advice when working with reflection is to wrap it in a few functions within an application or API which are called by otherwise type-safe code. An example could look like this:

\haxe{assets/ReflectionWrap.hx}

While the method \expr{reflective} could internally work with reflection (and \type{Dynamic} for that matter) a lot, its return value is a typed structure which the callers can use in a type-safe manner.


\section{Serialization}
\label{std-serialization}

Many runtime values can be serialized and deserialized using the \href{http://api.haxe.org/haxe/Serializer.html}{haxe.Serializer} and \href{http://api.haxe.org/haxe/Unserializer.html}{haxe.Unserializer} classes. Both support two usages:

\begin{enumerate}
	\item Create an instance and continuously call the \expr{serialize}/\expr{unserialize} method to handle multiple values.
	\item Call their static \expr{run} method to serialize/deserialize a single value.
\end{enumerate}

The following example demonstrates the first usage:

\haxe{assets/SerializationExample.hx}

The result of the serialization (here stored in local variable \expr{s}) is a \tref{String}{std-String} and can be passed around at will, even remotely. Its format is described in \Fullref{std-serialization-format}.

\paragraph{Supported values}

\begin{itemize}
	\item \expr{null}
	\item \type{Bool}, \type{Int} and \type{Float} (including infinities and \expr{NaN})
	\item \type{String}
	\item \type{Date}
	\item \type{haxe.io.Bytes} (encoded as base64)
	\item \tref{\type{Array}}{std-Array} and \tref{\type{List}}{std-List}
	\item \type{haxe.ds.StringMap}, \type{haxe.ds.IntMap} and \type{haxe.ds.ObjectMap}
	\item \tref{anonymous structures}{types-anonymous-structure}
	\item Haxe \tref{class instances}{types-class-instance} (not native ones)
	\item \tref{enum instances}{types-enum-instance}
\end{itemize}

\paragraph{Serialization configuration}

Serialization can be configured in two ways. For both a static variable can be set to influence all \href{http://api.haxe.org/haxe/Serializer.html}{haxe.Serializer} instances, and a member variable can be set to only influence a specific instance:

\begin{description}
	\item[\expr{USE_CACHE}, \expr{useCache}:] If true, repeated structures or class\slash enum instances are serialized by reference. This can avoid infinite loops for recursive data at the expense of longer serialization time. By default, object caching is disabled; strings however are always cached.
	\item[\expr{USE_ENUM_INDEX}, \expr{useEnumIndex}:] If true, enum constructors are serialized by their index instead of their name. This can make the resulting string shorter, but breaks if enum constructors are inserted into the type before deserialization. This behavior is disabled by default.
\end{description}

\paragraph{Deserialization behavior}

If the serialization result is stored and later used for deserialization, care has to be taken to maintain compatibility when working with class and enum instances. It is then important to understand exactly how unserialization is implemented.

\begin{itemize}
	\item The type has to be available in the runtime where the deserialization is made. If \tref{dead code elimination}{cr-dce} is active, a type which is used only through serialization might be removed.
	\item Each \type{Unserializer} has a member variable \expr{resolver} which is used to resolve classes and enums by name. Upon creation of the \type{Unserializer} this is set to \expr{Unserializer.DEFAULT_RESOLVER}. Both that and the instance member can be set to a custom resolver.
	\item Classes are resolved by name using \expr{resolver.resolveClass(name)}. The instance is then created using \expr{Type.createEmptyInstance}, which means that the class constructor is not called. Finally, the instance fields are set according to the serialized value.
	\item Enums are resolved by name using \expr{resolver.resolveEnum(name)}. The enum instance is then created using \expr{Type.createEnum}, using the serialized argument values if available. If the constructor arguments were changed since serialization, the result is unspecified.
\end{itemize}

\paragraph{Custom (de)serialization}

If a class defines the member method \expr{hxSerialize}, that method is called by the serializer and allows custom serialization of the class. Likewise, if a class defines the member method \expr{hxUnserialize} it is called by the deserializer:

\haxe{assets/SerializationCustom.hx}

In this example we decide that we want to ignore the value of member variable \expr{y} and do not serialize it. Instead we default it to \expr{-1} in \expr{hxUnserialize}. Both methods are annotated with the \expr{@:keep} metadata to prevent \tref{dead code elimination}{cr-dce} from removing them as they are never properly referenced in the code.

See \href{http://api.haxe.org/haxe/Serializer.html}{Serializer} and \href{http://api.haxe.org/haxe/Unserializer.html}{Unserializer} API documentation for details.

\subsection{Serialization format}
\label{std-serialization-format}

Each supported value is translated to a distinct prefix character, followed by the necessary data.

\begin{description}
	\item[\expr{null}:] \expr{n}
	\item[\type{Int}:] \expr{z} for zero, or \expr{i} followed by the integer display (e.g. \expr{i456})
	\item[\type{Float}:] \mbox{}
		\begin{description}
			\item[\expr{NaN}:] \expr{k}
			\item[negative infinity:] \expr{m}
			\item[positive infinity:] \expr{p}
			\item[finite floats:] \expr{d} followed by the float display (e.g. \expr{d1.45e-8})
		\end{description}
	\item[\type{Bool}:] \expr{t} for \expr{true}, \expr{f} for \expr{false}
	\item[\type{String}:] \expr{y} followed by the url encoded string length, then \expr{:} and the url encoded string (e.g. \expr{y10:hi\%20there for "hi there".}
	\item[name-value pairs:] a serialized string representing the name followed by the serialized value
	\item[structure:] \expr{o} followed by the list of name-value pairs and terminated by \expr{g} (e.g. \expr{oy1:xi2y1:kng} for \expr{\{x:2, k:null\}})
	\item[\type{List}:] \expr{l} followed by the list of serialized items, followed by \expr{h} (e.g. \expr{lnnh} for a list of two \expr{null} values)
	\item[\type{Array}:] \expr{a} followed by the list of serialized items, followed by \expr{h}. For multiple consecutive \expr{null} values, \expr{u} followed by the number of \expr{null} values is used (e.g. \expr{ai1i2u4i7ni9h for [1,2,null,null,null,null,7,null,9]})
	\item[\type{Date}:] \expr{v} followed by the date itself (e.g. \expr{v2010-01-01 12:45:10})
	\item[\type{haxe.ds.StringMap}:] \expr{b} followed by the name-value pairs, followed by \expr{h} (e.g. \expr{by1:xi2y1:knh} for \expr{\{"x" => 2, "k" => null\}})
	\item[\type{haxe.ds.IntMap}:] \expr{q} followed by the key-value pairs, followed by \expr{h}. Each key is represented as \expr{:<int>} (e.g. \expr{q:4n:5i45:6i7h} for \expr{\{4 => null, 5 => 45, 6 => 7\}})
	\item[\type{haxe.ds.ObjectMap}:] \expr{M} followed by serialized value pairs representing the key and value, followed by \expr{h}
	\item[\type{haxe.io.Bytes}:] \expr{s} followed by the length of the base64 encoded bytes, then \expr{:} and the byte representation using the codes \expr{A-Za-z0-9\%} (e.g. \expr{s3:AAA} for 2 bytes equal to \expr{0}, and \expr{s10:SGVsbG8gIQ} for \expr{haxe.io.Bytes.ofString("Hello !")})
	\item[exception:] \expr{x} followed by the exception value
	\item[class instance:] \expr{c} followed by the serialized class name, followed by the name-value pairs of the fields, followed by \expr{g} (e.g. \expr{cy5:Pointy1:xzy1:yzg} for \expr{new Point(0, 0)} (having two integer fields \expr{x} and \expr{y})
        \item[enum instance (by name):] \expr{w} followed by the serialized enum name, followed by the serialized constructor name, followed by \expr{:}, followed by the number of arguments, followed by the argument values (e.g. \expr{wy3:Fooy1:A:0} for \expr{Foo.A} (with no arguments), \expr{wy3:Fooy1:B:2i4n} for \expr{Foo.B(4,null)})
	\item[enum instance (by index):] \expr{j} followed by the serialized enum name, followed by \expr{:}, followed by the constructor index (starting from 0), followed by \expr{:}, followed by the number of arguments, followed by the argument values (e.g. \expr{jy3:Foo:0:0} for \expr{Foo.A} (with no arguments), \expr{jy3:Foo:1:2i4n} for \expr{Foo.B(4,null)})
	\item[cache references:] \mbox{}
		\begin{description}
			\item[\type{String}:] \expr{R} followed by the corresponding index in the string cache (e.g. \expr{R456})
			\item[class, enum or structure] \expr{r} followed by the corresponding index in the object cache (e.g. \expr{r42})
		\end{description}
	\item[custom:] \expr{C} followed by the class name, followed by the custom serialized data, followed by \expr{g}
\end{description}

\noindent Cached elements and enum constructors are indexed from zero.

\section{Xml}
\label{std-Xml}

Haxe provides built-in support for working with \emph{XML}\footnote{http://en.wikipedia.org/wiki/XML} data via the \href{http://api.haxe.org/Xml.html}{haxe.Xml} class. 

\subsection{Getting started with Xml}
\label{std-Xml-getting-started}

\paragraph{Creating a root element}

A \type{Xml} root element can be created using the \expr{Xml.createElement} method.
\begin{lstlisting}
var root = Xml.createElement('root');
trace(root); // <root />
\end{lstlisting}

An root node element can also be created by parsing a \type{String} containing the XML data.
\begin{lstlisting}
var root = Xml.parse('<root />').firstElement();
trace(root); // <root />
\end{lstlisting}

\paragraph{Creating child elements}

Adding child elements to the root can be done using the \expr{addChild} method.
\begin{lstlisting}
var child:Xml = Xml.createElement('child'); 
root.addChild(child);
trace(root); // <root><child/></root>
\end{lstlisting}


Adding attributes to an element can be done by using the \expr{set()} method.
\begin{lstlisting}
child.set('name', 'John');
trace(root); // <root><child name="John"/></root>
\end{lstlisting}

\paragraph{Accessing elements and values}

This code parses an XML string into an object structure \type{Xml} and then accesses properties of the object.
\begin{lstlisting}
var xmlString = '<hello name="world!">Haxe is great!</hello>';
var xml:Xml = Xml.parse(xmlString).firstElement();
	
trace(xml.nodeName); // hello
trace(xml.get('name')); // world!
trace(xml.firstChild().nodeValue); // Haxe is great!
\end{lstlisting}

The difference between \expr{firstChild} and \expr{firstElement} is that the second function will return the first child with the type \type{Xml.Element}. 

\paragraph{Iterate on Xml elements}
We can as well use other methods to iterate either over children or elements.
\begin{lstlisting}
for (child in xml) {
	// iterate on all children.
}
for (elt in xml.elements()) {
	// iterate on all elements.
}
for (user in xml.elementsNamed("user")) {
	// iterate on all elements with a nodeName "user".
}
for (att in xml.attributes()) {
	// iterator on all attributes.
}
\end{lstlisting}

See \href{http://api.haxe.org/Xml.html}{Xml} API documentation for details about its methods.


\subsection{Parsing Xml}
\label{std-Xml-parsing}

The static method \href{http://api.haxe.org/Xml.html#parse}{Xml.parse} can be used to parse \emph{XML} data and obtain a Haxe value from it.

\begin{lstlisting}
var xml = Xml.parse('<root>Haxe is great!</root>').firstElement();
trace(xml.firstChild().nodeValue);
\end{lstlisting}

\subsection{Encoding Xml}
\label{std-Xml-encoding}

The method \href{http://api.haxe.org/Xml.html#toString}{xml.toString()} can be used to obtain the \type{String} representation.
\begin{lstlisting}
var xml = Xml.createElement('root');
xml.addChild(Xml.createElement('child1'));
xml.addChild(Xml.createElement('child2'));

trace(xml.toString()); // <root><child1/><child2/></root>
\end{lstlisting}

\section{Json}
\label{std-Json}

Haxe provides built-in support for (de-)serializing \emph{JSON}\footnote{http://en.wikipedia.org/wiki/JSON} data via the \ic{haxe.Json} class.

\paragraph{Related content}
\begin{itemize}
	\item See the \href{http://api.haxe.org/haxe/Json.html}{Haxe Json API documentation}.
	\item \href{http://code.haxe.org/tag/json.html}{Haxe snippets and tutorials about JSON} in the Haxe Code Cookbook.
\end{itemize}

\subsection{Parsing JSON}
\label{std-Json-parsing}

Use the \href{http://api.haxe.org/haxe/Json.html#parse}{haxe.Json.parse} static method to parse \emph{JSON} data and obtain a Haxe value from it:
\haxe{assets/JsonParse.hx}

Note that the type of the object returned by \expr{haxe.Json.parse} is \expr{Dynamic}, so if the structure of our data is well-known, we may want to specify a type using \tref{anonymous structures}{types-anonymous-structure}. This way we provide compile-time checks for accessing our data and most likely more optimal code generation, because compiler knows about types in a structure:
\haxe{assets/JsonParseTyped.hx}

\subsection{Encoding JSON}
\label{std-Json-encoding}

Use the \href{http://api.haxe.org/haxe/Json.html#stringify}{haxe.Json.stringify} static method to encode a Haxe value into a \emph{JSON} string:
\haxe{assets/JsonStringify.hx}


\subsection{Implementation details}
\label{std-Json-implementation-details}

The \href{http://api.haxe.org/haxe/Json.html}{haxe.Json} API automatically uses native implementation on targets where it is available, i.e. \emph{JavaScript}, \emph{Flash} and \emph{PHP} and provides its own implementation for other targets.

Usage of Haxe own implementation can be forced with \expr{-D haxeJSON} compiler argument. This will also provide serialization of \tref{enums}{types-enum-instance} by their index, \tref{maps}{std-Map} with string keys and class instances.

Older browsers (Internet Explorer 7, for instance) may not have native \emph{JSON} implementation. In case it's required to support them, we can include one of the JSON implementations available on the internet in the HTML page. Alternatively, a \expr{-D old_browser} compiler argument that will make \type{haxe.Json} try to use native JSON and, in case it's not available, fallback to its own implementation.

\section{Input/Output}
\label{std-input-output}

\section{Sys/sys}
\label{std-sys}

\section{Remoting}
\label{std-remoting}

Haxe remoting is a way to communicate between different platforms. With Haxe remoting, applications can transmit data transparently, send data and call methods between server and client side.

\paragraph{Related content}
\begin{itemize}
	\item See the \href{http://api.haxe.org/haxe/remoting/}{remoting package} on the API documentation for more details on its classes.
\end{itemize}

\subsection{Remoting Connection}
\label{std-remoting-connection}

In order to use remoting, there must be a connection established. There are two kinds of Haxe Remoting connections: 
\begin{description}
	\item[\href{http://api.haxe.org/haxe/remoting/Connection.html}{haxe.remoting.Connection}] is used for \emph{synchronous connections}, where the results can be directly obtained when calling a method. 
	\item[\href{http://api.haxe.org/haxe/remoting/AsyncConnection.html}{haxe.remoting.AsyncConnection}] is used for \emph{asynchronous connections}, where the results are events that will happen later in the execution process.
\end{description}

\paragraph{Start a connection}
There are some target-specific constructors with different purposes that can be used to set up a connection:

\begin{description}
	\item[All targets:]
		\begin{description}
			\item[\expr{HttpAsyncConnection.urlConnect(url:String)}]  
				Returns an asynchronous connection to the given URL which should link to a Haxe server application. 
		\end{description}
		
	\item[Flash:]
		\begin{description}
			\item[\expr{ExternalConnection.jsConnect(name:String, ctx:Context)}]  
				Allows a connection to the local JavaScript Haxe code. The JS Haxe code must be compiled with the class ExternalConnection included. This only works with Flash Player 8 and higher.
			\item[\expr{AMFConnection.urlConnect(url:String)} and \expr{AMFConnection.connect( cnx : NetConnection )}]  
				Allows a connection to an \href{http://en.wikipedia.org/wiki/Action_Message_Format}{AMF Remoting server} such as \href{http://www.adobe.com/products/adobe-media-server-family.html}{Flash Media Server} or \href{http://www.silexlabs.org/amfphp/}{AMFPHP}.
			\item[\expr{SocketConnection.create(sock:flash.XMLSocket)}]  
				Allows remoting communications over an \type{XMLSocket}
			\item[\expr{LocalConnection.connect(name:String)}]  
				Allows remoting communications over a \href{http://api.haxe.org/haxe/remoting/LocalConnection.html}{Flash LocalConnection}
		\end{description}
		
	\item[JavaScript:]
		\begin{description}
			\item[\expr{ExternalConnection.flashConnect(name:String, obj:String, ctx:Context)}]  
				Allows a connection to a given Flash Object. The Haxe Flash content must be loaded and it must include the \expr{haxe.remoting.Connection} class. This only works with Flash 8 and higher. 
		\end{description}
		
	\item[Neko:]
		\begin{description}
			\item[\expr{HttpConnection.urlConnect(url:String)}]  
				Will work like the asynchronous version but in synchronous mode.
			\item[\expr{SocketConnection.create(...)}]  
				Allows real-time communications with a Flash client which is using an \type{XMLSocket} to connect to the server.
		\end{description}
\end{description}

\paragraph{Remoting context}

Before communicating between platforms, a remoting context has to be defined. This is a shared API that can be called on the connection at the client code.

This server code example creates and shares an API:
\begin{lstlisting}
class Server {
	function new() { }
	function foo(x, y) { return x + y; }

	static function main() {
		var ctx = new haxe.remoting.Context();
		ctx.addObject("Server", new Server());
		
		if(haxe.remoting.HttpConnection.handleRequest(ctx))
		{
			return;
		}
		
		// handle normal request
		trace("This is a remoting server !");
	} 
}
\end{lstlisting}

\paragraph{Using the connection}

Using a connection is pretty convenient. Once the connection is obtained, use classic dot-access to evaluate a path and then use \expr{call()} to call the method in the remoting context and get the result.
The asynchronous connection takes an additional function parameter that will be called when the result is available.

This client code example connects to the server remoting context and calls a function \expr{foo()} on its API.
\begin{lstlisting}
class Client {
  static function main() {
    var cnx = haxe.remoting.HttpAsyncConnection.urlConnect("http://localhost/");
    cnx.setErrorHandler( function(err) { trace('Error: $err'); } );
    cnx.Server.foo.call([1,2], function(data) { trace('Result: $data'); });
  }
}
\end{lstlisting}

To make this work for the Neko target, setup a Neko Web Server, point the url in the Client to \ic{"http://localhost2000/remoting.n"} and compile the Server using \ic{-main Server -neko remoting.n}.

\paragraph{Error handling}

\begin{itemize}
	\item When an error occurs in a asynchronous call, the error handler is called as seen in the example above.
	\item When an error occurs in a synchronous call, an exception is raised on the caller-side as if we were calling a local method.
\end{itemize}

\paragraph{Data serialization}

Haxe Remoting can send a lot of different kinds of data. See \tref{Serialization}{std-serialization}.


\paragraph{Related content}
\begin{itemize}
	\item See the \href{http://api.haxe.org/haxe/remoting/}{remoting package} on the API documentation for more details on its classes.
\end{itemize}

\subsection{Implementation details}
\label{std-remoting-implementation-details}

\paragraph{JavaScript security specifics}

The html-page wrapping the js client must be served from the same domain as the one where the server is running. The same-origin policy restricts how a document or script loaded from one origin can interact with a resource from another origin. The same-origin policy is used as a means to prevent some of the cross-site request forgery attacks.

To use the remoting across domain boundaries, CORS (cross-origin resource sharing) needs to be enabled by defining the header \ic{X-Haxe-Remoting} in the \ic{.htaccess}:

\lang{none}\begin{lstlisting}
# Enable CORS
Header set Access-Control-Allow-Origin "*"
Header set Access-Control-Allow-Methods: "GET,POST,OPTIONS,DELETE,PUT"
Header set Access-Control-Allow-Headers: X-Haxe-Remoting
\end{lstlisting} 

See \href{http://en.wikipedia.org/wiki/Same-origin_policy}{same-origin policy} for more information on this topic.

Also note that this means that the page can't be served directly from the file system \ic{"file:///C:/example/path/index.html"}.

\paragraph{Flash security specifics}

When Flash accesses a server from a different domain, set up a \ic{crossdomain.xml} file on the server, enabling the \ic{X-Haxe} headers.

\lang{xml}\begin{lstlisting}
<cross-domain-policy>
	<allow-access-from domain="*"/> <!-- or the appropriate domains -->
	<allow-http-request-headers-from domain="*" headers="X-Haxe*"/>
</cross-domain-policy>
\end{lstlisting} 

\paragraph{Arguments types are not ensured}

There is no guarantee of any kind that the arguments types will be respected when a method is called using remoting. 
That means even if the arguments of function \expr{foo} are typed to \type{Int}, the client will still be able to use strings while calling the method. 
This can lead to security issues in some cases. When in doubt, check the argument type when the function is called by using the \expr{Std.is} method.

\section{Unit Testing}
\label{std-unit-testing}

The Haxe Standard Library provides basic unit testing classes from the \href{http://api.haxe.org/haxe/unit/}{haxe.unit} package. 

\paragraph{Creating new test cases}

First, create a new class extending \href{http://api.haxe.org/haxe/unit/TestCase.html}{haxe.unit.TestCase} and add own test methods. Every test method name must start with "\ic{test}".

\haxe{assets/UnitTestCase.hx}

\paragraph{Running unit tests}
To run the test, an instance of \href{http://api.haxe.org/haxe/unit/TestRunner.html}{haxe.unit.TestRunner} has to be created. Add the \href{http://api.haxe.org/haxe/unit/TestCase.html}{TestCase} using the \expr{add} method and call \expr{run} to start the test.

\haxe{assets/UnitTestRunner.hx}

The result of the test looks like this:
\lang{none}\begin{lstlisting}
Class: MyTestCase
.
OK 1 tests, 0 failed, 1 success
\end{lstlisting} 

\paragraph{Test functions}
The \type{haxe.unit.TestCase} class comes with three test functions.

\begin{description}
	\item[\expr{assertEquals(expected, actual)}] Succeeds if \ic{expected} and \ic{actual} are equal
	\item[\expr{assertTrue(a)}] Succeeds if \ic{a} is \expr{true}
	\item[\expr{assertFalse(a)}] Succeeds if \ic{a} is \expr{false}
\end{description}

\paragraph{Setup and tear down}

To run code before or after the test, override the functions \expr{setup} and \expr{tearDown} in the \expr{TestCase}. 

\begin{description}
	\item[\expr{setup}] is called before each test runs.
	\item[\expr{tearDown}] is called once after all tests are run.
\end{description}

\haxe{assets/UnitTestSetup.hx}

\paragraph{Comparing Complex Objects}

With complex objects it can be difficult to generate expected values to compare to the actual ones. It can also be a problem that \expr{assertEquals} doesn't do a deep comparison. One way around these issues is to use a string as the expected value and compare it to the actual value converted to a string using \expr{Std.string}. Below is a trivial example using an array.

\begin{lstlisting} 
public function testArray() {
  var actual = [1,2,3];
  assertEquals("[1, 2, 3]", Std.string(actual));
}
\end{lstlisting} 

\paragraph{Run unit test}

This is an example showing how to run your unit tests (on Neko and Node.js) after compilation using a \tref{HXML}{compiler-usage-hxml}.

\begin{lstlisting} 
-cp source\main\haxe
-cp source\test\haxe
-main your.package.TestRunnerMain
--each

-neko output\neko\test.n
-cmd neko .\output\neko\test.n
--next

-js output\javascript\test.js
-cmd node .\output\javascript\test.js
\end{lstlisting} 

\paragraph{Related content}
\begin{itemize}
	\item See the \href{http://api.haxe.org/haxe/unit/}{haxe.unit} package on the API documentation for more details.
\end{itemize}


\part{Miscellaneous}
\chapter{Haxelib}
\label{haxelib}

Haxelib's documentation is available at \href{https://lib.haxe.org/documentation/using-haxelib/}{https://lib.haxe.org/documentation/using-haxelib/}.

\chapter{Target Details}
\label{target-details}
\state{NoContent}

\section{JavaScript}
\label{target-javascript}

Haxe provides the ability to target JavaScript. It does so by transpiling Haxe to JavaScript. The current implementation targets ECMAScript 5. But it is possible to target lower versions using \ic{-D js-es=<value>}.

When choosing the JavaScript target, only the used Haxe code of the project (and used parts of the standard library) are transpiled to JavaScript. This results in optimal filesize which is also readable. \tref{Source maps}{debugging-source-map-javascript} are available and there are several ways to get \tref{debug}{debugging} information. The standard library aims to have same functionality across all targets.

\paragraph{Usage}

You may want to compile Haxe to JavaScript in the following scenarios:

\emph{Client-side JavaScript}
Interacting with DOM elements. Haxe provides up to date typed interfaces to interact with the Document Object Model, allowing creation and update of DOM elements. 

Haxe can be used together with existing third-party libraries and frameworks, such as jQuery, React or Vue. To access third-party frameworks with a strongly-typed API, there are extern libraries available on \href{http://lib.haxe.org/t/js/}{Haxelib}. Alternatively, it is possible to create own externs (see \Fullref{target-javascript-external-libraries}) or use the dynamic type to access any framework, see \Fullref{target-javascript-injection}.

\emph{Creating graphics using Canvas and WebGL}
Use Haxe to create graphical elements on a web page using WebGL. 

Libraries like \href{http://www.openfl.org/}{OpenFL}, \href{http://heaps.io/}{Heaps} and \href{http://kha.tech/}{Kha} make use of WebGL as one of their backends.

\emph{Creating Haxe code that targets server-side JavaScript}
Working with server-side technology. Haxe can be used to create server-side JavaScript like Node.js.

\Fullref{target-javascript-getting-started}

\subsection{Getting started with Haxe/JavaScript}
\label{target-javascript-getting-started}

Haxe can be a powerful tool for developing JavaScript applications. Let's look at our first sample.
This is a very simple example showing the toolchain. 

Create a new folder and save this class as \ic{Main.hx}.

\begin{lstlisting}
import js.Browser;
class Main {
    static function main() {
        var button = Browser.document.createButtonElement();
        button.textContent = "Click me!";
        button.onclick = function(event) {
            Browser.alert("Haxe is great");
        }
        Browser.document.body.appendChild(button);
    }
}
\end{lstlisting}

To compile, either run the following from the command line:

\begin{lstlisting}
haxe -js main-javascript.js -main Main
\end{lstlisting}

Another possibility is to create and run (double-click) a file called \ic{compile.hxml}. In this example the hxml file should be in the same directory as the example class.

\begin{lstlisting}
-js main-javascript.js
-main Main
\end{lstlisting}

The output will be a main-javascript.js, which creates and adds a clickable button to the document body.

\paragraph{Run the JavaScript}

To display the output in a browser, create an HTML-document called \ic{index.html} and open it.

\begin{lstlisting}
<!DOCTYPE html>
<html>
	<body>
		<script src="main-javascript.js"></script>
	</body>
</html>
\end{lstlisting}

\paragraph{More information}

\begin{itemize}
	\item \Fullref{debugging-javascript}
	\item \href{https://code.haxe.org/category/javascript/}{Haxe/JavaScript tutorials}
	\item \href{https://api.haxe.org/js/}{Haxe JavaScript API docs}
	\item \href{https://developer.mozilla.org/en-US/docs/Web/JavaScript/Reference}{MDN JavaScript Reference}
\end{itemize}

\subsection{Using external JavaScript libraries}
\label{target-javascript-external-libraries}

The \tref{externs mechanism}{lf-externs} provides access to the native APIs in a type-safe manner. It assumes that the defined types exist at run-time but assumes nothing about how and where those types are defined. 

An example of an extern class is the \href{https://github.com/HaxeFoundation/haxe/blob/development/std/js/jquery/JQuery.hx}{jQuery class} of the Haxe Standard Library. 
To illustrate, here is a simplified version of this extern class:

\begin{lstlisting}
package js.jquery;
@:native("$") extern class JQuery {
	/**
		Creates DOM elements on the fly from the provided string of raw HTML.
		OR
		Accepts a string containing a CSS selector which is then used to match a set of elements.
		OR
		Binds a function to be executed when the DOM has finished loading.
	**/
	@:selfCall
	@:overload(function(element:js.html.Element):Void { })
	@:overload(function(selection:js.jquery.JQuery):Void { })
	@:overload(function(callback:haxe.Constraints.Function):Void { })
	@:overload(function(selector:String, ?context:haxe.extern.EitherType<js.html.Element, js.jquery.JQuery>):Void { })
	public function new():Void;

	/**
		Adds the specified class(es) to each element in the set of matched elements.
	**/
	@:overload(function(_function:Int -> String -> String):js.jquery.JQuery { })
	public function addClass(className:String):js.jquery.JQuery;

	/**
		Get the HTML contents of the first element in the set of matched elements.
		OR
		Set the HTML contents of each element in the set of matched elements.
	**/
	@:overload(function(htmlString:String):js.jquery.JQuery { })
	@:overload(function(_function:Int -> String -> String):js.jquery.JQuery { })
	public function html():String;
}
\end{lstlisting}

Note that functions can be overloaded to accept different types of arguments and return values, using the \expr{@:overload} metadata. Function overloading works only in externs.

Using this extern, we can use jQuery like this:

\begin{lstlisting}
import js.jquery.*;
..
new JQuery("#my-div").addClass("brand-success").html("haxe is great!");
..
\end{lstlisting}

The package and class name of the extern class should be the same as defined in the external library. If that is not the case, rewrite the path of a class using \expr{@:native}.

\begin{lstlisting}
package my.application.media;

@:native('external.library.media.video')
extern class Video {
..
\end{lstlisting}

Some JavaScript libraries favor instantiating classes without using the \expr{new} keyword. To prevent the Haxe compiler outputting the \expr{new} keyword when using a class, we can attach a \expr{@:selfCall} metadata to its constructor. For example, when we instantiate the jQuery extern class above, \expr{new JQuery()} will be outputted as \expr{\$()} instead of \expr{new \$()}. The \expr{@:selfCall} metadata can also be attached to a method. In this case, the method will be interpreted as a direct call to the object, illustrated as follows:

\begin{lstlisting}
extern class Functor {
	public function new():Void;
	@:selfCall function call():Void;
}

class Test {
	static function main() {
		var f = new Functor();
		f.call(); // will be outputted as `f();`
	}
}
\end{lstlisting}

Beside externs, \tref{Typedefs}{type-system-typedef} can be another great way to name (or alias) a JavaScript type. The major difference between typedefs and externs is that, typedefs are duck-typed but externs are not. Typedefs are suitable for common data structures, e.g. point (\expr{\{x:Float, y:Float\}}). Use of a point structure typedef for function arguments allows external JavaScript functions to accept point class instances from Haxe or from another JavaScript library. It is also useful for typing JSON objects.

The Haxe Standard Library comes with externs of \href{https://jquery.com/}{jQuery} and \href{http://blog.deconcept.com/swfobject/}{SWFObject}. Their version compatibility is summarized as follows:

\begin{center}
\begin{tabular}{| l | l | l |}
	\hline
	Haxe version & Library               & Externs location \\ \hline
	3.3          & jQuery 1.12.1 / 2.2.1 & <code>js.jquery.*</code> \\
	3.2-         & jQuery 1.6.4          & <code>js.JQuery</code> \\
	3.3          & SWFObject 2.3         & <code>js.swfobject.*</code> \\
	3.2-         & SWFObject 1.5         & <code>js.SWFObject</code> \\ \hline
\end{tabular}
\end{center}

There are many externs for other popular native libraries available on \tref{Haxelib library}{haxelib}. To view a list of them, check out the \href{http://lib.haxe.org/t/extern/}{extern tag}.

\subsection{Using Untyped JavaScript}
\label{target-javascript-injection}

In Haxe, it is possible to call an exposed function thanks to the \expr{untyped} keyword. This can be useful in some cases if we don't want to write externs. Anything untyped that is valid syntax will be generated as it is.

\begin{lstlisting}
untyped window.trackEvent("page1");  
\end{lstlisting}

See also \Fullref{target-javascript-untyped} to inject raw JavaScript.

\subsection{JavaScript untyped functions}
\label{target-javascript-untyped}

These functions allow to access specific JavaScript platform features. It works only when the Haxe compiler is targeting JavaScript and should always be prefixed with \expr{untyped}. 

\emph{Important note:} Before using these functions, make sure there is no alternative available in the Haxe Standard Library. The resulting syntax can not be validated by the Haxe compiler, which may result in invalid or error-prone code in the output.

\paragraph{\expr{untyped __js__(expr, params)}}
\tref{Injects raw JavaScript expressions}{target-javascript-injection}. It's allowed to use \ic{\{0\}}, \ic{\{1\}}, \ic{\{2\}} etc in the expression and use the rest arguments to feed Haxe fields. The Haxe compiler will take care of the surrounding quotes if needed. The function can also return values.

\begin{lstlisting}
untyped __js__('alert("Haxe is great!")');
// output: alert("Haxe is great!");

var myMessage = "Haxe is great!";
untyped __js__('alert({0})', myMessage);
// output: 
//	var myMessage = "Haxe is great!";
//	alert(myMessage);

var myVar:Bool = untyped __js__('confirm({0})', "Are you sure?");
// output: var myVar = confirm("Are you sure?");

var hexString:String = untyped __js__('({0}).toString({1})', 255, 16);
// output: var hexString = (255).toString(16);
\end{lstlisting}

\paragraph{\expr{untyped __instanceof__(o,cl)}} 
Same as \ic{o instanceof cl} in JavaScript.

\begin{lstlisting}
var myString = new String("Haxe is great");
var isString = untyped __instanceof__(myString, String);
output: var isString = (myString instanceof String);
\end{lstlisting}

\paragraph{\expr{untyped __typeof__(o)}} 
Same as \ic{typeof o} in JavaScript.

\begin{lstlisting}
var isNodeJS = untyped __typeof__(window) == null;
output: var isNodeJS = typeof(window) == null;
\end{lstlisting}

\paragraph{\expr{untyped __strict_eq__(a,b)}} 
Same as \ic{a === b}  in JavaScript, tests on \href{https://developer.mozilla.org/en-US/docs/Web/JavaScript/Equality_comparisons_and_sameness}{strict equality} (or "triple equals" or "identity").

\begin{lstlisting}
var a = "0";
var b = 0;
var isEqual = untyped __strict_eq__(a, b);
output: var isEqual = ((a) === b);
\end{lstlisting}

\paragraph{\expr{untyped __strict_neq__(a,b)}} 
Same as \ic{a !== b}  in JavaScript, tests on negative strict equality.

\begin{lstlisting}
var a = "0";
var b = 0;
var isntEqual = untyped __strict_neq__(a, b);
output: var isntEqual = ((a) !== b);
\end{lstlisting}

\paragraph{Expression injection} 

In some cases it may be needed to inject raw JavaScript code into Haxe-generated code. With the \expr{__js__} function we can inject pure JavaScript code fragments into the output. This code is always untyped and can not be validated, so it accepts invalid code in the output, which is error-prone.
This could, for example, write a JavaScript comment in the output.

\begin{lstlisting}
untyped __js__('// haxe is great!');
\end{lstlisting}

A more useful demonstration would be to call a function and pass  arguments using the \expr{__js__} function. This example illustrates how to call this function and how to pass parameters. Note that the \emph{code interpolation} will wrap the quotes around strings in the generated output.

\begin{lstlisting}
// Haxe code:
var myVar = untyped __js__('myObject.myJavaScriptFunction({0}, {1})', "Mark", 31);
\end{lstlisting}

This will generate the following JavaScript code:
\begin{lstlisting}
// JavaScript Code
var myVar = myObject.myJavaScriptFunction("Mark", 31);
\end{lstlisting}


\subsection{JavaScript target Metadata}
\label{target-javascript-metadata}

This is the list of JavaScript specific metadata. For more information, see also the complete list of all \tref{Haxe built-in metadata}{cr-metadata}.

\begin{center}
\begin{tabular}{| l | l |}
	\hline
	\multicolumn{2}{|c|}{JavaScript metadata} \\ \hline
	Metadata &  Description \\ \hline
	@:expose \_(?Name=Class path)\_  &  Makes the class available on the \expr{window} object or \expr{exports} for node.js  \\
	@:jsRequire  &  Generate javascript module require expression for given extern \\
	@:selfCall  &  Translates method calls into calling object directly \\
\end{tabular}
\end{center}

\subsection{Exposing Haxe classes for JavaScript}
\label{target-javascript-expose}

It is possible to make Haxe classes or static fields available for usage in plain JavaScript. 
To expose, add the \expr{@:expose} metadata to the desired class or static fields.

This example exposes the Haxe class \ic{MyClass}.

\haxe{assets/ClassExpose.hx}

It generates the following JavaScript output:

\begin{lstlisting}
(function ($hx_exports) { "use strict";
var MyClass = $hx_exports.MyClass = function(name) {
	this.name = name;
};
MyClass.prototype = {
	foo: function() {
		return "Greetings from " + this.name + "!";
	}
};
})(typeof window != "undefined" ? window : exports);
\end{lstlisting}

By passing globals (like \ic{window} or \ic{exports}) as parameters to our anonymous function in the JavaScript module, it becomes available which allows to expose the Haxe generated module.

In plain JavaScript it is now possible to create an instance of the class and call its public functions.

\begin{lstlisting}
// JavaScript code
var instance = new MyClass('Mark');
console.log(instance.foo()); // logs a message in the console
\end{lstlisting}

The package path of the Haxe class will be completely exposed. To rename the class or define a different package for the exposed class, use \expr{@:expose("my.package.MyExternalClass")}

\paragraph{Shallow expose}

When the code generated by Haxe is part of a larger JavaScript project and wrapped in a large closure it is not always necessary to expose the Haxe types to global variables.
Compiling the project using \ic{-D shallow-expose} allows the types or static fields to be available for the surrounding scope of the generated closure only.

When the code is compiled using \ic{-D shallow-expose}, the generated output will look like this:

\begin{lstlisting}
var $hx_exports = $hx_exports || {};
(function () { "use strict";
var MyClass = $hx_exports.MyClass = function(name) {
	this.name = name;
};
MyClass.prototype = {
	foo: function() {
		return "Greetings from " + this.name + "!";
	}
};
})();
var MyClass = $hx_exports.MyClass;
\end{lstlisting}

In this pattern, a var statement is used to expose the module; it doesn't write to the \ic{window} or \ic{exports} object. 

\subsection{Loading extern classes using "require" function}
\label{target-javascript-require}
\since{3.2.0}

Modern \target{JavaScript} platforms, such as Node.js provide a way of loading objects
from external modules using the "require" function. Haxe supports automatic generation
of "require" statements for \expr{extern} classes.

This feature can be enabled by specifying \expr{@:jsRequire} metadata for the extern class. If our \expr{extern} class represents a whole module, we pass a single argument to the \expr{@:jsRequire} metadata specifying the name of the module to load:

\haxe{assets/JSRequireModule.hx}

In case our \expr{extern} class represents an object within a module, second \expr{@:jsRequire} argument specifies an object to load from a module:

\haxe{assets/JSRequireObject.hx}

The second argument is a dotted-path, so we can load sub-objects in any hierarchy.

If we need to load custom JavaScript objects in runtime, a \expr{js.Lib.require} function can be used. It takes \expr{String} as its only argument and returns \expr{Dynamic}, so it is advised to be assigned to a strictly typed variable.

\section{Flash}
\label{target-flash}
\state{NoContent}

\subsection{Getting started with Haxe/Flash}
\label{target-flash-getting-started}

Developing Flash applications is really easy with Haxe. Let's look at our first code sample.
This is a basic example showing most of the toolchain. 

Create a new folder and save this class as \ic{Main.hx}.

\begin{lstlisting}
import flash.Lib;
import flash.display.Shape;
class Main {
    static function main() {
        var stage = Lib.current.stage;
        
        // create a center aligned rounded gray square
        var shape = new Shape();
        shape.graphics.beginFill(0x333333);
		shape.graphics.drawRoundRect(0, 0, 100, 100, 10);
		shape.x = (stage.stageWidth - 100) / 2;
		shape.y = (stage.stageHeight - 100) / 2;
		
		stage.addChild(shape);
    }    
}
\end{lstlisting}

To compile this, either run the following from the command line:

\begin{lstlisting}
haxe -swf main-flash.swf -main Main -swf-version 15 -swf-header 960:640:60:f68712
\end{lstlisting}

Another possibility is to create and run (double-click) a file called \ic{compile.hxml}. In this example the hxml file should be in the same directory as the example class.

\begin{lstlisting}
-swf main-flash.swf
-main Main
-swf-version 15
-swf-header 960:640:60:f68712
\end{lstlisting}

The output will be a main-flash.swf with size 960x640 pixels at 60 FPS with an orange background color and a gray square in the center.

\paragraph{Display the Flash}

Run the SWF standalone using the \href{https://www.adobe.com/support/flashplayer/downloads.html}{Standalone Debugger FlashPlayer}. 

To display the output in a browser using the Flash plugin, create an HTML-document called \ic{index.html} and open it.

\begin{lstlisting}
<!DOCTYPE html>
<html>
	<body>
		<embed src="main-flash.swf" width="960" height="640">
	</body>
</html>
\end{lstlisting}

\paragraph{More information}

\begin{itemize}
	\item \href{https://api.haxe.org/flash/}{Haxe/Flash API docs}
	\item \href{http://help.adobe.com/en_US/FlashPlatform/reference/actionscript/3/}{Adobe Livedocs}
\end{itemize}

\subsection{Embedding resources}
\label{target-flash-resources}

Embedding resources is different in Haxe compared to ActionScript 3. Instead of using \ic{\[embed\]} (AS3-metadata) use \tref{Flash specific compiler metadata}{target-flash-metadata} like \ic{@:bitmap}, \ic{@:font}, \ic{@:sound} or \ic{@:file}.

\begin{lstlisting}
import flash.Lib;
import flash.display.BitmapData;
import flash.display.Bitmap;

class Main {
  public static function main() {
    var img = new Bitmap( new MyBitmapData(0, 0) );
    Lib.current.addChild(img);
  }
}

@:bitmap("relative/path/to/myfile.png") 
class MyBitmapData extends BitmapData { }
\end{lstlisting}

\subsection{Using external Flash libraries}
\label{target-flash-external-libraries}

To embed external \ic{.swf} or \ic{.swc} libraries, use the following \href{https://haxe.org/documentation/introduction/compiler-usage.html}{compilation options}:

\begin{description}
	\item[\expr{-swf-lib <file>}] Embeds the SWF library in the compiled SWF.
	\item[\expr{-swf-lib-extern <file>}] Adds the SWF library for type checking but doesn't include it in the compiled SWF.
\end{description}

The standard compilation options also provide more Haxe sources to be added to the project:

\begin{itemize}
	\item To add another class path use \expr{-cp <directory>}.
	\item To add a \tref{Haxelib library}{haxelib} use \expr{-lib <library-name>}.
	\item To force a whole package to be included in the project, use \expr{--macro include('mypackage')} which will include all the classes declared in the given package and subpackages. 
\end{itemize}

\subsection{Flash target Metadata}
\label{target-flash-metadata}

This is the list of Flash specific metadata. For a complete list see \tref{Haxe built-in metadata}{cr-metadata}.

\begin{center}
\begin{tabular}{| l | l |}
	\hline
	\multicolumn{2}{|c|}{Flash metadata} \\ \hline
	Metadata &  Description  \\ \hline
	@:bind  &  Override Swf class declaration \\
	@:bitmap \_(Bitmap file path)\_  &  \_Embeds given bitmap data into the class (must extend \expr{flash.display.BitmapData}) \\
	@:debug  &  Forces debug information to be generated into the Swf even without \expr{-debug} \\
	@:file(File path)  &  Includes a given binary file into the target Swf and associates it with the class (must extend \expr{flash.utils.ByteArray}) \\
	@:font \_(TTF path Range String)\_  &  Embeds the given TrueType font into the class (must extend \expr{flash.text.Font}) \\
	@:getter \_(Class field name)\_  &  Generates a native getter function on the given field  \\
	@:noDebug &  Does not generate debug information into the Swf even if \expr{-debug} is set \\
	@:ns  &  Internally used by the Swf generator to handle namespaces \\
	@:setter \_(Class field name)\_  &  Generates a native setter function on the given field \\
	@:sound \_(File path)\_  &  Includes a given \_.wav\_ or \_.mp3\_ file into the target Swf and associates it with the class (must extend \expr{flash.media.Sound}) \\
\end{tabular}
\end{center}

\section{Neko}
\label{target-neko}

\section{PHP}
\label{target-php}
\state{NoContent}

\subsection{Getting started with Haxe/PHP}
\label{target-php-getting-started}

To get started with Haxe/PHP, create a new folder and save this class as \ic{Main.hx}.

\haxe{assets/HelloPHP.hx}

To compile, either run the following from the command line:

\begin{lstlisting}
haxe -php bin -main Main
\end{lstlisting}

Another possibility is to create and run (double-click) a file called \ic{compile.hxml}. In this example the hxml file should be in the same directory as the example class.

\begin{lstlisting}
-php bin
-main Main
\end{lstlisting}

The compiler outputs in the given \emph{bin}-folder, which contains the generated PHP classes that prints the traced message when you run it. The generated PHP-code runs for version 5.1.0 and later.

\paragraph{More information}

\begin{itemize}
	\item \href{https://api.haxe.org/php/}{Haxe/PHP API docs}
	\item \href{http://php.net/docs.php}{PHP.net Documentation}
	\item \href{http://phptohaxe.haqteam.com/code.php}{PHP to Haxe tool}
\end{itemize}


\subsection{PHP untyped functions}
\label{target-php-untyped}

These functions allow to access specific PHP platform features. It works only when the Haxe compiler is targeting PHP and should always be prefixed with \expr{untyped}. 

\emph{Important note:} Before using these functions, make sure there is no alternative available in the Haxe Standard Library. The resulting syntax can not be validated by the Haxe compiler, which may result in invalid or error-prone code in the output.

\paragraph{\expr{untyped __php__(expr)}}
Injects raw PHP code expressions. It's possible to pass fields from Haxe source code using \tref{String Interpolation}{lf-string-interpolation}.

\begin{lstlisting}
var value:String = "test";
untyped __php__("echo '<pre>'; print_r($value); echo '</pre>';");
// output: echo '<pre>'; print_r('test'); echo '</pre>';
\end{lstlisting}

\paragraph{\expr{untyped __call__(function, arg, arg, arg...)}}
Calls a PHP function with the desired number of arguments and returns what the PHP function returns.

\begin{lstlisting}
var value = untyped __call__("array", 1,2,3); 
// output returns a NativeArray with values [1,2,3]
\end{lstlisting}

\paragraph{\expr{untyped __var__(global, paramName)}}
Get the values from global vars. Note that the dollar sign in the Haxe code is omitted.

\begin{lstlisting}
var value : String = untyped __var__('_SERVER', 'REQUEST_METHOD')  
// output: $value = $_SERVER['REQUEST_METHOD']
\end{lstlisting}

\paragraph{\expr{untyped __physeq__(val1, val2)}}
Strict equals test between the two values. Returns a \type{Bool}.

\begin{lstlisting}
var isFalse = untyped __physeq__(false, value);
// output: $isFalse = false === $value;
\end{lstlisting}


\section{C++}
\label{target-cpp}
\state{NoContent}

\subsection{Getting started with Haxe/C++}
\label{target-cpp-getting-started}
The c++ target uses various c++ compilers, which are assumed to be already installed on the system, to create native executables or libraries.  The compilation happens in two phases.  Firstly, the Haxe compiler generates source, header and build files in an output directory.  Secondly, the "hxcpp" \tref{Haxelib library}{haxelib} is invoked to run the system compilers and linkers required to generate the ultimate result.

\paragraph{Prerequisites}
Before you can use the C++ target, you need in install
\begin{itemize}
	\item hxcpp, e.g. \ic{haxelib install hxcpp}.
	\item A system or cross-compiler
\end{itemize}

\paragraph{System Compilers}
System compilers are supported on the three primary operating systems - Mac, Linux, and Windows.  On Mac, it is recommended that you install the latest Xcode from the Mac App Store.  On Linux, it is recommended that you use the system package manager to install the compilers and on Windows, Microsoft Visual Studio is recommended.  On Windows, you can also use gcc-based compilers.  A minimal distribution is included in a \tref{Haxelib library}{haxelib}, and can be installed with \ic{haxelib install minimingw}.

\paragraph{Cross Compilers}
Hxcpp can be used to compile for non-host architectures if you have a suitable cross-compiler installed.  The compilers are usually supplied in the form of a Software Development Kits (SDK), or in the case of iOS devices, come with the system compiler (Xcode).  Selecting which compiler to use is achieved by defining particular variables in the hxcpp build environment.  Note that the hxcpp build tool is only responsible for producing a native executable or a native library (static or dynamic), not the complete bundling and packaging of assets and meta-data that is typically required for mobile devices.  Additional Haxe libraries can be used for this task.

\begin{itemize}
	\item \href{https://api.haxe.org/cpp/}{Haxe/C++ API docs}
\end{itemize}

\subsection{The Hxcpp Build Environment}
\label{target-cpp-build-environment}
The hxcpp build environment can be controlled by setting, or defining, variables in key-value pairs. Defining a key without a particular value is usually enough to trigger the desired behaviour, and these keys are often referred to as "defines".

\paragraph{Defining From the Command Line}
The easiest way to change the hxcpp build environment is to pass the defines though the Haxe command line using \ic{-D}.  Key-value pairs can also be passed this way, e.g.:

\ic{ haxe -main Main -cpp cpp -D android -D static_link  -D PLATFORM=android-9 }

Here, \ic{android} is defined - this causes hxcpp to cross-compile to android, \ic{static_link} is defined, which causes a static (rather than dynamic) library to be created and the android target platform is set to a particular value.  This platform information is passed on to the SDK so it can generate the appropriate code.

Advanced users can add additional defines to the system at compile time using macros.  These definitions will also be passed on to the hxcpp build tool.

\paragraph{Defining From the System Environment Variables}
The hxcpp build tool will import all the system environment variables, so you can configure the processes using the system like:

\ic{ setenv HXCPP_VERBOSE }

If you are running Haxe though an IDE, some care must be taken with environment variables since the variables may be read once from the environment in which the IDE was started, rather than changing when the variables are changed on the system.

\paragraph{Defining From .hxcpp_config.xml}
The hxcpp build tool parses several "build files".  These files are in a basic xml file format, and can be used to set, or conditionally set, configuration variables.  As part of the build process, the \ic{.hxcpp_config.xml} file, known as the configuration file, will be read (twice).  This file is located in the user's home directory (or user's profile directory on windows) and is the best place to configure variables that are specific to the system that rarely change, such as the location of the cross-compiler SDKs.  A placeholder file will be generated the first time hxcpp is run by a user on the machine.  You can see the exact location of this file in the build log if you compile with the \ic{HXCPP_VERBOSE} define.

The configuration file is read twice.  The first time the \ic{"vars"} section is read early in the configuration process, and is used to set up and configure the location of the compilers.  The second time, the \ic{"exes"} section is read near the end of the processes and allows modification of the compiler or build process based on all the available information.  See \Fullref{target-cpp-file-format} for details on the file format.

\paragraph{Defining Via @:buildXml Metadata}
Configuration data can be injected into the "build.xml" file that is created by Haxe.  This is done attaching metadata to a Haxe class that is directly or indirectly included in the build.  For example:
\begin{lstlisting}
@:buildXml("
<target id='haxe'>
  <lib name='${haxelib:nme}/lib/${BINDIR}/libnme${LIBEXTRA}${LIBEXT}'/>
</target>
<include name='${haxelib:nme}/lib/NmeLink.xml'/>
")
@:keep class StaticNme
{
  ...
\end{lstlisting}

This metadata is best for adding libraries or include paths to the build.
Some notes:
\begin{itemize}
	\item The @:keep metadata is added to prevent dead-code-elimination from removing this class
	\item Quoting can be a bit tricky - here the double-quotes are used for Haxe, and the single quotes are added to the build.xml.
	\item Injecting a single "include" command is a good way to manage the quoting issue.
	\item Knowledge of the build system is required to get this right. See \Fullref{target-cpp-file-format} for details on the file format.
	\item The build.xml file is read after the choice of compiler has been made, so it is generally not suitable for configuring the compiler.
\end{itemize}


\subsection{build.xml File Format}
\label{target-cpp-file-format}
The build.xml file format is designed to address the very specific goal of generating compiled output as fast as possible on a wide variety of compilers.  Therefore, at its core, it is simply a list of files to compile, and their dependencies, and a list of compiler exes and flags required to get the job done.  Conditional variable setting is added to support the great variety of compilers and use cases encountered by hxcpp.  However, much of the logic required to orchestrate the compiler sequences, such as precompiling headers, is defined in Haxe and implemented in the build tool itself, rather than being scriptable.

A hxcpp build file consists of a series of commands and data structures in xml format.  The commands are executed immediately as they are encountered, while the data structures are accumulated and used after all the build files have been parsed.

\paragraph{Running a build.xml File With Hxcpp}
When you compile a Haxe program with hxcpp, the Haxe compiler will normally run the hxcpp build tool on the generated build.xml file automatically.  You can, however, prevent this by adding \ic{-D no-compilation} to the Haxe command line.

You can run the hxcpp build tool on your own build files using

\ic{haxelib run hxcpp myfile.xml [-Ddefine] [-Dkey=value]}

Note the lack of space between the "-D" and the variable name.

A minimal build.xml file consists of an xml container and a dummy default target, like:
\begin{lstlisting}
<xml>
   <echo value="Hello!" />
   <target id="default" />
</xml>
\end{lstlisting}

\paragraph{Conditions, Substitutions and Sections}
Most elements in the hxcpp xml file allow a common syntax for dynamic configuration.

Xml elements can contain conditional "if" and/or "unless" values.  These conditions are evaluated at parse time and the entire element will be skipped if the condition fails.  For example adding this lines:
\begin{lstlisting}
<xml>
  <echo value="Hello A" if="A" />
  <echo value="Hello A && B" if="A B" />
  <echo value="Hello A || B" if="A || B" />
  <echo value="Hello !A" unless="A" />
  <echo value="Hello !(A && B)" unless="A B" />
  <echo value="Hello !(A || B)" unless="A || B" />
  <echo value="Never Never" if="A" unless="A" />
  <target id="default" />
</xml>
\end{lstlisting}

and running:

\ic{haxelib run hxcpp myfile.xml -DA}

shows how the logic depends on whether or not A or B has been defined.

Sections can be used to group commands together based on a common condition.  They can also be used to include only part of another xml file, but this technique is currently only used when parsing the .hxcpp_config.xml file.  For example:

\begin{lstlisting}
  <section if="C" >
    <echo value="I See" />
    <echo value="You" />
  </section>
\end{lstlisting}

The xml attribute values can be substituted with variable values using dollars-brace syntax.  Using a colon allows a function-call to be substituted, e.g.:

\begin{lstlisting}
  <echo value="some var = ${SOME_VAR}" />
  <echo value="${haxelib:hxcpp}/include" />
\end{lstlisting}

\subsection{Using C++ Defines}
\label{target-cpp-defines}
\begin{itemize}
	\item ANDROID_HOST
	\item ANDROID_NDK_DIR
	\item ANDROID_NDK_ROOT
	\item BINDIR
	\item DEVELOPER_DIR
	\item HXCPP
	\item HXCPP_32
	\item HXCPP_COMPILE_CACHE
	\item HXCPP_COMPILE_THREADS
	\item HXCPP_CONFIG
	\item HXCPP_CYGWIN
	\item HXCPP_DEPENDS_OK
	\item HXCPP_EXIT_ON_ERROR
	\item HXCPP_FORCE_PDB_SERVER
	\item HXCPP_M32
	\item HXCPP_M64
	\item HXCPP_ARMV6
	\item HXCPP_ARMV7
	\item HXCPP_ARMV7S
	\item HXCPP_ARM64
	\item HXCPP_X86
	\item HXCPP_MINGW
	\item HXCPP_MSVC
	\item HXCPP_MSVC_CUSTOM
	\item HXCPP_MSVC_VER
	\item HXCPP_NO_COLOR
	\item HXCPP_NO_COLOUR
	\item HXCPP_VERBOSE
	\item HXCPP_WINXP_COMPAT
	\item HXCPP_API_LEVEL
	\item HXCPP_DEBUG_LINK
	\item DHXCPP_STACK_TRACE
	\item HXCPP_STACK_LINE
	\item HXCPP_CHECK_POINTER
	\item HXCPP_DEBUGGER
	\item HXCPP_CPP11
	\item HXCPP_STRICT_CASTS
	\item HXCPP_VISIT_ALLOCS
	\item HXCPP_WINXP_COMPAT
	\item HXCPP_OPTIMIZE_FOR_SIZE
	\item HXCPP_GC_MOVING
	\item IPHONE_VER
	\item LEGACY_MACOSX_SDK
	\item LEGACY_XCODE_LOCATION
	\item MACOSX_VER
	\item MSVC_VER
	\item NDKV
	\item NO_AUTO_MSVC
	\item PLATFORM
	\item QNX_HOST
	\item QNX_TARGET
	\item TOOLCHAIN_VERSION
	\item USE_GCC_FILETYPES
	\item USE_PRECOMPILED_HEADERS
	\item android
	\item apple
	\item blackberry
	\item cygwin
	\item dll_import
	\item dll_import_include
	\item dll_import_link
	\item emcc
	\item emscripten
	\item gph
	\item hardfp
	\item haxe_ver
	\item ios
	\item iphone
	\item iphoneos
	\item iphonesim
	\item linux
	\item linux_host
	\item mac_host
	\item macos
	\item mingw
	\item rpi
	\item simulator
	\item tizen
	\item toolchain
	\item webos
	\item windows
	\item windows_host
	\item winrt
	\item xcompile
\end{itemize}

\subsection{Using C++ Pointers}
\label{target-cpp-pointers}


\section{Cppia}
\label{target-cppia}


\subsection{Getting started with Haxe/Cppia}
\label{target-java-getting-started}

Cppia is a scriptable cpp subtarget for Haxe. A cppia script is an instructions assembly script that can be run inside a cppia host and gives you runtime speed at near-zero compilation time. It also lets add performance critical code to the host, wich gives you full cpp runtime speed for those parts.

\paragraph{More information}
\begin{itemize}
	\item \href{https://api.haxe.org/cpp/cppia/index.html}{Cppia API documentation}
	\item \href{https://code.haxe.org/category/other/working-with-cppia/index.html}{Cppia tutorial}
\end{itemize}

\section{Java}
\label{target-java}
\state{NoContent}

\subsection{Getting started with Haxe/Java}
\label{target-java-getting-started}

To get started with Haxe/Java, create a new folder and save this class as \ic{Main.hx}.

\haxe{assets/HelloWorld.hx}

To compile Haxe to Java we need two obvious prerequisites installed:

\begin{itemize}
	\item \href{http://lib.haxe.org/p/hxjava}{hxjava haxelib}. This is the support library for the Java backend of the Haxe compiler.
	\item \href{https://java.com/download/}{JRE - Java Runtime Environment}.
\end{itemize}

Run the following from the command line:

\begin{lstlisting}
haxe -java bin -main Main
\end{lstlisting}

Another possibility is to create and run (double-click) a file called \ic{compile.hxml}. In this example the hxml-file should be in the same directory as the example class.

\begin{lstlisting}
-java bin
-main Main
\end{lstlisting}

The compiler outputs in the given \emph{bin}-folder, which contains the generated sources / .jar files which prints the traced message when you execute it. 

To execute, run the following command:

\begin{lstlisting}
java -jar bin/Main.jar
\end{lstlisting}

\paragraph{More information}

\begin{itemize}
	\item \href{https://api.haxe.org/java/}{Haxe/Java API docs}
	\item \href{https://docs.oracle.com/javase/}{Java Platform Documentation}
\end{itemize}


\section{C\#}
\label{target-cs}
\state{NoContent}

\subsection{Getting started with Haxe/C#}
\label{target-cs-getting-started}

Haxe can be used as a language for .NET platform through its C# target. Let's make a simple program using .NET Console class:

\haxe{assets/HelloWorld.hx}

To compile Haxe to C# we need two obvious prerequisites installed:

\begin{itemize}
	\item \href{http://lib.haxe.org/p/hxcs}{hxcs haxelib}. This is the support library for the C# backend of the Haxe compiler.
	\item \href{https://www.microsoft.com/net}{.NET development framework (either Microsoft.NET or Mono)}
\end{itemize}

After that we can compile to C# using the \ic{-cs} option from either the command line or an hxml-file:

\begin{lstlisting}
haxe -cs out -main Main
\end{lstlisting}

The compiler will output C# sources into  \emph{out/src} folder, then call C# compiler to build  \emph{Main.exe} file into  \emph{out/bin} folder.

\paragraph{More information}

\begin{itemize}
	\item \href{https://api.haxe.org/cs/}{Haxe/C# API docs}
	\item \href{https://msdn.microsoft.com/en-us/library/kx37x362.aspx}{C# Documentation}
\end{itemize}


\subsection{.NET version and external libraries}
\label{target-cs-external-libraries}

By default, Haxe uses basic .NET 2.0 API provided by hxcs library (it ships mscorlib.dll and System.dll from the Mono project). We can specify different .NET version by providing \ic{-D net-ver=xx} define, where xx is major and minor digits of .NET version number, i.e. \ic{-D net-ver=40} for setting .NET version to 4.0. Note that currently, hxcs library only ships DLL files for .NET 2.0 and 4.0.

\paragraph{Using custom .NET distribution}

We can make Haxe use a custom set of DLL files as standard .NET framework. To do that, we need to first learn about how Haxe finds standard .NET libraries. Haxe/C# looks for .DLL files in a directory path, constructed from three components:

\begin{itemize}
	\item .NET version (set by \ic{-D net-ver=xx}, defaults to \ic{20} as described above)
	\item .NET target (by default set to \ic{net}, but could be changed using \ic{-D net-target=xxx}, where \ic{xxx} could be \ic{micro}, \ic{compact} or some other).
	\item .NET std path (set by -net-std option, by default points to netlib directory inside hxcs library)
\end{itemize}

The resulting search path will be \emph{<net_std_path>/<net_target>-<net_ver>/}, taking in the consideration default values described above, without any specific configuration haxe will load all .NET DLL files found in \emph{<hxcs_install_path>/netlib/net-20/}.

Now if we provide the following options:

\begin{lstlisting}
-D net-target=micro -D net-ver=35 -net-std=/dotnet
\end{lstlisting}

Haxe will load all .NET DLL files found in \emph{/dotnet/micro-35/}.


\paragraph{Using external libraries}

Haxe can directly load .NET assembly files (.DLL) and convert its type definitions for use as Haxe types. To load a .NET assembly, use \ic{-net-lib library.dll} compiler option. Haxe will then automatically parse types defined in that assembly file and make them available for import as Haxe types.

Some changes are performed to type naming of C# classes to make them fit into Haxe type system, namely:

\begin{itemize}
	\item Namespaces are lowercased to follow Haxe package naming rules, so i.e. \ic{UnityEngine} becomes \ic{unityengine} (note that \ic{System} namespace is also prefixed with \ic{cs}, so  \ic{System.Core} becomes  \ic{cs.system.core})
	\item Inner classes are generated as \ic{OuterClassName_InnerClassName} and placed into the \ic{OuterClassName} module. So for example for an inner class \ic{B} inside a class \ic{A} inside a namespace \ic{Something}, the full Haxe type path will be \ic{something.A.A_B}. Note however, that if you do \ic{import something.A}, both \ic{A} and \ic{A_B} class will be available within your module as per standard Haxe import mechanism.
	\item Classes with type parameters have numbers of type params appended to their name, for example \ic{Dictionary<K,V>} becomes \ic{Dictionary_2<K,V>}
\end{itemize}


\subsection{Haxe/C# Defines}
\label{target-cs-defines}

Besides \ic{-D net-ver} and \ic{-D net-target}:

\begin{itemize}
	\item \ic{-D dll} compile to a .NET assembly instead of an executable file. Added automatically when no \ic{-main} is specified.
	\item \ic{-D real-position} don't generate #line directives that map C# expression positions to original .hx files. Useful for tracking down issues related to code generation.
	\item \ic{-D no-root} generate package-less haxe types in the haxe.root namespace to avoid conflicts with other types in the root namespace
	\item \ic{-D erase-generics} fully erase type parameters from generated C# files and generate non-generic classes. This is useful in some cases, like working with .NET Micro Framework or preventing generics-related issues with Unity3D AOT compiler.
	\item \ic{-D no-compilation} only generate C# sources and don't invoke C# compiler on them.
	\item \ic{-D keep-old-output} by default haxe cleans up stale generated source files from the output directory. This define disables that behaviour.
	\item \ic{-D dll-import}
\end{itemize}
	
Haxe automatically adds \ic{NET_xx} defines where xx is major and minor version numbers .NET versions up to selected one. For example, when using .NET 4.0 (by providing \ic{-D net-ver=40}), we have the following defines set automatically: \ic{NET_20}, \ic{NET_21}, \ic{NET_30}, \ic{ NET_35} and \ic{NET_40}. If we had \ic{-D net-ver=30}, we would only have \ic{NET_20}, \ic{NET_21} and \ic{NET_30}.

\subsection{Haxe/C# Metadata}
\label{target-cs-metadata}

This is the list of C# specific metadata. For more information, see also the complete list of all \tref{Haxe built-in metadata}{cr-metadata}.

\begin{center}
\begin{tabular}{| l | l | l |}
	\hline
	\multicolumn{3}{|c|}{JavaScript metadata} \\ \hline
	Metadata & Usage & Description \\ \hline
	@:nativeGen  &  on classes & don't generate reflection, generate proper type parameters. This is useful for some sort of interop, but slows down reflection and structural typing \\
	@:nativeGen  &  on "flat" enums & generate C# enum, but note that C# enums are not-nullable unlike haxe enums, so using null will be generated as a default enum value (0-indexed constructor). \\
	@:property  &  on non-physical fields (those with get/set/never accessors) & generate native C# properties. useful for implementing extern interfaces or providing API for use from C# \\
	@:event  &  on variables & generate an event delegate (this also requires pairing add_EventName, remove_EventName methods with relevant signatures \\
	@:protected  &  on a field & mark field as protected instead of public (could affect reflection, but useful for hiding fields when providing API for use from outside Haxe) \\
	@:struct  &  on classes  &  generate struct instead of class \\
\end{tabular}
\end{center}

\subsection{Injecting raw C# code}
\label{target-cs-code-injection}

\paragraph{Expression injection}

In some cases it may be needed to inject raw C# code into Haxe-generated code. This is possible by using  \expr{untyped __cs__} call, for example:

\begin{lstlisting}
public function isBool(v:Dynamic):Bool {
    return untyped __cs__("v is bool");
}
\end{lstlisting}
The  \expr{untyped __cs__} syntax also supports code interpolation which means that you can insert Haxe expressions into injected C# code. For example, the example above could have been made inline, but because it always generates  \expr{v is bool}, it won't work when the given argument is not named  \expr{v} in the calling scope. To deal with that, we could rewrite our function using code interpolation, as follows:

\begin{lstlisting}
public inline function isBool(v:Dynamic):Bool {
    return untyped __cs__("{0} is bool", v);
}
\end{lstlisting}

\paragraph{Class code injection}

TODO: @:classCode

\paragraph{Function code injection}

We can use \expr{@:functionCode} metadata for a method to generate raw C# code inside a method body. It completely replaces any haxe expressions in method body. For example:

\begin{lstlisting}
@:functionCode("return (v is int);")
function isInt(v:Dynamic):Bool {
    return false;
}
\end{lstlisting}

Which will generate:

\begin{lstlisting}
public virtual bool isInt(object v) {
    return (v is int);
}
\end{lstlisting}


\section{Python}
\label{target-python}



\section{Lua}
\label{target-lua}
\state{NoContent}

\subsection{Getting started with Haxe/Lua}
\label{target-lua-getting-started}

To get started with Haxe for Lua, it's necessary to pick a Lua version and install
dependencies.  All versions of Lua are supported, but may require different
libraries.  Lua 5.1, 5.2, 5.3, and LuaJIT 2.0 and 2.1 (beta) are supported.

Lua is a very lightweight language that ships with a much smaller  feature set
than Haxe.  In some cases (e.g. regex), it's necessary to install supplementary
libraries that are used to support basic Haxe functionality.

In order to cover all dependencies, it is recommended to install and use
\href{https://github.com/luarocks/luarocks/wiki/Download}{LuaRocks}.  However,
if you do not utilize relevant behavior (e.g. regex) on a given platform,
or if you are using an embedded Lua client, then it is not necessary to
install any packages for basic language functionality.

With LuaRocks, install the following packages:

\begin{lstlisting}
luarocks install lrexlib-pcre
luarocks install environ
luarocks install luasocket
luarocks install luv
\end{lstlisting}

On Lua 5.1, install the bitops library:
\begin{lstlisting}
luarocks install luabitop
\end{lstlisting}

On Lua 5.3, install the bit32 library instead:
\begin{lstlisting}
luarocks install bit32
\end{lstlisting}

When developing for multiple Lua versions, it is recommended to use
the Python package \href{https://github.com/mpeterv/hererocks}{hererocks}.

With Lua installed, it is possible to write a simple command line application.

Create a new folder, and save this class as \ic{Main.hx}.

\begin{lstlisting}
class Main {
    static function main() {
        trace("hello world");
    }
}
\end{lstlisting}

To compile, run the following:
\begin{lstlisting}
haxe -lua out.lua -main Main
\end{lstlisting}

\paragraph{More information}
\begin{itemize}
	\item \href{https://www.lua.org/}{Lua Homepage}
	\item \href{http://luajit.org/}{LuaJIT Homepage}
\end{itemize}

\subsection{Using external Lua libraries}
\label{target-lua-external-libraries}

The \tref{extern functionality}{lf-externs} in Haxe provides a way of declaring type signatures
for native Lua libraries.

\subsection{Version flags}
\label{target-lua-flags}

The Lua target enables the following define flags for the Haxe complier:

\begin{description}
	\item[\expr{-D lua_ver}] Enable special features for a specific Lua version. Currently, this flag will enable extern methods that are specific to certain versions (e.g. table.pack in Lua > 5.2).
	\item[\expr{-D luajit}] Enable special features for LuaJIT.  Currently this flag will enable the \ic{jit} and \ic{ffi} module namespaces.
\end{description}

\subsection{Multireturns}
\label{target-lua-multireturns}

Lua allows for multiple values to be returned from a given function.  Haxe
does not support this by default, but can allow extern definitions to reference
multireturn values through the \ic{@:multiReturn} metadata.

\begin{lstlisting}
class Main {
    static function main() {
        var strfind = NativeString.find("foobar", "bar");
        trace(strfind.begin);
        trace(strfind.end);
    }
}

@:native("string")
extern class NativeString {
	public static function find(str : String, target : String): StringFind;
}

@:multiReturn extern class StringFind {
	var begin : Int;
	var end : Int;
}
\end{lstlisting}

This example has three parts:

\begin{itemize}
	\item The extern class \ic{NativeString} which is an extern for the base \ic{string} library in Lua.
	\item The \ic{StringFind} class which is marked as \ic{@:multiReturn} that describes the return values.
	\item The Main class that invokes the string method as a simple example.
\end{itemize}

The multireturn behavior in Haxe is optimized based on usage.  If fields are
only accessed directly, the Haxe compiler will allocate the multireturn to
individual variables.  But, if you pass or assign the entire multireturn value,
the compiler will wrap all values into a table object.  This operation ensures
that multireturn variable handling only carries as much overhead as needed.



\section{HashLink}
\label{target-hl}

\chapter{Debugging}
\label{debugging}
\state{NoContent}

\section{Logging and Trace}
\label{debugging-trace-log}

Haxe provides developers with a powerful logging/trace system. Simply call \expr{trace} within functions:

\begin{lstlisting}
trace("Hello world!");
\end{lstlisting}

In most Haxe targets trace will be printed to stdout. JavaScript uses \ic{console.log}. Each trace is displayed with the filename and line number information where the trace occurred:

\lang{none}\begin{lstlisting}
Test.hx:11: Hello world!
\end{lstlisting}

To trace without the default position information \ic{haxe.Log.trace(msg, null)} can be used.

\paragraph{Custom trace}

The trace can have a custom output by changing the \expr{Log.trace} method where all trace calls are redirected. 

\haxe{assets/CustomTrace.hx}

The \ic{v} argument is the first parameter of the trace call. It can be a \expr{String} or any other value. The optional \ic{infos} argument contains extra position parameter, see below.

The \expr{infos.customParams} array contains all extra arguments that were given to the original trace. If no extra parameters are passed, it will be \expr{null}. 

As illustration, the previous example will be compiled as if it was calling the following:

\lang{js}\begin{lstlisting}
haxe.Log.trace("hello", {
	fileName : "Test.hx", 
	lineNumber : 6, 
	className : "Test", 
	methodName : "main", 
	customParams : ["warning",123]
});
\end{lstlisting}

\paragraph{Removing traces}

You can simply remove all trace informations by compiling your project with \ic{--no-traces} argument. This will remove all trace calls as if they were not present in the program.

\section{Position Information Parameter}
\label{debugging-posinfos}

\href{http://api.haxe.org/haxe/PosInfos.html}{\expr{haxe.PosInfos}} is a magic type which can be used to generate position information into the output for debugging use.
If a function has a final optional argument of this type, i.e. \expr{(..., ?pos:haxe.PosInfos)}, each call to that function which does not assign a value to that argument has its position added as call argument. 

It is sometimes useful to define a custom method that does some traces in some case. The following usage is possible since in Haxe when the \expr{haxe.PosInfos} optional parameter is not set, its default value will always be replaced by the compiler:

\haxe{assets/AssertTrace.hx}

\section{Tracing Types}
\label{debugging-type-function}

\expr{\$type} is a \emph{compile-time} mechanism being called like a function, with a single argument. The compiler evaluates the argument expression and then outputs the type of that expression.

This is useful to evaluate if an expression has a certain type, mostly when dealing with \tref{Type inference}{type-system-type-inference}, which leaves the definition of the type up to the compiler.

\begin{lstlisting}
var myValue = "foo";
$type(myValue); // String
\end{lstlisting}


\section{Debugging in JavaScript}
\label{debugging-javascript}

\paragraph{Console}

Beside \tref{trace}{debugging-trace-log} Haxe exposes most of the browsers console functions, which can be accessed using \href{http://api.haxe.org/v/dev/js/html/Console.html}{\expr{js.Browser.console}}:

\begin{lstlisting}
js.Browser.console.log("Hello world"); 
js.Browser.console.info("Haxe is great!"); 
js.Browser.console.warn("Something could be wrong"); 
js.Browser.console.error("Something went wrong"); 
\end{lstlisting}


\begin{itemize}
	\item More info about the \href{https://developer.mozilla.org/en-US/docs/Web/API/Console}{browser console functions}.
\end{itemize}

\paragraph{Breakpoints}

In most browser developer tools breakpoints can be set to pause the code execution and start debugging. This mostly can be done by clicking near the line numbers. At each breakpoint JavaScript will stop executing and let the current values be inspected. After examining the values, the execution of code can be resumed (typically with a play button).

In JavaScript a developer can use the \ic{debugger} statement functionality to do the same from code.
In Haxe the same can be done with \href{http://api.haxe.org/v/js/Lib.html#debug}{\expr{js.Lib.debug}} function; this inserts a \ic{debugger} statement that will make a breakpoint if a debugger is available. If no debugging functionality is available, this statement has no effect. 

\begin{itemize}
	\item Read more on the \href{https://developer.mozilla.org/en/docs/Web/JavaScript/Reference/Statements/debugger}{debugger statement}.
\end{itemize}


\section{Source Maps}
\label{debugging-source-map}

Haxe is able to generate source maps, allowing debuggers to map from generated source back to the original Haxe source. This makes reading error stack traces, debugging with breakpoints, and profiling much easier.

When an error occurs, the developer is helped out by showing them where the error occurred in the original Haxe source file. This saves the developer time every single time they hit an error. The web console log messages also include links to the line that generated the log message, so these messages could link back to the original source lines as well. 

Compiling with the `-debug` flag will create a source map (.map) alongside the .js file. Source maps can also be generated for release builds with \ic{-D js-source-map}.


\subsection{Source Maps in JavaScript}
\label{debugging-source-map-javascript}

In the generated JavaScript the last line will have a reference to the source map that looks like this:

\begin{lstlisting}
//# sourceMappingURL=Main.js.map
\end{lstlisting}

To include the hx sources as part of the JS source map, compile with \ic{-D source-map-content}.

Make sure to enable 'JS source maps' in the browser developer tool settings. 

\begin{itemize}
	\item \href{https://developers.google.com/web/tools/chrome-devtools/debug/readability/source-maps}{Chrome source-maps}
	\item \href{https://developer.mozilla.org/en-US/docs/Tools/Debugger/How_to/Use_a_source_map}{Firefox source-maps}
	\item \href{https://developer.apple.com/library/safari/documentation/AppleApplications/Conceptual/Safari_Developer_Guide/ResourcesandtheDOM/ResourcesandtheDOM.html#//apple_ref/doc/uid/TP40007874-CH3-SW2}{Safari source-maps}
\end{itemize}

\subsection{Source Maps in Php7}
\label{debugging-sourcemap-php7}

Haxe source code positions in call stack and exception stack.

Since 3.4.1 Haxe can generate source maps for PHP target when compiling with \ic{-D php7} and \ic{-D source_map} flags.
Those source maps can be utilized by \href{https://lib.haxe.org/p/jstack/}{JStack} library to automatically transform \expr{haxe.CallStack.callStack()}, \expr{haxe.CallStack.exceptionStack()} and uncaught exceptions to make them point at Haxe sources instead of generated PHP files.


\begin{lstlisting}
class Main {
	static function main() {
		terribleError();
	}

	static function terribleError() {
		throw "Terrible error";
	}
}
\end{lstlisting}

Building it with flags:

\lang{hxml}\begin{lstlisting}
-main Main
-D php7
-php build
-debug
\end{lstlisting}

Running this build will trace the uncaught exception:

\lang{none}\begin{lstlisting}
$ php build/index.php
PHP Fatal error:  Uncaught php/_Boot/HxException: Terrible error in build/lib/Main.php:25
Stack trace:
#0 build/lib/Main.php(16): Main::terribleError()
#1 build/index.php(13): Main::main()
#2 {main}
  thrown in build/lib/Main.php on line 25
\end{lstlisting}

Install JStack using \ic{haxelib install jstack}. JStack automatically adds \ic{-D source_map} there is no need to add it manually.
Now if JStack is installed, add it to the compilation:

\lang{hxml}\begin{lstlisting}
-main Main
-D php7
-php build
-debug
-lib jstack
\end{lstlisting}

The output will have more informative stack trace for exceptions:

\lang{none}\begin{lstlisting}
$ php build/index.php
Terrible error
Called from Main.terribleError (src/Main.hx line 7)
Called from Main.main (src/Main.hx line 3)
Called from build/index.php line 13
\end{lstlisting}


\end{document}
