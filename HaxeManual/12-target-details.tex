\chapter{Target Details}
\label{target-details}
\state{NoContent}

\section{JavaScript}
\label{target-javascript}

Haxe provides the ability to target JavaScript. It does so by transpiling Haxe to JavaScript. The current implementation targets ECMAScript 5. But it is possible to target lower versions using \ic{-D js-es=<value>}.

When choosing the JavaScript target, only the used Haxe code of the project (and used parts of the standard library) are transpiled to JavaScript. This results in optimal filesize which is also readable. \tref{Source maps}{debugging-source-map-javascript} are available and there are several ways to get \tref{debug}{debugging} information. The standard library aims to have same functionality across all targets.

\paragraph{Usage}

You may want to compile Haxe to JavaScript in the following scenarios:

\emph{Client-side JavaScript}
Interacting with DOM elements. Haxe provides up to date typed interfaces to interact with the Document Object Model, allowing creation and update of DOM elements. 

Haxe can be used together with existing third-party libraries and frameworks, such as jQuery, React or Vue. To access third-party frameworks with a strongly-typed API, there are extern libraries available on \href{http://lib.haxe.org/t/js/}{Haxelib}. Alternatively, it is possible to create own externs (see \Fullref{target-javascript-external-libraries}) or use the dynamic type to access any framework, see \Fullref{target-javascript-injection}.

\emph{Creating graphics using Canvas and WebGL}
Use Haxe to create graphical elements on a web page using WebGL. 

Libraries like \href{http://www.openfl.org/}{OpenFL}, \href{http://heaps.io/}{Heaps} and \href{http://kha.tech/}{Kha} make use of WebGL as one of their backends.

\emph{Creating Haxe code that targets server-side JavaScript}
Working with server-side technology. Haxe can be used to create server-side JavaScript like Node.js.

\Fullref{target-javascript-getting-started}

\subsection{Getting started with Haxe/JavaScript}
\label{target-javascript-getting-started}

Haxe can be a powerful tool for developing JavaScript applications. Let's look at our first sample.
This is a very simple example showing the toolchain. 

Create a new folder and save this class as \ic{Main.hx}.

\begin{lstlisting}
import js.Browser;
class Main {
    static function main() {
        var button = Browser.document.createButtonElement();
        button.textContent = "Click me!";
        button.onclick = function(event) {
            Browser.alert("Haxe is great");
        }
        Browser.document.body.appendChild(button);
    }
}
\end{lstlisting}

To compile, either run the following from the command line:

\begin{lstlisting}
haxe -js main-javascript.js -main Main
\end{lstlisting}

Another possibility is to create and run (double-click) a file called \ic{compile.hxml}. In this example the hxml file should be in the same directory as the example class.

\begin{lstlisting}
-js main-javascript.js
-main Main
\end{lstlisting}

The output will be a main-javascript.js, which creates and adds a clickable button to the document body.

\paragraph{Run the JavaScript}

To display the output in a browser, create an HTML-document called \ic{index.html} and open it.

\begin{lstlisting}
<!DOCTYPE html>
<html>
	<body>
		<script src="main-javascript.js"></script>
	</body>
</html>
\end{lstlisting}

\paragraph{More information}

\begin{itemize}
	\item \Fullref{debugging-javascript}
	\item \href{https://code.haxe.org/category/javascript/}{Haxe/JavaScript tutorials}
	\item \href{http://api.haxe.org/js/}{Haxe JavaScript API docs}
	\item \href{https://developer.mozilla.org/en-US/docs/Web/JavaScript/Reference}{MDN JavaScript Reference}
\end{itemize}

\subsection{Using external JavaScript libraries}
\label{target-javascript-external-libraries}

The \tref{externs mechanism}{lf-externs} provides access to the native APIs in a type-safe manner. It assumes that the defined types exist at run-time but assumes nothing about how and where those types are defined. 

An example of an extern class is the \href{https://github.com/HaxeFoundation/haxe/blob/development/std/js/jquery/JQuery.hx}{jQuery class} of the Haxe Standard Library. 
To illustrate, here is a simplified version of this extern class:

\begin{lstlisting}
package js.jquery;
@:native("$") extern class JQuery {
	/**
		Creates DOM elements on the fly from the provided string of raw HTML.
		OR
		Accepts a string containing a CSS selector which is then used to match a set of elements.
		OR
		Binds a function to be executed when the DOM has finished loading.
	**/
	@:selfCall
	@:overload(function(element:js.html.Element):Void { })
	@:overload(function(selection:js.jquery.JQuery):Void { })
	@:overload(function(callback:haxe.Constraints.Function):Void { })
	@:overload(function(selector:String, ?context:haxe.extern.EitherType<js.html.Element, js.jquery.JQuery>):Void { })
	public function new():Void;

	/**
		Adds the specified class(es) to each element in the set of matched elements.
	**/
	@:overload(function(_function:Int -> String -> String):js.jquery.JQuery { })
	public function addClass(className:String):js.jquery.JQuery;

	/**
		Get the HTML contents of the first element in the set of matched elements.
		OR
		Set the HTML contents of each element in the set of matched elements.
	**/
	@:overload(function(htmlString:String):js.jquery.JQuery { })
	@:overload(function(_function:Int -> String -> String):js.jquery.JQuery { })
	public function html():String;
}
\end{lstlisting}

Note that functions can be overloaded to accept different types of arguments and return values, using the \expr{@:overload} metadata. Function overloading works only in externs.

Using this extern, we can use jQuery like this:

\begin{lstlisting}
import js.jquery.*;
..
new JQuery("#my-div").addClass("brand-success").html("haxe is great!");
..
\end{lstlisting}

The package and class name of the extern class should be the same as defined in the external library. If that is not the case, rewrite the path of a class using \expr{@:native}.

\begin{lstlisting}
package my.application.media;

@:native('external.library.media.video')
extern class Video {
..
\end{lstlisting}

Some JavaScript libraries favor instantiating classes without using the \expr{new} keyword. To prevent the Haxe compiler outputting the \expr{new} keyword when using a class, we can attach a \expr{@:selfCall} metadata to its constructor. For example, when we instantiate the jQuery extern class above, \expr{new JQuery()} will be outputted as \expr{\$()} instead of \expr{new \$()}. The \expr{@:selfCall} metadata can also be attached to a method. In this case, the method will be interpreted as a direct call to the object, illustrated as follows:

\begin{lstlisting}
extern class Functor {
	public function new():Void;
	@:selfCall function call():Void;
}

class Test {
	static function main() {
		var f = new Functor();
		f.call(); // will be outputted as `f();`
	}
}
\end{lstlisting}

Beside externs, \tref{Typedefs}{type-system-typedef} can be another great way to name (or alias) a JavaScript type. The major difference between typedefs and externs is that, typedefs are duck-typed but externs are not. Typedefs are suitable for common data structures, e.g. point (\expr{\{x:Float, y:Float\}}). Use of a point structure typedef for function arguments allows external JavaScript functions to accept point class instances from Haxe or from another JavaScript library. It is also useful for typing JSON objects.

The Haxe Standard Library comes with externs of \href{https://jquery.com/}{jQuery} and \href{http://blog.deconcept.com/swfobject/}{SWFObject}. Their version compatibility is summarized as follows:

\begin{center}
\begin{tabular}{| l | l | l |}
	\hline
	Haxe version & Library               & Externs location \\ \hline
	3.3          & jQuery 1.12.1 / 2.2.1 & <code>js.jquery.*</code> \\
	3.2-         & jQuery 1.6.4          & <code>js.JQuery</code> \\
	3.3          & SWFObject 2.3         & <code>js.swfobject.*</code> \\
	3.2-         & SWFObject 1.5         & <code>js.SWFObject</code> \\ \hline
\end{tabular}
\end{center}

There are many externs for other popular native libraries available on \tref{Haxelib library}{haxelib}. To view a list of them, check out the \href{http://lib.haxe.org/t/extern/}{extern tag}.

\subsection{Using Untyped JavaScript}
\label{target-javascript-injection}

In Haxe, it is possible to call an exposed function thanks to the \expr{untyped} keyword. This can be useful in some cases if we don't want to write externs. Anything untyped that is valid syntax will be generated as it is.

\begin{lstlisting}
untyped window.trackEvent("page1");  
\end{lstlisting}

See also \Fullref{target-javascript-untyped} to inject raw JavaScript.

\subsection{JavaScript untyped functions}
\label{target-javascript-untyped}

These functions allow to access specific JavaScript platform features. It works only when the Haxe compiler is targeting JavaScript and should always be prefixed with \expr{untyped}. 

\emph{Important note:} Before using these functions, make sure there is no alternative available in the Haxe Standard Library. The resulting syntax can not be validated by the Haxe compiler, which may result in invalid or error-prone code in the output.

\paragraph{\expr{untyped __js__(expr, params)}}
\tref{Injects raw JavaScript expressions}{target-javascript-injection}. It's allowed to use \ic{\{0\}}, \ic{\{1\}}, \ic{\{2\}} etc in the expression and use the rest arguments to feed Haxe fields. The Haxe compiler will take care of the surrounding quotes if needed. The function can also return values.

\begin{lstlisting}
untyped __js__('alert("Haxe is great!")');
// output: alert("Haxe is great!");

var myMessage = "Haxe is great!";
untyped __js__('alert({0})', myMessage);
// output: 
//	var myMessage = "Haxe is great!";
//	alert(myMessage);

var myVar:Bool = untyped __js__('confirm({0})', "Are you sure?");
// output: var myVar = confirm("Are you sure?");

var hexString:String = untyped __js__('({0}).toString({1})', 255, 16);
// output: var hexString = (255).toString(16);
\end{lstlisting}

\paragraph{\expr{untyped __instanceof__(o,cl)}} 
Same as \ic{o instanceof cl} in JavaScript.

\begin{lstlisting}
var myString = new String("Haxe is great");
var isString = untyped __instanceof__(myString, String);
output: var isString = (myString instanceof String);
\end{lstlisting}

\paragraph{\expr{untyped __typeof__(o)}} 
Same as \ic{typeof o} in JavaScript.

\begin{lstlisting}
var isNodeJS = untyped __typeof__(window) == null;
output: var isNodeJS = typeof(window) == null;
\end{lstlisting}

\paragraph{\expr{untyped __strict_eq__(a,b)}} 
Same as \ic{a === b}  in JavaScript, tests on \href{https://developer.mozilla.org/en-US/docs/Web/JavaScript/Equality_comparisons_and_sameness}{strict equality} (or "triple equals" or "identity").

\begin{lstlisting}
var a = "0";
var b = 0;
var isEqual = untyped __strict_eq__(a, b);
output: var isEqual = ((a) === b);
\end{lstlisting}

\paragraph{\expr{untyped __strict_neq__(a,b)}} 
Same as \ic{a !== b}  in JavaScript, tests on negative strict equality.

\begin{lstlisting}
var a = "0";
var b = 0;
var isntEqual = untyped __strict_neq__(a, b);
output: var isntEqual = ((a) !== b);
\end{lstlisting}

\paragraph{Expression injection} 

In some cases it may be needed to inject raw JavaScript code into Haxe-generated code. With the \expr{__js__} function we can inject pure JavaScript code fragments into the output. This code is always untyped and can not be validated, so it accepts invalid code in the output, which is error-prone.
This could, for example, write a JavaScript comment in the output.

\begin{lstlisting}
untyped __js__('// haxe is great!');
\end{lstlisting}

A more useful demonstration would be to call a function and pass  arguments using the \expr{__js__} function. This example illustrates how to call this function and how to pass parameters. Note that the \emph{code interpolation} will wrap the quotes around strings in the generated output.

\begin{lstlisting}
// Haxe code:
var myVar = untyped __js__('myObject.myJavaScriptFunction({0}, {1})', "Mark", 31);
\end{lstlisting}

This will generate the following JavaScript code:
\begin{lstlisting}
// JavaScript Code
var myVar = myObject.myJavaScriptFunction("Mark", 31);
\end{lstlisting}


\subsection{JavaScript target Metadata}
\label{target-javascript-metadata}

This is the list of JavaScript specific metadata. For more information, see also the complete list of all \tref{Haxe built-in metadata}{cr-metadata}.

\begin{center}
\begin{tabular}{| l | l |}
	\hline
	\multicolumn{2}{|c|}{JavaScript metadata} \\ \hline
	Metadata &  Description \\ \hline
	@:expose \_(?Name=Class path)\_  &  Makes the class available on the \expr{window} object or \expr{exports} for node.js  \\
	@:jsRequire  &  Generate javascript module require expression for given extern \\
	@:selfCall  &  Translates method calls into calling object directly \\
\end{tabular}
\end{center}

\subsection{Exposing Haxe classes for JavaScript}
\label{target-javascript-expose}

It is possible to make Haxe classes or static fields available for usage in plain JavaScript. 
To expose, add the \expr{@:expose} metadata to the desired class or static fields.

This example exposes the Haxe class \ic{MyClass}.

\haxe{assets/ClassExpose.hx}

It generates the following JavaScript output:

\begin{lstlisting}
(function ($hx_exports) { "use strict";
var MyClass = $hx_exports.MyClass = function(name) {
	this.name = name;
};
MyClass.prototype = {
	foo: function() {
		return "Greetings from " + this.name + "!";
	}
};
})(typeof window != "undefined" ? window : exports);
\end{lstlisting}

By passing globals (like \ic{window} or \ic{exports}) as parameters to our anonymous function in the JavaScript module, it becomes available which allows to expose the Haxe generated module.

In plain JavaScript it is now possible to create an instance of the class and call its public functions.

\begin{lstlisting}
// JavaScript code
var instance = new MyClass('Mark');
console.log(instance.foo()); // logs a message in the console
\end{lstlisting}

The package path of the Haxe class will be completely exposed. To rename the class or define a different package for the exposed class, use \expr{@:expose("my.package.MyExternalClass")}

\paragraph{Shallow expose}

When the code generated by Haxe is part of a larger JavaScript project and wrapped in a large closure it is not always necessary to expose the Haxe types to global variables.
Compiling the project using \ic{-D shallow-expose} allows the types or static fields to be available for the surrounding scope of the generated closure only.

When the code is compiled using \ic{-D shallow-expose}, the generated output will look like this:

\begin{lstlisting}
var $hx_exports = $hx_exports || {};
(function () { "use strict";
var MyClass = $hx_exports.MyClass = function(name) {
	this.name = name;
};
MyClass.prototype = {
	foo: function() {
		return "Greetings from " + this.name + "!";
	}
};
})();
var MyClass = $hx_exports.MyClass;
\end{lstlisting}

In this pattern, a var statement is used to expose the module; it doesn't write to the \ic{window} or \ic{exports} object. 

\subsection{Loading extern classes using "require" function}
\label{target-javascript-require}
\since{3.2.0}

Modern \target{JavaScript} platforms, such as Node.js provide a way of loading objects
from external modules using the "require" function. Haxe supports automatic generation
of "require" statements for \expr{extern} classes.

This feature can be enabled by specifying \expr{@:jsRequire} metadata for the extern class. If our \expr{extern} class represents a whole module, we pass a single argument to the \expr{@:jsRequire} metadata specifying the name of the module to load:

\haxe{assets/JSRequireModule.hx}

In case our \expr{extern} class represents an object within a module, second \expr{@:jsRequire} argument specifies an object to load from a module:

\haxe{assets/JSRequireObject.hx}

The second argument is a dotted-path, so we can load sub-objects in any hierarchy.

If we need to load custom JavaScript objects in runtime, a \expr{js.Lib.require} function can be used. It takes \expr{String} as its only argument and returns \expr{Dynamic}, so it is advised to be assigned to a strictly typed variable.

\section{Flash}
\label{target-flash}
\state{NoContent}

\subsection{Getting started with Haxe/Flash}
\label{target-flash-getting-started}

Developing Flash applications is really easy with Haxe. Let's look at our first code sample.
This is a basic example showing most of the toolchain. 

Create a new folder and save this class as \ic{Main.hx}.

\begin{lstlisting}
import flash.Lib;
import flash.display.Shape;
class Main {
    static function main() {
        var stage = Lib.current.stage;
        
        // create a center aligned rounded gray square
        var shape = new Shape();
        shape.graphics.beginFill(0x333333);
		shape.graphics.drawRoundRect(0, 0, 100, 100, 10);
		shape.x = (stage.stageWidth - 100) / 2;
		shape.y = (stage.stageHeight - 100) / 2;
		
		stage.addChild(shape);
    }    
}
\end{lstlisting}

To compile this, either run the following from the command line:

\begin{lstlisting}
haxe -swf main-flash.swf -main Main -swf-version 15 -swf-header 960:640:60:f68712
\end{lstlisting}

Another possibility is to create and run (double-click) a file called \ic{compile.hxml}. In this example the hxml file should be in the same directory as the example class.

\begin{lstlisting}
-swf main-flash.swf
-main Main
-swf-version 15
-swf-header 960:640:60:f68712
\end{lstlisting}

The output will be a main-flash.swf with size 960x640 pixels at 60 FPS with an orange background color and a gray square in the center.

\paragraph{Display the Flash}

Run the SWF standalone using the \href{https://www.adobe.com/support/flashplayer/downloads.html}{Standalone Debugger FlashPlayer}. 

To display the output in a browser using the Flash plugin, create an HTML-document called \ic{index.html} and open it.

\begin{lstlisting}
<!DOCTYPE html>
<html>
	<body>
		<embed src="main-flash.swf" width="960" height="640">
	</body>
</html>
\end{lstlisting}

\paragraph{More information}

\begin{itemize}
	\item \href{http://api.haxe.org/flash/}{Haxe Flash API docs}
	\item \href{http://help.adobe.com/en_US/FlashPlatform/reference/actionscript/3/}{Adobe Livedocs}
\end{itemize}

\subsection{Embedding resources}
\label{target-flash-resources}

Embedding resources is different in Haxe compared to ActionScript 3. Instead of using \ic{\[embed\]} (AS3-metadata) use \tref{Flash specific compiler metadata}{target-flash-metadata} like \ic{@:bitmap}, \ic{@:font}, \ic{@:sound} or \ic{@:file}.

\begin{lstlisting}
import flash.Lib;
import flash.display.BitmapData;
import flash.display.Bitmap;

class Main {
  public static function main() {
    var img = new Bitmap( new MyBitmapData(0, 0) );
    Lib.current.addChild(img);
  }
}

@:bitmap("relative/path/to/myfile.png") 
class MyBitmapData extends BitmapData { }
\end{lstlisting}

\subsection{Using external Flash libraries}
\label{target-flash-external-libraries}

To embed external \ic{.swf} or \ic{.swc} libraries, use the following \href{http://haxe.org/documentation/introduction/compiler-usage.html}{compilation options}:

\begin{description}
	\item[\expr{-swf-lib <file>}] Embeds the SWF library in the compiled SWF.
	\item[\expr{-swf-lib-extern <file>}] Adds the SWF library for type checking but doesn't include it in the compiled SWF.
\end{description}

The standard compilation options also provide more Haxe sources to be added to the project:

\begin{itemize}
	\item To add another class path use \expr{-cp <directory>}.
	\item To add a \tref{Haxelib library}{haxelib} use \expr{-lib <library-name>}.
	\item To force a whole package to be included in the project, use \expr{--macro include('mypackage')} which will include all the classes declared in the given package and subpackages. 
\end{itemize}

\subsection{Flash target Metadata}
\label{target-flash-metadata}

This is the list of Flash specific metadata. For a complete list see \tref{Haxe built-in metadata}{cr-metadata}.

\begin{center}
\begin{tabular}{| l | l |}
	\hline
	\multicolumn{2}{|c|}{Flash metadata} \\ \hline
	Metadata &  Description  \\ \hline
	@:bind  &  Override Swf class declaration \\
	@:bitmap \_(Bitmap file path)\_  &  \_Embeds given bitmap data into the class (must extend \expr{flash.display.BitmapData}) \\
	@:debug  &  Forces debug information to be generated into the Swf even without \expr{-debug} \\
	@:file(File path)  &  Includes a given binary file into the target Swf and associates it with the class (must extend \expr{flash.utils.ByteArray}) \\
	@:font \_(TTF path Range String)\_  &  Embeds the given TrueType font into the class (must extend \expr{flash.text.Font}) \\
	@:getter \_(Class field name)\_  &  Generates a native getter function on the given field  \\
	@:noDebug &  Does not generate debug information into the Swf even if \expr{-debug} is set \\
	@:ns  &  Internally used by the Swf generator to handle namespaces \\
	@:setter \_(Class field name)\_  &  Generates a native setter function on the given field \\
	@:sound \_(File path)\_  &  Includes a given \_.wav\_ or \_.mp3\_ file into the target Swf and associates it with the class (must extend \expr{flash.media.Sound}) \\
\end{tabular}
\end{center}

\section{Neko}
\label{target-neko}

\section{PHP}
\label{target-php}
\state{NoContent}

\subsection{Getting started with Haxe/PHP}
\label{target-php-getting-started}

To get started with Haxe/PHP, create a new folder and save this class as \ic{Main.hx}.

\haxe{assets/HelloPHP.hx}

To compile, either run the following from the command line:

\begin{lstlisting}
haxe -php bin -main Main
\end{lstlisting}

Another possibility is to create and run (double-click) a file called \ic{compile.hxml}. In this example the hxml file should be in the same directory as the example class.

\begin{lstlisting}
-php bin
-main Main
\end{lstlisting}

The compiler outputs in the given \emph{bin}-folder, which contains the generated PHP classes that prints the traced message when you run it. The generated PHP-code runs for version 5.1.0 and later.

\paragraph{More information}

\begin{itemize}
	\item \href{http://api.haxe.org/php/}{Haxe PHP API docs}
	\item \href{http://php.net/docs.php}{PHP.net Documentation}
	\item \href{http://phptohaxe.haqteam.com/code.php}{PHP to Haxe tool}
\end{itemize}


\subsection{PHP untyped functions}
\label{target-php-untyped}

These functions allow to access specific PHP platform features. It works only when the Haxe compiler is targeting PHP and should always be prefixed with \expr{untyped}. 

\emph{Important note:} Before using these functions, make sure there is no alternative available in the Haxe Standard Library. The resulting syntax can not be validated by the Haxe compiler, which may result in invalid or error-prone code in the output.

\paragraph{\expr{untyped __php__(expr)}}
Injects raw PHP code expressions. It's possible to pass fields from Haxe source code using \tref{String Interpolation}{lf-string-interpolation}.

\begin{lstlisting}
var value:String = "test";
untyped __php__("echo '<pre>'; print_r($value); echo '</pre>';");
// output: echo '<pre>'; print_r('test'); echo '</pre>';
\end{lstlisting}

\paragraph{\expr{untyped __call__(function, arg, arg, arg...)}}
Calls a PHP function with the desired number of arguments and returns what the PHP function returns.

\begin{lstlisting}
var value = untyped __call__("array", 1,2,3); 
// output returns a NativeArray with values [1,2,3]
\end{lstlisting}

\paragraph{\expr{untyped __var__(global, paramName)}}
Get the values from global vars. Note that the dollar sign in the Haxe code is omitted.

\begin{lstlisting}
var value : String = untyped __var__('_SERVER', 'REQUEST_METHOD')  
// output: $value = $_SERVER['REQUEST_METHOD']
\end{lstlisting}

\paragraph{\expr{untyped __physeq__(val1, val2)}}
Strict equals test between the two values. Returns a \type{Bool}.

\begin{lstlisting}
var isFalse = untyped __physeq__(false, value);
// output: $isFalse = false === $value;
\end{lstlisting}


\section{C++}
\label{target-cpp}
\state{NoContent}

\subsection{Getting started with Haxe/C++}
\label{target-cpp-getting-started}
The c++ target uses various c++ compilers, which are assumed to be already installed on the system, to create native executables or libraries.  The compilation happens in two phases.  Firstly, the Haxe compiler generates source, header and build files in an output directory.  Secondly, the "hxcpp" \tref{Haxelib library}{haxelib} is invoked to run the system compilers and linkers required to generate the ultimate result.

\paragraph{Prerequisites}
Before you can use the C++ target, you need in install
\begin{itemize}
	\item hxcpp, e.g. \ic{haxelib install hxcpp}.
	\item A system or cross-compiler
\end{itemize}

\paragraph{System Compilers}
System compilers are supported on the three primary operating systems - Mac, Linux, and Windows.  On Mac, it is recommended that you install the latest Xcode from the Mac App Store.  On Linux, it is recommended that you use the system package manager to install the compilers and on Windows, Microsoft Visual Studio is recommended.  On Windows, you can also use gcc-based compilers.  A minimal distribution is included in a \tref{Haxelib library}{haxelib}, and can be installed with \ic{haxelib install minimingw}.

\paragraph{Cross Compilers}
Hxcpp can be used to compile for non-host architectures if you have a suitable cross-compiler installed.  The compilers are usually supplied in the form of a Software Development Kits (SDK), or in the case of iOS devices, come with the system compiler (Xcode).  Selecting which compiler to use is achieved by defining particular variables in the hxcpp build environment.  Note that the hxcpp build tool is only responsible for producing a native executable or a native library (static or dynamic), not the complete bundling and packaging of assets and meta-data that is typically required for mobile devices.  Additional Haxe libraries can be used for this task.

\subsection{The Hxcpp Build Environment}
\label{target-cpp-build-environment}
The hxcpp build environment can be controlled by setting, or defining, variables in key-value pairs. Defining a key without a particular value is usually enough to trigger the desired behaviour, and these keys are often referred to as "defines".

\paragraph{Defining From the Command Line}
The easiest way to change the hxcpp build environment is to pass the defines though the Haxe command line using \ic{-D}.  Key-value pairs can also be passed this way, e.g.:

\ic{ haxe -main Main -cpp cpp -D android -D static_link  -D PLATFORM=android-9 }

Here, \ic{android} is defined - this causes hxcpp to cross-compile to android, \ic{static_link} is defined, which causes a static (rather than dynamic) library to be created and the android target platform is set to a particular value.  This platform information is passed on to the SDK so it can generate the appropriate code.

Advanced users can add additional defines to the system at compile time using macros.  These definitions will also be passed on to the hxcpp build tool.

\paragraph{Defining From the System Environment Variables}
The hxcpp build tool will import all the system environment variables, so you can configure the processes using the system like:

\ic{ setenv HXCPP_VERBOSE }

If you are running Haxe though an IDE, some care must be taken with environment variables since the variables may be read once from the environment in which the IDE was started, rather than changing when the variables are changed on the system.

\paragraph{Defining From .hxcpp_config.xml}
The hxcpp build tool parses several "build files".  These files are in a basic xml file format, and can be used to set, or conditionally set, configuration variables.  As part of the build process, the \ic{.hxcpp_config.xml} file, known as the configuration file, will be read (twice).  This file is located in the user's home directory (or user's profile directory on windows) and is the best place to configure variables that are specific to the system that rarely change, such as the location of the cross-compiler SDKs.  A placeholder file will be generated the first time hxcpp is run by a user on the machine.  You can see the exact location of this file in the build log if you compile with the \ic{HXCPP_VERBOSE} define.

The configuration file is read twice.  The first time the \ic{"vars"} section is read early in the configuration process, and is used to set up and configure the location of the compilers.  The second time, the \ic{"exes"} section is read near the end of the processes and allows modification of the compiler or build process based on all the available information.  See \Fullref{target-cpp-file-format} for details on the file format.

\paragraph{Defining Via @:buildXml Metadata}
Configuration data can be injected into the "build.xml" file that is created by Haxe.  This is done attaching metadata to a Haxe class that is directly or indirectly included in the build.  For example:
\begin{lstlisting}
@:buildXml("
<target id='haxe'>
  <lib name='${haxelib:nme}/lib/${BINDIR}/libnme${LIBEXTRA}${LIBEXT}'/>
</target>
<include name='${haxelib:nme}/lib/NmeLink.xml'/>
")
@:keep class StaticNme
{
  ...
\end{lstlisting}

This metadata is best for adding libraries or include paths to the build.
Some notes:
\begin{itemize}
	\item The @:keep metadata is added to prevent dead-code-elimination from removing this class
	\item Quoting can be a bit tricky - here the double-quotes are used for Haxe, and the single quotes are added to the build.xml.
	\item Injecting a single "include" command is a good way to manage the quoting issue.
	\item Knowledge of the build system is required to get this right. See \Fullref{target-cpp-file-format} for details on the file format.
	\item The build.xml file is read after the choice of compiler has been made, so it is generally not suitable for configuring the compiler.
\end{itemize}


\subsection{build.xml File Format}
\label{target-cpp-file-format}
The build.xml file format is designed to address the very specific goal of generating compiled output as fast as possible on a wide variety of compilers.  Therefore, at its core, it is simply a list of files to compile, and their dependencies, and a list of compiler exes and flags required to get the job done.  Conditional variable setting is added to support the great variety of compilers and use cases encountered by hxcpp.  However, much of the logic required to orchestrate the compiler sequences, such as precompiling headers, is defined in Haxe and implemented in the build tool itself, rather than being scriptable.

A hxcpp build file consists of a series of commands and data structures in xml format.  The commands are executed immediately as they are encountered, while the data structures are accumulated and used after all the build files have been parsed.

\paragraph{Running a build.xml File With Hxcpp}
When you compile a Haxe program with hxcpp, the Haxe compiler will normally run the hxcpp build tool on the generated build.xml file automatically.  You can, however, prevent this by adding \ic{-D no-compilation} to the Haxe command line.

You can run the hxcpp build tool on your own build files using

\ic{haxelib run hxcpp myfile.xml [-Ddefine] [-Dkey=value]}

Note the lack of space between the "-D" and the variable name.

A minimal build.xml file consists of an xml container and a dummy default target, like:
\begin{lstlisting}
<xml>
   <echo value="Hello!" />
   <target id="default" />
</xml>
\end{lstlisting}

\paragraph{Conditions, Substitutions and Sections}
Most elements in the hxcpp xml file allow a common syntax for dynamic configuration.

Xml elements can contain conditional "if" and/or "unless" values.  These conditions are evaluated at parse time and the entire element will be skipped if the condition fails.  For example adding this lines:
\begin{lstlisting}
<xml>
  <echo value="Hello A" if="A" />
  <echo value="Hello A && B" if="A B" />
  <echo value="Hello A || B" if="A || B" />
  <echo value="Hello !A" unless="A" />
  <echo value="Hello !(A && B)" unless="A B" />
  <echo value="Hello !(A || B)" unless="A || B" />
  <echo value="Never Never" if="A" unless="A" />
  <target id="default" />
</xml>
\end{lstlisting}

and running:

\ic{haxelib run hxcpp myfile.xml -DA}

shows how the logic depends on whether or not A or B has been defined.

Sections can be used to group commands together based on a common condition.  They can also be used to include only part of another xml file, but this technique is currently only used when parsing the .hxcpp_config.xml file.  For example:

\begin{lstlisting}
  <section if="C" >
    <echo value="I See" />
    <echo value="You" />
  </section>
\end{lstlisting}

The xml attribute values can be substituted with variable values using dollars-brace syntax.  Using a colon allows a function-call to be substituted, e.g.:

\begin{lstlisting}
  <echo value="some var = ${SOME_VAR}" />
  <echo value="${haxelib:hxcpp}/include" />
\end{lstlisting}

\subsection{Using C++ Defines}
\label{target-cpp-defines}
\begin{itemize}
	\item ANDROID_HOST
	\item ANDROID_NDK_DIR
	\item ANDROID_NDK_ROOT
	\item BINDIR
	\item DEVELOPER_DIR
	\item HXCPP
	\item HXCPP_32
	\item HXCPP_COMPILE_CACHE
	\item HXCPP_COMPILE_THREADS
	\item HXCPP_CONFIG
	\item HXCPP_CYGWIN
	\item HXCPP_DEPENDS_OK
	\item HXCPP_EXIT_ON_ERROR
	\item HXCPP_FORCE_PDB_SERVER
	\item HXCPP_M32
	\item HXCPP_M64
	\item HXCPP_ARMV6
	\item HXCPP_ARMV7
	\item HXCPP_ARMV7S
	\item HXCPP_ARM64
	\item HXCPP_X86
	\item HXCPP_MINGW
	\item HXCPP_MSVC
	\item HXCPP_MSVC_CUSTOM
	\item HXCPP_MSVC_VER
	\item HXCPP_NO_COLOR
	\item HXCPP_NO_COLOUR
	\item HXCPP_VERBOSE
	\item HXCPP_WINXP_COMPAT
	\item HXCPP_API_LEVEL
	\item HXCPP_DEBUG_LINK
	\item DHXCPP_STACK_TRACE
	\item HXCPP_STACK_LINE
	\item HXCPP_CHECK_POINTER
	\item HXCPP_DEBUGGER
	\item HXCPP_CPP11
	\item HXCPP_STRICT_CASTS
	\item HXCPP_VISIT_ALLOCS
	\item HXCPP_WINXP_COMPAT
	\item HXCPP_OPTIMIZE_FOR_SIZE
	\item HXCPP_GC_MOVING
	\item IPHONE_VER
	\item LEGACY_MACOSX_SDK
	\item LEGACY_XCODE_LOCATION
	\item MACOSX_VER
	\item MSVC_VER
	\item NDKV
	\item NO_AUTO_MSVC
	\item PLATFORM
	\item QNX_HOST
	\item QNX_TARGET
	\item TOOLCHAIN_VERSION
	\item USE_GCC_FILETYPES
	\item USE_PRECOMPILED_HEADERS
	\item android
	\item apple
	\item blackberry
	\item cygwin
	\item dll_import
	\item dll_import_include
	\item dll_import_link
	\item emcc
	\item emscripten
	\item gph
	\item hardfp
	\item haxe_ver
	\item ios
	\item iphone
	\item iphoneos
	\item iphonesim
	\item linux
	\item linux_host
	\item mac_host
	\item macos
	\item mingw
	\item rpi
	\item simulator
	\item tizen
	\item toolchain
	\item webos
	\item windows
	\item windows_host
	\item winrt
	\item xcompile
\end{itemize}

\subsection{Using C++ Pointers}
\label{target-cpp-pointers}


\section{Cppia}
\label{target-cppia}

\section{Java}
\label{target-java}

\section{C\#}
\label{target-cs}

\section{Python}
\label{target-python}



\section{Lua}
\label{target-lua}
\state{NoContent}

\subsection{Getting started with Haxe/Lua}
\label{target-lua-getting-started}

To get started with Haxe for Lua, it's necessary to pick a Lua version and install
dependencies.  All versions of Lua are supported, but may require different
libraries.  Lua 5.1, 5.2, 5.3, and LuaJIT 2.0 and 2.1 (beta) are supported.

Lua is a very lightweight language that ships with a much smaller  feature set
than Haxe.  In some cases (e.g. regex), it's necessary to install supplementary
libraries that are used to support basic Haxe functionality.

In order to cover all dependencies, it is recommended to install and use
\href{https://github.com/luarocks/luarocks/wiki/Download}{LuaRocks}.  However,
if you do not utilize relevant behavior (e.g. regex) on a given platform,
or if you are using an embedded Lua client, then it is not necessary to
install any packages for basic language functionality.

With LuaRocks, install the following packages:

\begin{lstlisting}
luarocks install lrexlib-pcre
luarocks install environ
luarocks install luasocket
luarocks install luv
\end{lstlisting}

On Lua 5.1, install the bitops library:
\begin{lstlisting}
luarocks install luabitop
\end{lstlisting}

On Lua 5.3, install the bit32 library instead:
\begin{lstlisting}
luarocks install bit32
\end{lstlisting}

When developing for multiple Lua versions, it is recommended to use
the Python package \href{https://github.com/mpeterv/hererocks}{hererocks}.

With Lua installed, it is possible to write a simple command line application.

Create a new folder, and save this class as \ic{Main.hx}.

\begin{lstlisting}
class Main {
    static function main() {
        trace("hello world");
    }
}
\end{lstlisting}

To compile, run the following:
\begin{lstlisting}
haxe -lua out.lua -main Main
\end{lstlisting}

\paragraph{More information}
\begin{itemize}
	\item \href{https://www.lua.org/}{Lua Homepage}
	\item \href{http://luajit.org/}{LuaJIT Homepage}
\end{itemize}

\subsection{Using external Lua libraries}
\label{target-lua-external-libraries}

The \tref{extern functionality}{lf-externs} in Haxe provides a way of declaring type signatures
for native Lua libraries.

\subsection{Version flags}
\label{target-lua-flags}

The Lua target enables the following define flags for the Haxe complier:

\begin{description}
	\item[\expr{-D lua_ver}] Enable special features for a specific Lua version. Currently, this flag will enable extern methods that are specific to certain versions (e.g. table.pack in Lua > 5.2).
	\item[\expr{-D luajit}] Enable special features for LuaJIT.  Currently this flag will enable the \ic{jit} and \ic{ffi} module namespaces.
\end{description}

\subsection{Multireturns}
\label{target-lua-multireturns}

Lua allows for multiple values to be returned from a given function.  Haxe
does not support this by default, but can allow extern definitions to reference
multireturn values through the \ic{@:multiReturn} metadata.

\begin{lstlisting}
class Main {
    static function main() {
        var strfind = NativeString.find("foobar", "bar");
        trace(strfind.begin);
        trace(strfind.end);
    }
}

@:native("string")
extern class NativeString {
	public static function find(str : String, target : String): StringFind;
}

@:multiReturn extern class StringFind {
	var begin : Int;
	var end : Int;
}
\end{lstlisting}

This example has three parts:

\begin{itemize}
	\item The extern class \ic{NativeString} which is an extern for the base \ic{string} library in Lua.
	\item The \ic{StringFind} class which is marked as \ic{@:multiReturn} that describes the return values.
	\item The Main class that invokes the string method as a simple example.
\end{itemize}

The multireturn behavior in Haxe is optimized based on usage.  If fields are
only accessed directly, the Haxe compiler will allocate the multireturn to
individual variables.  But, if you pass or assign the entire multireturn value,
the compiler will wrap all values into a table object.  This operation ensures
that multireturn variable handling only carries as much overhead as needed.



\section{HashLink}
\label{target-hl}
